\chapter{Validazione}
%\myChapter{Validazione}
%%%%%%%%%%%%%%%%%%%%%%%%%%% INTRODUZIONE

La validazione � stata fatta mediante un semplice meccanismo di per fare una rapida verifica di funzionamento del lavoro totale. Raccogliendo i dati in un ambiente controllato, si possono stimare con accuratezza la  cadenza (passi/minuto) e la velocit� di deambulazione (m/sec). Una volta ottenuti i due parametri, questi si possono correlare tramite una regressione lineare stimare la distanza percorsa.  Allo stesso momento, si pu� usare un dispositivo GPS\footnote{Global Positioning System} per stimare la distanza percorsa e confrontare i due risultati. Ci si aspetta di avere degli errori considerevoli sull'approssimazione della distanza del GPS, in quanto ha un'accuratezza di ($\pm$ 15m).

%%%%%%%%%%%%%%%%%%%%%%%%%%% 



%%%%%%%%%%%%%%%%%%%%%%%%%%% DESCRIZIONE A GRANDI LINEE DELLA VALIDAZIONE
%%%%%%%%%%%%%%%%%%%%%%%%%%% ACQUISIZIONE DATI (ESPERIMENTO)%%%%%%%%%%%%% LAB

\subsection{Dati}
La raccolta dei dati su cui fare la validazione, � stata fatta su una persona. All'individuo � stata applicata la IMU sul collo del piede destro, dopodich� l'individuo � stato sottoposto a 7 sessioni di cammino su tapis roulant senza inclinazioni. Le 7 sessioni di cammino andavano dai 2 agli 8 Km/h, ciascuno della durata di 1,30 min.\\
I dati sono stati raccolti mediate l'applicazione Android istallata su uno smartphone Samsung Galaxy S2.


La segmentazione di ciascuna sessione di cammino � stata salvata su un file, dal quale � stata calcolata la cadenza.
Per calcolare la cadenza � sufficiente conoscere contare il numero di cicli di deambulazione al minuto. Quello che abbiamo fatto � calcolare la cadenza usando ognuno dei 4 eventi, determinati dalla segmentazione, per calcolare la cadenza. Facendo la media dei 4 risultati abbiamo ottenuto una buona stima della cadenza. Questa operazione � stata ripetuta per ciascuna velocit�. \\
Abbiamo ottenuto una distribuzione di valori di cadenza per ciascuna velocit�. A questo punto applicando una regressione lineare abbiamo trovato una relazione tra cadenza e velocit�.\\
%%%%%%%%%%%%% FUORI

Dopo aver caricato bene le batterie dello Smartphone, siamo usciti dal laboratorio ed abbiamo fattola la prova per le strade di Pontedera. Il soggetto degli esperimenti, ha indossato anche in questo caso la IMU sul collo del piede. La velocit� di cammino, qui � stata scelta del soggetto. Sullo Smartphone � stato attivato il segmentatore e su un altro il GPS. La distanza ed il percorso fatto tra i due punti � stata stimata dal GPS.

I dati del GPS, sono riferiti a coordinate geografiche\footnote{longitudine, latitudine} che sono state convertite per poter calcolare distanze in metri.
 
 