\chapter{Conclusioni}
In questo lavoro � stato affrontato il problema della segmentazione del segnale giroscopico (con un giroscopio posizionato sul collo del piede) prodotto della deambulazione umana, mediante l'algoritmo di Viterbi, su un'HMM minimale a 4 stati, addestrato con Apprendimento Supervisionato su un \textit{training set} etichettato mediante stereofotogrammetria Vicon\tm. Il sistema di segmentazione � stato quindi implementato con una versione in linea dell'algoritmo di Viterbi su uno \textit{Smartphone} Android. Infine � stato ideato un meccanismo di verifica del funzionamento del sistema, basato sul confronto della velocit� calcolata mediante il sistema e quella calcolata da un dispositivo GPS.\\
Il sistema fornisce un surrogato inerziale temporaneo al sistema GPS per quanto riguarda il calcolo della distanza. Vale a dire, che dato che il sistema GPS pu� incappare in problemi di assenza di segnale, il sistema pu� subentrare nella misurazione della distanza per brevi periodi di tempo (per adesso un paio di minuti) ed essere sostituito dal GPS, quando il segnale � nuovamente disponibile. Il vantaggio della misurazione inerziale della distanza sul modello di misurazione mediante GPS � la sua autonomia, infatti funziona anche al chiuso. Il suo svantaggio � invece il fatto che sia inaccurata. Sviluppi futuri del lavoro dovranno pensare al miglioramento del modello con HMM ad emissioni a misture di densit� gaussiane multivariate. Ci� migliorerebbe sicuramente le prestazioni del sistema sulla capacit� di segmentazione del sistema.\\
Un secondo sviluppo futuro del lavoro � quello di progettare con lo stesso meccanismo modelli di semplici attivit� quotidiane, come la corsa, salire dei gradini, sedersi, alzarsi ed aliti simili esempi ed addestrare una HMM a riconoscere ciascuna delle attivit�.

 