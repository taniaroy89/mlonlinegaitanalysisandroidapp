%% Based on a TeXnicCenter-Template by Tino Weinkauf.
%%%%%%%%%%%%%%%%%%%%%%%%%%%%%%%%%%%%%%%%%%%%%%%%%%%%%%%%%%%%%

%%%%%%%%%%%%%%%%%%%%%%%%%%%%%%%%%%%%%%%%%%%%%%%%%%%%%%%%%%%%%
%% HEADER
%%%%%%%%%%%%%%%%%%%%%%%%%%%%%%%%%%%%%%%%%%%%%%%%%%%%%%%%%%%%%
\documentclass[a4paper,twoside,10pt]{report}

\usepackage[USenglish]{babel} %francais, polish, spanish, ...
\usepackage[T1]{fontenc}
\usepackage[ansinew]{inputenc}

\usepackage{lmodern} %Type1-font for non-english texts and characters

\usepackage{graphicx} 
\usepackage{amsmath}
\usepackage{amsthm}
\usepackage{amsfonts}

\usepackage{geometry}
\geometry{
	a4paper,
	ignoremp,
	bindingoffset = 1.5cm, 
	textwidth     = 12cm,
	textheight    = 20cm,
	lmargin       = 2cm, % left margin
	rmargin       = 2cm, % left margin
	tmargin       = 2cm,    % top margin 
	bmargin 			= 3.5cm
}


\def\tm{\leavevmode\hbox{$\rm {}^{TM}$}}

\begin{document}

\pagestyle{empty} %No headings for the first pages.


%% Title Page %%%%%%%%%%%%%%%%%%%%%%%%%%%%%%%%%%%%%%%%%%%%%%%
%% ==> Write your text here or include other files.


\title{Machine Learning Methods for online Gait Analysis:\\
Developement and Validation of a Smartphone Application.\\ 
Experimental Validation Procedure}

\author{Andrea Mannini, Ahadu Tsegaye}

\maketitle

%\tableofcontents 
%\cleardoublepage 

\pagestyle{plain}



%%%%%%%%%%%%%%%%%%%%%%%%%%%%%%%%%%%%%%%%%%%%%%%%%%%%%%%%%%%%%%%%%%%%%%%%%%%%%%%%%%%%%%%%%%%%%%%%%%
%%%%%%%%%%%%%%%%%%%%%%%%%%%%%%%%%%%%%%%%%%%%%%%%%%%%%%%%%%%%%%%%%%%%%%%%%%%%%%%%%%%%%%%%%%%%%%%%%%
%%%%%%%%%%%%%%%%%%%%%%%%%%%%%%%%%%%%%%%%%%%%%%%%%%%%%%%%%%%%%%%%%%%%%%%%%%%%%%%%%%%%%%%%%%%%%%%%%%
%\begin{chapter}{}
\section{First trial}
We wrapped the IMU on a shoe and went out in the proximity of the laboratory to test if there were problems of any kind. 
We saved the gyroscopic data locally (i.e. on the IMU) by setting an acquisition time of $400\,s$. We used Andrea's Smartphone with a metronome app to set a cadence. We walked for about $500\,m$ and back for about $300\,m$, for a total of about $800\,m$.  We tried to import the data on the IMU to the Smartphone, but then interrupted the procedure, because it was too slow. We imported the data from the IMU to a computer with the WIMU Interface software (written in Visual Basic). We had also acquired GPS data, which we transfered via \textit{DorpBox\tm} to Andrea's laptop.\\
We needed a computer to download the WIMU and GPS data, so we installed the WIMU Interface software on a \textit{Dell Latitude E6500} laptop. Since the bluetooth device was not working we used a Bluetooth USB key. \\
\textbf{Bluetooth USB configuration Procedure}\\
\begin{enumerate}
  \item Activate physical radio button (WiFi and Bluetooth)
	\item Insert Bluetooth USB key
	\item go to Control Panel > Bluetooth Device
	\item Follow instructions
	\item Once connection is obtained, go to Control Panel > Devices > COM 
	\item change to COM 2
\end{enumerate}

At this point we could see the IMU from the laptop 
 



\section{DAY 1}
The WIMU was tied up to a running shoe using the laces of the shoe. \\
We took two Smartphones, one with the app, the other for the metronome. \\

\subsection{Settings} 
\begin{enumerate}
	\item Set sensors: 
		\begin{enumerate}
			\item Gyro frequency  $=\, 100\,Hz$
			\item Acc Bandwidth  $=\, 20\, HZ$
			\item Acc Background $=\, 8\,g$
		\end{enumerate}
	\item Start IMU data logging on Smartphone 1 for time given in tables \ref{tab:FirstDataAcquisitionParameters}, \ref{tab:SecondDataAcquisitionParameters}
	\item Start GPS data logging on Smartphone 1
	\item Set metronome freq on Smartphone 2 with cadence given in tables \ref{tab:FirstDataAcquisitionParameters}, \ref{tab:SecondDataAcquisitionParameters}
\end{enumerate}

\begin{table}[h]
	\centering
		\begin{tabular}{|l|l|}
		  \hline
			\textbf{Cadence} & $100\,bps$\\
			\hline
			\textbf{Time} & $420\,s$\\
			\hline
		\end{tabular}
	\caption{First data acquisition parameters}
	\label{tab:FirstDataAcquisitionParameters}
\end{table}

\begin{table}[h]
	\centering
		\begin{tabular}{|l|l|}
		  \hline
			\textbf{Cadence} & $120\,bps$\\
			\hline
			\textbf{Time} & $600\,s$\\
			\hline
		\end{tabular}
	\caption{Second data acquisition parameters}
	\label{tab:SecondDataAcquisitionParameters}
\end{table}

\subsection{Acquisition Procedure}
Before starting the subject with the WIMU jumped to mark the gyro signal. 
Once set we walked around the running field on the innermost track for $400\,m$. One with the WIMU laced, the other holding the Smartphones.\\ 
Every $100\,m$ a GPS way point was indicated to sub segment single subject data in 4 parts. 

\section{DAY 2}
The software was changed to get the data logging start time (so that the subject did not have to jump to indicate the start of the experiment). The gyroscope frequency was set to $100\,Hz$ by default.\\
The WIMU was laced to a second subject, and 4 walking sessions were performed. Each time 4 files were saved: 

\begin{enumerate}
	\item Data logging time start: [size = 1] 
	\item GPS out: [size = 8 X 318] contains (longitude, latitude,,,,) every $100\,s$
	\item Way point: [size = 2 X 5] contains (longitude, latitude) every $100\,m$
	\item WIMU out: [size = 9 X 40000] (named $place\_SubjectCadence$ eg. $Stadio\_Andrea120$)
\end{enumerate}

\begin{table}[h]
	\centering
		\begin{tabular}{|l|l|}
		  \hline
			\textbf{Cadence} & $120\,bps$\\
			\hline
			\textbf{Time} & $400\,s$\\
			\hline
		\end{tabular}
	\caption{First data acquisition parameters}
	\label{tab:FirstDataAcquisitionParametersAndrea}
\end{table}

\begin{table}[h]
	\centering
		\begin{tabular}{|l|l|}
		  \hline
			\textbf{Cadence} & $110\,bps$\\
			\hline
			\textbf{Time} & $400\,s$\\
			\hline
		\end{tabular}
	\caption{Second data acquisition parameters}
	\label{tab:SecondDataAcquisitionParametersAndrea}
\end{table}

\begin{table}[h]
	\centering
		\begin{tabular}{|l|l|}
		  \hline
			\textbf{Cadence} & $100\,bps$\\
			\hline
			\textbf{Time} & $500\,s$\\
			\hline
		\end{tabular}
	\caption{Third data acquisition parameters}
	\label{tab:ThirdDataAcquisitionParametersAndrea}
\end{table}

\begin{table}[h]
	\centering
		\begin{tabular}{|l|l|}
		  \hline
			\textbf{Cadence} & $90\,bps$\\
			\hline
			\textbf{Time} & $550\,s$\\
			\hline
		\end{tabular}
	\caption{Forth data acquisition parameters}
	\label{tab:ForthDataAcquisitionParametersAndrea}
\end{table}

\end{document}

