\chapter{Sommario}
%\myChapter{Sommario}
\TODO{riscrivere alla fine}

In questo lavoro si affronta il problema dell'analisi della deambulazione umana mediante l'uso di sensori indossabili e dispositivi di computazione portabili smartphone.\\
%Tale studio ha applicazioni importanti in almeno tre settori: in campo medico, nella diagnosi e/o riabilitazione di pazienti affetti da problemi motori; nel campo della robotica e per simulazione del movimento umano ed in campo sportivo per il perfezionamento di tecniche e monitoraggio di allenamenti specifici. 
%
%I metodi con cui � stato affrontato tale problema possono essere suddivisi sia per gli strumenti di misurazione che usano: con sistemi di telecamere o con uso di sensori inerziali (giroscopi, sensori di forza, sensori di campi elettromagnetici ecc); per la contesto: fisico in cui vengono usati metodi di meccanica, contesto di intelligenza artificiale in cui vengono usati metodi di apprendimento automatico. 
%Noi raccogliamo dati usando un giroscopio posizionato sul collo del piede e metodi di Machine Learning per l'analisi dei dati. Pi� precisamente utilizziamo un Modello di Markov Nascosto come classificatore, addestrato su un training set che consiste nei dati del giroscopio etichettati grazie a metodi di tipo classico \cite{walking_features_from_inertials}.
%\end{abstract}