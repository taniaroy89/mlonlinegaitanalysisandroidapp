\chapter{Sommario}
%\myChapter{Sommario}

In questo lavoro si affronta il problema dell'analisi in linea della deambulazione umana mediante metodi di apprendimento automatico e lo sviluppo di un'applicazione per \textit{Smartphone Android}.\\
Una HMM (\textit{Hidden Markov Model}) a quattro stati addestrata in differita su segnali giroscopici provenienti da sessioni di cammino e corsa su un tappeto rullante a diverse velocit� ed inclinazioni, viene usata per segmentare in linea le fasi del cammino grazie ad una versione modificata dell'algoritmo di decodifica di Viterbi.\\
Il giroscopio usato � contenuto in una IMU (\textit{Inertial Measurement Unit}) collocato sul collo del piede ed orientato con l'asse sensibile sul piano mediale laterale.\\
L'applicazione per \textit{Smartphone} permette di controllare la IMU via \textit{Bluetooth}, nonch� di segmentare e visualizzare in linea il segnale giroscopico relativo alla deambulazione. \\
La validazione del sistema viene fatta stimando la velocit� e la distanza percorsa in sessioni di cammino all'aperto. Tali grandezze sono stimate a partire dalla stima della cadenza ottenuta con il modello HMM seguendo la relazione tra cadenza e velocit�, le quali sono state ottenute in sessioni di cammino su tappeto rullante in laboratorio.\\