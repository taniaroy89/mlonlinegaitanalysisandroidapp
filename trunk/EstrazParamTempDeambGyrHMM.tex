	%--------------------------------------------------------------
% thesis.tex 
%--------------------------------------------------------------
% Corso di Laurea in Informatica 
% http://if.dsi.unifi.it/
% @Facolt� di Scienze Matematiche, Fisiche e Naturali
% @Universit� degli Studi di Firenze
%--------------------------------------------------------------
% - template for the main file of Informatica@Unifi Thesis 
% - based on Classic Thesis Style Copyright (C) 2008 
%   Andr\'e Miede http://www.miede.de   
%--------------------------------------------------------------
\documentclass[twoside,openright,titlepage,fleqn,headinclude,11pt,a4paper,BCOR5mm,footinclude]{scrbook}
\usepackage[pdftex,hyperfootnotes=false,pdfpagelabels,pagebackref]{hyperref}	%creates hyperlinks within the document.
\usepackage[pdftex]{graphicx}
\usepackage[dvipsnames]{xcolor}
\usepackage{microtype}
\usepackage[T1]{fontenc}
\usepackage[italian]{babel}	%typographic rules, hyphenations and special characters directly from keyboard
\usepackage[latin1]{inputenc}	%keyboard input encoding
\usepackage[square,numbers]{natbib}	%standard and non-standard bibliographic style files (BibTeX).  
\usepackage[fleqn]{amsmath}	%american mathematica society math symbols
\usepackage{amsfonts}	%collection is a set of miscellaneous TeX fonts 
\usepackage{amssymb}	%symbols
\usepackage{algorithmic}
\usepackage{algorithm}
\usepackage{bm}
\usepackage{url}%http URLs
\usepackage{breakurl}%http URLs that can be broken on multiple lines
\usepackage{soul}
\usepackage[framed]{ntheorem}
\usepackage{framed}
\usepackage{booktabs}
\usepackage{textcase}
\usepackage[automark]{scrpage2} 
\usepackage{titlesec}
%\usepackage{pstricks}
\usepackage[titles]{tocloft}
\usepackage{scrtime}
\usepackage{ifthen}
\usepackage{textcomp}
\usepackage{mparhack}
\usepackage{tabularx}
\usepackage{fixltx2e}
\usepackage{relsize}
\usepackage[smaller]{acronym}
\usepackage{caption}
\usepackage{remreset}
\usepackage{subfig}
\usepackage{listings}
\usepackage{ifpdf}
\usepackage{hyperref}
\usepackage[hyperpageref]{backref}
\usepackage[osf,sc]{mathpazo}
\usepackage{listings}
\usepackage{lstcustom}
\usepackage{bookmark}
\usepackage{setspace}
%\usepackage{index}
\usepackage[toc,page]{appendix}
\usepackage[titles]{tocloft}
\usepackage{multirow}
\usepackage[normalem]{ulem}
\definecolor{lightBlue}{rgb}{0.95,0.95,0.95}
%\lstset{language=Java, backgroundcolor=\color{SpringGreen}, basicstyle=\small, commentstyle=\color{green}}
\definecolor{javared}{rgb}{0.6,0,0} % for strings
\definecolor{javagreen}{rgb}{0.25,0.5,0.35} % comments\definecolor{javapurple}{rgb}{0.5,0,0.35} % keywords
\definecolor{javadocblue}{rgb}{0.25,0.35,0.75} % javadoc

%\definecolor{javabackground}{rgb}{0.25,0.35,0.55}
%\lstset{language=Java, backgroundcolor=\color{superLightBlue}, basicstyle=\small, commentstyle=\color{green}}
%\usepackage[acronym]{glossaries} % make a separate list of acronyms
%\makeglossaries
%--------------------------------------------------------------

\newcommand{\myTitle}{Metodi di Apprendimento Automatico per l'analisi in linea
della Deambulazione: Sviluppo e Validazione di un'applicazione per Smartphone\xspace}
\newcommand{\myEngTitle}{Machine Learning Methods for online Gait Analysis: Developement and Validation of a Smartphone Application}
%fr Machine Learning M�thodes d'analyse temps r�el la marche: D�veloppement et validation sur un Smartphone
%de Methoden des maschinellen Lernens f�r Real-Time-Ganganalyse: Entwicklung und Validierung auf einem Smartphone
%es Machine Learning m�todos de an�lisis de la marcha real de tiempo: Desarrollo y validaci�n en un Smartphone

% use the right myDegree option
%\newcommand{\myDegree}{Corso di Laurea in Informatica\xspace}
\newcommand{\myDegree}{	
	Corso di Laurea Specialistica in Scienze e Tecnologie 
	dell'Informazione\xspace}
\newcommand{\myName}{Ahadu Tsegaye\xspace}
\newcommand{\myProf}{Prof. Angelo Maria Sabatini\xspace}
\newcommand{\myOtherProf}{Prof. Maria Cecilia Verri\xspace}
%\newcommand{\mySupervisor}{Dott. Andrea Mannini\xspace}

%\newcommand{\myOtherProf}{Prof. Frasconi}
\newcommand{\myFaculty}{
	Facolt� di Scienze Matematiche, Fisiche e Naturali\xspace}
\newcommand{\myDepartment}{	Dipartimento di Sistemi e Informatica\xspace}
\newcommand{\myUni}{\protect{	Universit� degli Studi di Firenze}\xspace}
\newcommand{\myLocation}{Firenze\xspace}
\newcommand{\myTime}{Anno Accademico 2010-2011\xspace}
\newcommand{\myVersion}{\small \textcolor{red}{Revisione 23.0} \xspace} %last committed svn version.build
\newcommand{\myShortVersion}{\small \textcolor{red}{v23.0} \xspace}
%-----------------------------------------------------------\hypersetup{pdfstartview=FitW}
\hypersetup{pdfstartview=FitW}
%% Define a new 'leo' style for the package that will use a smaller font.
\makeatletter
\def\url@leostyle{%
  \@ifundefined{selectfont}{\def\UrlFont{\sf}}{\def\UrlFont{\small\ttfamily}}}
\makeatother
%% Now actually use the newly defined style.
\urlstyle{leo}
%--------------------------------------------------------------
\usepackage{dia-classicthesis-ldpkg} 
%--------------------------------------------------------------
% Options for classicthesis.sty:
% tocaligned eulerchapternumbers drafting linedheaders 
% listsseparated subfig nochapters beramono eulermath parts 
% minionpro pdfspacing
\usepackage[eulerchapternumbers,subfig,beramono,eulermath,parts]{classicthesis}
%--------------------------------------------------------------
\newlength{\abcd} % for ab..z string length calculation
% how all the floats will be aligned
\newcommand{\myfloatalign}{\centering} 
\setlength{\extrarowheight}{3pt} % increase table row height
\captionsetup{format=hang,font=small}
%--------------------------------------------------------------
% Layout setting
%--------------------------------------------------------------

%\usepackage{geometry}
%\geometry{
%	a4paper,
%	ignoremp,
%	bindingoffset = 1cm, 
%	textwidth     = 13.5cm,
%	textheight    = 21.5cm,
%	lmargin       = 3.5cm, % left margin
%	tmargin       = 3cm    % top margin 
%}
%--------------------------------------------------------------
%- Theorem Environments
%\theoremstyle{plain}% already by default
%---------------------------------------Begin Examples
\theoremstyle{change}
\theorembodyfont{\upshape}
\theoremsymbol{\ensuremath{\ast}}
\theoremseparator{}
\newtheorem{example}{Example}
%---------------------------------------End Examples
\newtheorem{exercise}{Exercise}
%\newtheorem{lemma}{Lemma}
\newtheorem{notation}{Notation}
\newtheorem{problem}{Problem}
\newtheorem{proposition}{Proposition}
\newtheorem{remark}{Remark}
\newtheorem{solution}{Solution}
\newtheorem{summary}{Summary}
%\newtheorem{theorem}{Theorem}

%\newtheorem{definizione}{Definizione}
%---------------------------------------Begin Definitions
\theoremstyle{plain}
\theoremsymbol{\ensuremath{\clubsuit}}
\theoremseparator{.}
\theoremprework{\bigskip\hrule}
\theorempostwork{\hrule\bigskip}
\newtheorem{definition}{Definizione}
%\newframedtheorem{definizione}{Definizione}
%\newshadedtheorem{definizione}{Definizione}
%-----------------------------------------End Definitions


\theoremstyle{marginbreak}
\theoremheaderfont{\normalfont\bfseries}\theorembodyfont{\slshape}
\theoremsymbol{\ensuremath{\diamondsuit}}
\theoremseparator{:}
\newtheorem{theorem}{Teorema} 
%--------------------------------------.For Lemmas:
\theoremstyle{changebreak}
\theoremsymbol{\ensuremath{\heartsuit}}
\theoremindent0.5cm
\theoremnumbering{greek}
\newtheorem{lemma}{Lemma} 

%--------------------------------------.For Examples:

%.For Proofs (note that theoremprework and theorempostwork are reset �
%proofs do not have lines above and below):
\theoremheaderfont{\sc}\theorembodyfont{\upshape}
\theoremstyle{marginbreak}
\theoremindent0.5cm
\theoremsymbol{\rule{1ex}{1ex}}
\newtheorem{dimostrazione}{Dimostrazione}

\newcommand{\TODO}[1]{[ {\em\marginpar{} \textcolor{blue}{TODO: #1} } ] }
\newcommand{\ERR}[1]{[ {\em\marginpar{} \textcolor{red}{ERR: #1} } ] }
\def\tm{\leavevmode\hbox{$\rm {}^{TM}$}}

\renewcommand\lstlistingname{Codice}
\renewcommand\lstlistlistingname{Codice}
\def\lstlistingautorefname{Cod.}


\renewcommand{\appendixtocname}{Appendici}
\renewcommand{\appendixpagename}{Appendici}

%\newglossaryentry{sample}{name={sample},description={a sample entry}}
%\newacronym[\glsshortpluralkey=cas,\glslongpluralkey=contrived acronyms]{aca}{aca}{a contrived acronym}

%--------------------------------------------------------------
%--------------------------------------------------------------
\makeatletter
\renewcommand{\@pnumwidth}{3em}
%\renewcommand{\linewidth}{3em}
\renewcommand{\@dotsep}{4.5}
\def\l@chapter{\@dottedtocline{0}{2em}{1em}}
\def\l@section{\@dottedtocline{1}{4em}{1.8em}}
\def\l@subsection{\@dottedtocline{2}{6em}{2.4em}}
\def\l@subsubsection{\@dottedtocline{3}{8em}{3.6em}}
\makeatother
%%% Aesthetic spacing redefines that look nicer to me than the defaults.
%
%
\renewcommand{\cftpartpresnum}{PARTE  }
\renewcommand{\cftpartfont}{\hfill\Large\bfseries\hfill}

\makeindex
\onehalfspacing
\begin{document}
\frenchspacing
\raggedbottom
\pagenumbering{roman}
\pagestyle{plain}
%\chapter*{@Todo List}
Instructions on how to use this list: instead of writing a TODO right in the middle of the thesis, I should write it in this list and refer to it with a label such as \verb|\label{TODO:keyword}|
where keyword has to be unique.
\begin{enumerate}
	\item rivedere parte thread
	\item Eliminate repetitions.
	\item Uniformare immagini: dimensioni, didascalia, posizione testo, colore, tipo di disegno (schizzo, preciso ...).
	\item Uniformare tabelle: dimensioni, didascalia, posizione testo, colore, tipo di disegno (schizzo, preciso ...).
	\item Inserire elenco di figure e tabelle 
	\item Curare sezione \ref{sensori_inerziali} con articolo Mannini-Sabatini: HealthCare and Accelerometry...
	\item Curare sezione \ref{sec:creazione_addestramento_modello} con migliore linguaggio e citazioni
	\item spostare tutte le definizioni, spiegazioni, immagini ecc sui materiali in Stato dell'Arte > Materiali, poi da Lavoro svolto > Materiali e metodi scegliere dire quali sono e fare riferimento a Stato dell'Arte > Materiali
	\item Mettere spiegazione di un acronimo la prima volta che viene usato
	\item migliorare sezione sensori da tesi paolo, articolo sabatini mannini
	\item Rispondere alla domanda, cosa deve essere forte?
	\item rivedere parte meccanica classica
	\item ogni figura e tabella e codice deve essere citato nel testo
	\item mettere la parte di segmentazione offline e online
	\item \sout{Stop using the $eng\_to\_ita$ file. Define a word the first time and use it with an acronym if possible.}
	\item \sout{Tableofcontents: make parts centered, chapters bold.}
	\item \sout{Write all non Italian words in italic (Ctrl+k) which is ironc!}
	\item \sout{Collect all TODOs. }
	\item \sout{Insert relevant words in TeXnicCenter's dictionary so that I can notice any typing mistakes.}
	\item \sout{mettere 1 spazio prima e dopo  }\verb|\cite|
	\item \sout{finire capitolo risultati e conclusioni}
	\item \sout{inserire dati calcolo cadenza}
	\item \sout{inserire correzioni Dada}
	\item \sout{inserire correzioni Charlie}
	\item \sout{scrivere sull'unit� di controllo}
\end{enumerate}

\section*{Indicazioni Unifi, a cura di M.C.Verri}

\begin{description}
	

\item[Errori comuni]
\begin{enumerate}
	\item spazi e punteggiatura: i simboli di punteggiatura devono essere attaccati alla parola precedente e separati dalla successiva da uno ed un solo spazio;
\item l'apostrofo non vuole spazi n� prima n� dopo;
\item la parentesi aperta vuole lo spazio prima ma non dopo, mentre la parentesi chiusa, vuole lo spazio dopo ma non prima: come (in questo) esempio;
\item mai mettere doppi o tripli spazi tra parole e frasi;
usare gli stili speciali in modo consistente:
\item il corsivo � riservato ai nomi scientifici di specie animali o vegetali e alle parole in lingue estere; 
\item il MAIUSCOLO � riservato a sigle e acronimi brevi; 
\item il grassetto � riservato a titoli di capitoli e sottocapitoli/paragrafi; 
\item la sottolineatura � meglio non usarla;
\item gli accenti in italiano si scrivono quasi tutti gravi (�, �, �, �, �); fanno eccezione le parole perch�, affinch�, poich�, n� ... n�;
\item un po' � il troncamento di un poco e si scrive con l'apostrofo non con l'accento;
\item non usare verbi coniugati alla prima persona singolare o plurale (meglio infinito o impersonale o passivo);
\item andare a capo solo quando cambia il soggetto o l'argomento;
\end{enumerate}
\item[Evitare]
\begin{enumerate}
	\item verbi coniugati con tempi al futuro;
\item periodi troppo lunghi (massimo sei righe tra un punto e l'altro);
\item termini stranieri (es.: bypassare, stress, ...);
\item abbreviazioni (es.: no);
\item ripetizioni;
\item i superlativi di aggettivi gi� di per s� 'assoluti' (es.: elevatissima, rapidissimo,  notevolissime, ...);
\item personalizzazioni e i soggettivismi (es.: per me, per noi, personalmente, � giusto dire, ovviamente, � chiaro che, ...).
\end{enumerate}
\item[Attenzione a]
\begin{enumerate}
\item concordanza tra soggetto e verbo;
\item consecutio temporum;
\item uso dei sinonimi;
\item particelle "che" a catena;
\item mantenere le elencazioni consistenti con la premessa;
\item "grosso" al posto di "grande";
\item riportare per esteso il significato di una sigla la prima volta che la si usa.
\end{enumerate}
\end{description}

%
%\section*{Guida alla stesura di una relazione scientifica}
%Piccola guida alla stesura di una relazione scientifica 
%Giovanni Righini 
%Polo Didattico e di Ricerca di Crema 
%Universit� degli Studidi Milano 
%righini@crema.unimi.it 
%Marzo 1999 
% Questa guida � dedicata a tutti gli studenti che sono in difficolt� quando devono mettere per 
%iscritto una relazione sul loro lavoro: una relazione, e ancor pi� una tesina o una tesi, non deve essere 
%una penitenza n� per chi la scrive n� per chi la legger�. Per evitare di dover correggere sempre gli stessi 
%errori, mi sono deciso a mettere a mia volta per iscritto alcune regole e consigli per la stesura e per la 
%revisione dei testi. Esistono molti pregevoli lavori in commercio, pensati per lo stesso scopo e molto
%pi� esaurienti di questa piccola guida, che ne � come un �bigino� e che non ha alcuna pretesa di essere 
%un completo manuale di stile, n� di grammatica. D�altronde, poich� chi fa fatica a scrivere solitamente 
%fa fatica anche a leggere, mi � sembrato necessario rendere disponibile una guida piccola e sintetica.
% Ho organizzato l�esposizione in tre parti: la struttura della relazione, i consigli per la 
%composizione e gli errori comuni. 1. La struttura della relazione 
% Generalmente una relazione scientifica � composta da un sommario (abstract), 
%da alcuni capitoli (chapters nei libri, sections negli altri casi), da una bibliografia ed 
%eventualmente da appendici. A loro volta i capitoli riguardano solitamente il 
%problema, il modello, la tecnica risolutiva (per esempio l�algoritmo nel caso della 
%Ricerca Operativa), gli esperimenti (se il lavoro � di natura sperimentale); sono inoltre 
%preceduti da un�introduzione e seguiti dalle conclusioni e da eventuali ringraziamenti. 
%Verso ciascuna delle parti il lettore ha delle aspettative, che devi conoscere e tenere 
%presenti mentre scrivi. 
%Il titolo 
% Il titolo � la pietra di paragone della relazione: il contenuto verr� giudicato in 
%funzione del titolo. Perci� nel titolo devi dire immediatamente al lettore che tipo di 
%lavoro stai presentando: una rassegna, l�illustrazione di un modello, la dimostrazione 
%di un nuovo teorema, una nuova dimostrazione di un teorema gi� noto, le propriet� di 
%un algoritmo, i risultati di esperimenti compiuti, la soluzione di un problema reale in 
%un�azienda, eccetera. Poich� il tuo lavoro verr� giudicato rispetto al titolo, dare al tuo 
%lavoro il titolo giusto � indispensabile, se vuoi che esso sia valutato nel modo giusto. 
% Per esempio da una relazione intitolata �Neural networks for optimization: the 
%state of the art� il lettore non si attende la descrizione di risultati innovativi, ma 
%pretende la completezza, una forma espositiva discorsiva e con poche formule, una 
%bibliografia assai nutrita e aggiornata. Al contrario da una relazione intitolata �A new 
%neural algorithm for the Job Shop Scheduling Problem� il lettore non si aspetta una 
%panoramica vasta e discorsiva, ma piuttosto una dettagliata descrizione, con tanto di 
%formalismo matematico, di un algoritmo originale, la dimostrazione delle sue 
%propriet�, il confronto con altri algoritmi su esemplari del JSSP, una bibliografia 
%specifica sul JSSP e sugli algoritmi neurali. Infine, se il titolo � �The optimization of 
%the production scheduling of floppy disks by neural networks. a case study�, il lettore 
%non cerca alcun contributo originale allo stato delle conoscenze, n� una rassegna, ma 
%sicuramente vuole trovare la descrizione dettagliata del sistema di produzione 
%studiato, la discussione sulla validit� del modello, i risultati sperimentali ottenuti su 
%esemplari reali del problema in azienda, la valutazione delle conseguenze del lavoro 
%sulla produttivit� di quell�azienda. 
% Il titolo deve essere sintetico. Evita titoli del tipo �Studio, realizzazione e 
%testing sperimentale di tecniche di Soft Computing per la valutazione dell�impatto sul 
%traffico di un sistema ottimizzato di gestione di una flotta di veicoli per applicazioni di 
%tipo Dial-a-Ride nel contesto di aree fortemente urbanizzate�. Prima di scegliere il 
%titolo chiediti quali sono le �parole chiave� (keywords) che useresti per classificare il 
%lavoro. Non sceglierne pi� di tre: una per il problema o il settore di applicazione, una 
%per il metodo adottato, una per il tipo di lavoro. Componi quindi il titolo usando solo 
%quelle parole chiave (pi� i necessari articoli e preposizioni ovviamente). Ad esempio 
%il titolo citato sopra pu� convenientemente diventare: �Studio di fattibilit� di un 
%sistema di trasporto a chiamata in ambito urbano�. Privilegia sempre l�indicazione del 
%problema rispetto a quella del metodo. Solo se il metodo � originale vale la pena di 
%scegliere una parola chiave apposita e di inserirla nel titolo;  altrimenti si pu� 
%tranquillamente omettere, come nell�ultimo esempio. 
% Un altro esempio: "Realizzazione di componenti per l'implementazione di 
%algoritmi neuro-fuzzy internamente ad un ambiente di simulazione distribuito in Java" 
%equivale ad "Algoritmi neuro-fuzzy per la simulazione distribuita".  Per facilitare la classificazione ed il reperimento del lavoro (anche automaticamente, con 
%motori di ricerca), � utile specificarne esplicitamente le parole chiave. Solitamente l�elenco delle parole 
%chiave � posto tra il sommario e l�introduzione. 
%Il sommario 
% Lo scopo del sommario � quello di consentire al lettore di farsi un�idea dei 
%contenuti del lavoro per decidere se leggerlo o no. La decisione � importante perch� 
%molto spesso implica dei costi (in denaro o in tempo): nelle basi di  dati in rete i 
%sommari sono consultabili gratuitamente, i testi no; leggere il sommario di un articolo 
%in biblioteca non costa nulla; fotocopiare l�articolo, per leggerlo in seguito, costa. 
% Il sommario dovrebbe essere un�esposizione in forma discorsiva delle parole 
%chiave scelte per formare il titolo. Perci� pu� essere strutturato seguendo la stessa 
%traccia usata per il titolo: una frase per inquadrare il problema o l�ambito applicativo 
%che ha motivato il lavoro, una frase per identificare quali tecniche, metodi o algoritmi 
%hai adottato, una frase per illustrare quali risultati hai conseguito, evidenziando i 
%contributi originali, se ce ne sono. 
% Il sommario deve quindi rispondere a tre domande: �Perch� � stato fatto 
%questo lavoro?�, �Come � stato fatto?�, �Che obiettivi ha conseguito?�, le quali a loro 
%volta sono riassumibili nell�unica domanda: �Perch� questo lavoro si intitola cos�?�. 
%L�introduzione 
% L�introduzione � diretta a quel lettore che, avendo gi� letto il sommario, ha 
%capito che il lavoro gli interessa. Scopo principale dell�introduzione � quello di 
%inquadrare il lavoro nel suo contesto. Poich� il sommario deve riassumere il contenuto 
%dell�introduzione (appunto perci� si chiama �sommario�), l�introduzione non pu� che 
%essere un�ulteriore espansione del contenuto del sommario. Ad esempio,  se nel 
%sommario hai parlato di �impiego di reti neurali per simulare il traffico su una rete di 
%trasporto�, nell�introduzione dovrai spiegare perch� la simulazione del traffico � 
%importante, come si costruiscono i modelli delle reti di trasporto, quali risultati sono 
%gi� noti sulla simulazione del traffico e quali obiettivi non sono ancora stati raggiunti. 
% E� importante che l�introduzione ponga in evidenza il rapporto che esiste tra il 
%lavoro e i lavori precedenti sullo stesso argomento. Per questo motivo l�introduzione � 
%spesso il capitolo pi� ricco di citazioni bibliografiche. Dopo aver letto l�introduzione 
%il lettore deve essersi fatto un�idea chiara del contributo originale del lavoro, della sua 
%utilit� e del suo valore. 
% Scopo dell�introduzione � anche di avvisare il lettore che per comprendere 
%quanto legger�, egli deve possedere delle conoscenze di base. Non rispiegare daccapo 
%ogni volta cos�� un algoritmo, cos�� un calcolatore parallelo, cos�� il �Branch & Cut�, 
%cos�� un algoritmo genetico� Devi richiedere al lettore certe  conoscenze, ne hai 
%diritto; hai per� il dovere di spiegargli chiaramente quali sono, e di dargli i mezzi (le 
%citazioni bibliografiche) per mettersi al passo se non lo �. 
% E� una buona abitudine illustrare brevemente in fondo all�introduzione 
%l�organizzazione dei capitoli successivi. Per� se la struttura del lavoro � molto 
%articolata in capitoli e sottocapitoli, � preferibile usare un indice. 
%Il modello 
% Le scienze matematiche, tra cui l�Informatica e la Ricerca Operativa, non 
%studiano fenomeni reali, come fanno la Fisica e la Biologia, ma  modelli che li 
%rappresentano. Perci� � importante distinguere chiaramente tra i  due livelli di 
%conoscenza, quello riferito alla realt� e quello riferito al modello. Il passaggio da un 
%fenomeno ad un suo modello (e viceversa) � estemamente critico e deve essere giustificato con cura. Tieni presente che il modello non � mai n� unico n� perfetto: � 
%sempre possibile rappresentare lo stesso sistema reale con un modello diverso da 
%quello che hai scelto tu. 
% La definizione di un modello non � mai automatica ma � fatta da scelte e le 
%scelte vanno giustificate. Devi mettere in evidenza i limiti del modello che usi, 
%spiegando che cosa  non viene rappresentato e spiegando anche perch� ci� sia 
%ugualmente accettabile. Questo � particolarmente importante nei lavori di simulazione 
%e nei lavori di natura applicativa riferiti a casi reali. Nei lavori di carattere pi� teorico, 
%che si riferiscono a modelli standard (ad esempio il modello matematico del TSP), la 
%giustificazione del modello non serve, ma devi sostituirla con gli opportuni richiami 
%bibliografici. Se il lettore non � convinto della ragionevolezza del modello, tutto il 
%resto del tuo lavoro perde completamente di significato ai suoi occhi. 
% Se la definizione del modello � opera tua, devi anche spiegare come hai 
%validato il modello stesso, cio� in che modo hai ottenuto la ragionevole certezza che il 
%modello che usi rappresenta davvero il sistema reale che stai studiando. 
%L�algoritmo 
% Se il lavoro prevede la spiegazione di un algoritmo, distingui bene l�algoritmo 
%dal programma. Un conto � descrivere un algoritmo e un altro conto � scrivere la 
%documentazione o il manuale d�uso di un programma. 
% La descrizione di un algoritmo non richiede alcun linguaggio di 
%programmazione, ma si pu� fare con uno pseudo-linguaggio (ad esempio uno pseudoPascal), che metta in evidenza i costrutti fondamentali: la sequenza, la selezione, 
%l�iterazione. Anche i diagrammi di flusso sono un mezzo illustrativo efficace. Evita 
%accuratamente di citare i nomi delle variabili o delle funzioni che fanno parte del tuo 
%programma: hanno significato solo per te. 
% Se l�algoritmo, come spesso accade, � complesso, dividi la sua descrizione in 
%parti. Ad esempio la descrizione di un algoritmo di branch-and-bound pu� essere 
%divisa in almeno quattro diverse parti: la strategia di partizione, la politica di 
%esplorazione dell�albero di ricerca, il calcolo del limite superiore e il calcolo del limite 
%inferiore. Puoi applicare la stessa idea anche in senso gerarchico, definendo 
%sottoprogrammi che possano essere descritti separatamente, proprio  come si fa nel 
%progetto del software. 
%Le proposizioni 
% Se esistono nel tuo lavoro propriet� formali dimostrate, devi porle in evidenza. 
%Non tutte le affermazioni dimostrate sono �teoremi�. Solo le pi� importanti, generali e 
%ricche di conseguenze vengono definite cos�. Il �lemma� � un�affermazione che serve 
%come passaggio intermedio per dimostrare un teorema. Il �corollario� invece � una 
%conseguenza diretta di un teorema, per lo pi� un�applicazione del teorema ad un caso 
%particolare. Un�affermazione che non richiede una particolare dimostrazione pu� 
%essere denominata �osservazione� e se viene posta in evidenza si suppone che verr� 
%usata o richiamata nel seguito. Un�affermazione di cui non si conosce la 
%dimostrazione � una �congettura�. 
% E� bene evidenziare nel testo il principio e la fine delle dimostrazioni, che 
%devono sempre seguire immediatamente l�enunciato delle proposizioni: ad esempio si 
%pu� marcare l�inizio con la parola �Dimostrazione (Proof)� e la fine con la classica 
%sigla �cvd� (=�come volevasi dimostrare�) o, alla latina,  �qed� (=�quod erat 
%demostrandum�) o ancora con un simbolo convenzionale (spesso si usa un quadratino). Questo accorgimento serve al lettore che vuole solo conoscere gli 
%enunciati per saltarne la dimostrazione. 
%I risultati sperimentali 
% La descrizione degli esperimenti riguarda almeno tre cose fondamentali: i 
%mezzi usati, i dati di ingresso e i risultati in uscita. 
% A proposito dei mezzi che hai usato, se hai fatto esperimenti nei quali sono 
%significativi i tempi di calcolo, non dimenticarti di riportare il tipo di macchina, la 
%frequenza della CPU e la quantit� di memoria di lavoro a disposizione. 
% A proposito dei dati di ingresso, ricordati che i tuoi esperimenti devono essere 
%riproducibili: un esperimento non riproducibile non � un esperimento scientifico! 
%Perci� devi dare al lettore tutte le informazioni necessarie perch� egli possa ripetere i 
%tuoi esperimenti e verificarli di persona. Indica i valori assegnati ad ogni parametro (le 
%costanti nei programmi). Descrivi come hai generato i files  di ingresso, se li hai 
%generati tu. Non basta dire �generati a caso�: bisogna specificare anche con quale 
%distribuzione di probabilit� (uniforme, normale,�) ed entro quali valori massimi e 
%minimi. Se hai usato dei benchmarks forniti da altri, specifica da chi e come fare per 
%ottenerli. 
% A proposito dei risultati in uscita, anzitutto non chiamarli �dati�. Appunto 
%perch� sono in uscita, non sono affatto �dati�. �Dato� non � sinonimo di �costante�, di 
%�numero� o di �valore�. �Dato� � tutto e solo ci� che non � stato costruito o calcolato 
%ma esiste gi�, � conosciuto in partenza, � �in ingresso� per definizione. 
%Come organizzare in modo leggibile e sintetico la presentazione dei risultati 
%dipende fortemente dal tipo di esperimento e quindi non esistono regole valide in 
%generale. Alcune indicazioni di massima per� valgono sempre: organizza i valori in 
%tabelle in modo da far risaltare i confronti significativi, ad  esempio tra tempi di 
%calcolo di algoritmi diversi o tra valori di soluzioni diverse dello stesso esemplare di 
%problema. Se nell�intestazione delle tabelle usi delle abbreviazioni o dei simboli, devi 
%spiegarli per esteso nel testo: �nella colonna z_UB sono riportati i valori dell�upper 
%bound calcolato con�; nella colonna t sono riportati i tempi di calcolo espressi in 
%secondi, escludendo il tempo di I/O��. 
% I commenti ai risultati sono utili ma non devono essere banali, n� arbitrari. Un commento � 
%banale quando non d� alcuna informazione significativa che non possa essere ricavata a prima vista 
%dalle tabelle dei valori o semplicemente dal buon senso: �All�aumentare della capacit� dei veicoli 
%diminuisce il tempo medio di servizio dei passeggeri�. Un commento � arbitrario quando usa i risultati
%sperimentali per trarre conclusioni non dimostrate. Di solito i commenti di questo tipo contengono 
%espressioni fumose e mal definite: �Con l�impiego di cinque veicoli si ottiene un livello di servizio 
%decisamente buono� (�buono� secondo quale criterio?), �Questi risultati, paragonati con quelli di Tizio, 
%sono da ritenersi soddisfacenti� (chi l�ha detto?), �I valori sono molto simili� (o sono uguali o sono
%diversi), �Scartando i valori meno significativi,�� (meno significativi in base a cosa?). 
% Perch� il lettore possa apprezzare il significato dei risultati devi fornirgli il 
%giusto termine di paragone. Ad esempio, se il lavoro propone un nuovo algoritmo di 
%approssimazione � necessario confrontare i risultati del nuovo algoritmo con quelli 
%ottenuti dal miglior algoritmo noto sugli stessi esemplari di problema. 
%L�interpretazione dei risultati � un�operazione tanto delicata quanto lo � quella 
%di definizione del modello. In effetti sono le due facce di una stessa medaglia: quanto 
%accurato � il modello, tanto significativi sono i risultati. Soprattutto se il lavoro � di 
%natura applicativa (un case study) devi mettere in stretta relazione le due parti. E� una 
%buona idea scriverle insieme, oppure riscrivere l�una tenendo presente l�altra, 
%alternatamente. Le conclusioni 
% A volte le conclusioni sono gi� contenute nei capitoli precedenti. E�  inutile 
%aggiungere frasi del tipo: �In questo lavoro � stato presentato un nuovo modello di �, 
%� stato dimostrato un teorema�,� dato che gi� il sommario e l�introduzione 
%contengono le stesse informazioni. Piuttosto il capitolo conclusivo pu� essere 
%dedicato ad una breve discussione critica del lavoro e a tratteggiare i futuri sviluppi; � 
%l�operazione simmetrica a quella fatta nell�introduzione, nella quale si discute la 
%relazione con i lavori precedenti e lo stato delle conoscenze. 
% Evita le conclusioni subdolamente autocelebrative: �Le tecniche qui esposte si 
%sono rivelate assai utili��, �Questo metodo appare molto promettente��, �Sembra 
%quindi da preferisrsi tale approccio rispetto a quello tradizionale�. Tocca al lettore 
%dare giudizi di questo tipo, non a te. 
%I ringraziamenti 
% Dopo le conclusioni e prima della bibliografia a volte viene aggiunto un breve paragrafo di 
%rigraziamenti a persone che, senza essere gli autori del lavoro, hanno contribuito alla sua realizzazione. 
%Essere semplicemente citati dall�autore � gi� di per s� una soddisfazione; non servono frasi servili o
%retoriche. 
%La bibliografia 
% Deve essere completa e aggiornata, non necessariamente lunga. Spesso viene usata dal 
%potenziale lettore insieme al sommario, per avere un�idea del lavoro prima di leggerlo. Deve quindi 
%essere compilata con molta cura. Di ogni citazione devono essere specificati l�autore (o gli autori), il 
%titolo, il libro o la rivista in cui il lavoro appare, la casa editrice e l�anno. 
% Ogni lavoro citato nella bibliografia deve essere anche richiamato nel testo, almeno una volta 
%(e viceversa, ovviamente). 
%Le appendici 
% Le appendici contengono parti del lavoro, che appesantirebbero 
%eccessivamente la lettura se poste nel testo, interrompendo il  filo logico 
%dell�esposizione per troppo tempo. Ad esempio, grandi tabelle di valori numerici (dati 
%o risultati), lunghe dimostrazioni di proposizioni, risultati collaterali del lavoro, casi 
%particolari. Se ci sono appendici, devono esistere nel testo i corrispondenti richiami: 
%��la cui dimostrazione (v. appendice A)��. 
%Le figure e le tabelle 
% Ogni figura deve essere numerata e deve avere una didascalia. Inoltre � bene 
%che ci sia un riferimento nel testo per ciascuna: �...il grafo mostrato in figura 3...�, 
%�...la soluzione (fig. 5)...�. 
% Nel caso delle tabelle si pu� scegliere tra una didascalia come per le figure o 
%un�intestazione sopra la tabella o sulla sua prima riga. Anche per le tabelle vale la 
%regola che nel testo deve comparire almeno un puntatore: �...sono riassunti nella 
%tabella 1...�, �...il confronto tra i tempi di calcolo (tab. 2-4)...�. 2. Consigli per la composizione 
%Il miglior consiglio che io ti possa dare � di dedicare un�oretta della tua vita alla lettura di 
%�The elements of style� di W.Strunk e E.B.White, ed. MacMillan (New York 1979). Io l�ho fatto e lo 
%consiglio vivamente. Un altro testo raccomandabile � il pi� corposo �La scrittura tecnico-scientifica� di 
%E.Matricciani, ed. Citt� Studi (Milano, 1994), utile soprattutto per redigere tesi di laurea. Qui di seguito 
%elenco alcuni consigli, tutti tratti da esperienze personali. 
%1) Usa il tempo presente 
% Esiste la tendenza errata ad usare fuori luogo i tempi passati (soprattutto 
%l�imperfetto) e futuri. 
%�Da tali prove � risultato che i migliori risultati si ottenevano ponendo�� 
%�Da tali prove risulta che i migliori risultati si ottengono ponendo��
%�Se l�intensit� del segnale ricevuto supera il valore di soglia, il neurone 
%recettore si attiver� a sua volta��  �Se l�intensit� del segnale ricevuto supera il 
%valore di soglia, il neurone recettore si attiva a sua volta��
%La relazione su un lavoro non � la cronistoria del lavoro (�In una prima fase 
%del nostro studio abbiamo considerato��). Al lettore non interessa sapere cosa � stato 
%fatto prima e cosa dopo. Interessano i metodi e i risultati, interessano le idee. 
%2) Riconosci ed evita le tautologie e le contraddizioni 
% Le tautologie sono periodi che dicono due volte la stessa cosa in modo 
%diverso. Le contraddizioni sono periodi che dicono una cosa e il suo contrario. 
%�Per simulare l�evoluzione di un sistema dato�  �Per simulare l�evoluzione 
%di un sistema��. E� ovvio che il sistema � �dato�, dal momento che lo vuoi simulare. 
%�mobilit� dinamica�  �mobilit��. La mobilit� non pu� essere statica, lo dice 
%la parola stessa. 
%�il primo vero e concreto modello�  �il primo modello�. L�aggettivo �vero� 
%provoca una tautologia, poich� un modello non pu� essere falso; l�aggettivo 
%�concreto� provoca una contraddizione, poich� un modello � per sua natura astratto. 
%�Ogni elemento � definito da un insieme di attibuti che formano il pattern 
%descrittivo dell�elemento�. Se l�elemento coincide col suo pattern descrittivo, la frase 
%� tautologica; se l�elemento ed il pattern sono cose diverse, � contraddittoria  �Ogni 
%elemento � descritto da un pattern di attributi�. 
%�La soluzione � quella di fornire all�utenza un servizio di trasporto pubblico, 
%alternativo all�uso del veicolo proprio, che trasporti gli utenti dalla loro abitazione�� 
%contiene pi� di una tautologia  �La soluzione consiste nel rendere disponibile un 
%servizio pubblico di trasporto a domicilio�� 
%3) Scegli i sostantivi giusti e riferisci gli aggettivi, i pronomi e i verbi al sostantivo 
%giusto 
%La �conoscenza� � certa, per via logica o sperimentale; la �stima� invece � 
%basata su ipotesi o modelli. 
%�Col modello X si pu� risalire dalla conoscenza di un limitato numero di flussi 
%alla conoscenza di tutti i flussi�  �Col modello X si pu� risalire dalla conoscenza di 
%un limitato numero di flussi alla stima di tutti i flussi�. 
%�Sulla realt� di New York sono state eseguite le seguenti simulazioni�� 
%�Sul modello di New York sono state eseguite le seguenti simulazioni�� Non si �riporta un�applicazione�: si �riportano i risultati relativi ad 
%un�applicazione�. 
%�Il neurone � stato enunciato nel 1943��  �Il neurone � stato definito nel 
%1943�� oppure �La definizione del neurone � stata enunciata nel 1943��. 
%��un numero di neuroni inferiore a� evidenzia difficolt� di apprendimento� 
% �la rete neurale evidenzia difficolt� di apprendimento se il numero di neuroni � 
%inferiore a��. Il verbo "evidenzia" non ha come soggetto "il numero". 
%��per realizzare modelli di simulazione di sistemi biologici sempre pi� 
%accurati�   �...per realizzare modelli sempre pi� accurati di sistemi  biologici�. 
%L'aggettivo "accurati" � riferito ai modelli, non ai sistemi. 
%4) Scegli le preposizioni giuste 
%�la mobilit� di una rete di trasporto�  �la mobilit� su una rete di trasporto�, 
%poich� non � la rete a muoversi. 
%��si propaga a mezzo di��  ��si propaga per mezzo di�� 
%5) Sii parco di aggettivi 
%Gli aggettivi servono ad illustrare meglio, non a eccitare il lettore (pi� spesso 
%lo innervosiscono). 
%�I rapidi progressi della ricerca informatica rendono disponibili nuovi e potenti 
%strumenti, che consentono��   �I progressi della ricerca informatica rendono 
%disponibili strumenti la cui potenza consente�� 
%�insostituibile�   �utile�, �particolarmente indicato� (spiegando anche 
%perch�). 
%�innumerevoli�  �alcuni�, �molti� (quanti esattamente?). 
%6) Non usare i simboli matematici come nomi e non descrivere  le formule a 
%parole. 
%�Il delta di un nodo�� non � corretto. Esiste sempre un modo di sostituire il 
%nome del simbolo col suo significato: �l�etichetta di��, �il valore di soglia di��, �la 
%variazione di��, �il valore associato a��. Meglio ancora sostituire la frase con una 
%formula. 
% Frasi come �la somma dei primi N elementi � sempre limitata da K, per valori 
%di N inferiori alla cardinalit� dell�insieme dei nodi selezionati all�iterazione t.� devono 
%essere sostituite da formule. 
% E� buona abitudine numerare le formule per poterle facilmente richiamare nel 
%testo: �Sostituendo il vincolo (3) con il vincolo (8) si ottiene un rilassamento��. 
%7) Usa termini uguali per cose uguali e termini diversi per cose diverse. 
% Se un �oggetto� poche righe dopo diventa un �elemento�, se un �piano� 
%diventa uno �strato�, se un �nodo� diventa un �vertice�, il lettore � portato a 
%confondersi. La stessa cosa deve ricevere lo stesso nome. 
% Viceversa non usare lo stesso termine per indicare cose diverse: �sistema�, 
%�iterazione�, �applicazione� e altri ancora sono termini che purtroppo si prestano ad 
%essere usati pi� volte con significati diversi nello stesso contesto. In questi casi devi 
%evitare le ambiguit�, usando gli opportuni aggettivi. 
%8) Risparmia su virgolette, corsivi, grassetti, sottolineature. 
% Quando non sai come esprimere un�idea non pensare di cavartela con la 
%scappatoia della frase tra virgolette o del corsivo. Al momento giusto, accanto ad una descrizione pi� formale e precisa, pu� star bene un�espressione virgolettata. Ma non 
%deve diventare un�abitudine, che porta all�abuso. Le virgolette possono essere usate 
%solo da chi sa farne a meno. Tu dimostra innanzitutto di saperne fare a meno. 
% Se nel testo compaiono termini stranieri, soprattutto se non molto comuni, � 
%bene evidenziarli in corsivo. 
% Se � proprio necessario evidenziare alcuni termini nel testo, scegli una sola 
%modalit� per farlo: o il grassetto, o la sottolineatura, o il corsivo, ma non un miscuglio 
%di queste. Se appena � possibile per� � meglio evitare di evidenziare con questi artifici 
%grafici: la costruzione stessa della frase e del testo dovrebbe porre in evidenza ci� che 
%lo esige o lo merita. 
%9) Elimina i �tali�, �il fatto che�, i �comunque� e gli �eccetera�.
%L�aggettivo �tale� spesso fa da ponte tra punti diversi della frase, che 
%purtroppo sono lontani. Basta allora riscrivere la frase ponendoli vicini e risparmiando 
%il �tale�, facendo eventualmente ricorso ad un pronome. 
%�Lo scanner esamina l�immagine analogica e produce una rappresentazione 
%digitale di tale immagine�   �Lo  scanner esamina l�immagine analogica e ne 
%produce una rappresentazione digitale�. 
%La stessa operazione consente di unire due frasi consecutive, collegate da un 
%�tale�. 
%�L�impiego di una struttura-dati ad albero permette di aumentare l�efficienza 
%dell�algoritmo. Tale modifica influenza fortemente i tempi di calcolo�  �L�impiego 
%di una struttura-dati ad albero permette di aumentare l�efficienza dell�algoritmo, 
%riducendone fortemente i tempi di calcolo�. 
% Anche il fastidioso �fatto che� si pu� eliminare con poco sforzo cambiando il 
%soggetto della frase. 
% �E� evidente il fatto che non esiste differenza in prestazioni��  �Non si 
%notano differenze in prestazioni�� (o ancor meglio �Le prestazioni� risultano 
%identiche�). 
% �Il fatto che l�algoritmo A sia superiore all�algoritmo B, non significa che�� 
% �La superiorit� dell�algoritmo A rispetto all�algoritmo B, non significa che�� 
% Se spesso chi scrive �tale� non ha voglia di spiegare quale, chi scrive 
%�comunque� non ha voglia di spiegare perch�. Prova a togliere i �comunque� e vedrai 
%che novantanove volte su cento la frase sta in piedi lo stesso. 
%Invece chi scrive �eccetera� di solito non saprebbe rispondere alla domanda 
%del lettore �e cos�altro?�. Chieditelo anche tu e togli gli �eccetera�. 
%10) Evita le espressioni indefinite e incerte: fai affermazioni chiare. 
% I termini tecnici e in particolar modo le sigle e le abbreviazioni, vanno sempre 
%definiti per esteso la prima volta che vengono usati. �Il Vehicle Routing Problem with 
%Time Windows (VRPTW) � un problema...�. 
% Espressioni  fuzzy del tipo �un po��, �piuttosto�, �discreta�, �circa� vanno 
%sostituite con termini pi� precisi. 
%��� di discrete dimensioni�. Quanto � grande? 
%��penalizza un po� le prestazioni�. Di quanto? 
%��� quindi possibile con questo metodo risalire in tempi rapidi a��. Quanto 
%rapidi? �la capacit� di generalizzazione�� davvero molto elevata�. Quanto elevata? 
%�Un�interessante osservazione che si pu� fare osservando il grafico � che 
%sembrerebbe circa equivalente l�utilizzo di 5 veicoli con capienza 8 e l�utilizzo di 9 
%veicoli con capienza 4�  �Cinque veicoli con capienza 8 forniscono prestazioni 
%simili a quelle di nove veicoli con capienza 4 (v. grafico)�. 
%11) Evita affermazioni opinabili, non dimostrate o non dimostrabili. 
%�In questo caso un algoritmo di ricerca locale otterr� probabilmente dei 
%risultati migliori�. E� un�opinione. Meglio fare esperimenti, o citare esperimenti altrui, 
%e scrivere �In questo caso l�algoritmo di ricerca locale X ha ottenuto risultati 
%migliori�, eliminando il �probabilmente� (e il verbo al futuro). 
%�Per risolvere problemi non linearmente separabili, � necessario che la 
%funzione di attivazione sia non-lineare�. E� un�affermazione vera ma non dimostrata. 
%Bisogna citare il riferimento bibliografico dove � reperibile la dimostrazione. 
%12) Evita perifrasi ridondanti e buffe circonlocuzioni. 
% Il giudizio del lettore sul tuo lavoro non � direttamente proporzionale alla tua 
%prolissit�, n� tantomeno alla tua fantasia nel coniare espressioni originali. Il lettore 
%gradisce invece la sintesi e la chiarezza. Tieni conto che nell�accostarsi a relazioni 
%scientifiche chi legge deve gi� compiere uno sforzo intellettuale, per l�argomento 
%stesso. Non gravarlo anche di uno sforzo di interpretazione del linguaggio. Il 
%linguaggio � solo un umile strumento, non deve diventare il protagonista della tua 
%relazione. I protagonisti sono le idee. 
% Alcuni esempi: 
%�strumenti computazionali�  si chiamano �calcolatori� o �elaboratori� 
%�applicativo software�  �programma� 
%�crea non pochi problemi�  �crea molti problemi� 
%�non � affatto attrezzato�  �non � attrezzato� 
%�Come ovvia conseguenza di questa situazione�  �Quindi� 
%�supposta vera�  �stimata� 
%�che � possibile adoperare per realizzare�  �con cui si pu� realizzare� 
%�� espresso matematicamente dalla formula�  �� espresso dalla formula� 
%�operare in modo pratico rilevando i flussi�  �rilevare i flussi� 
%�Un�altra eventualit� � che il sistema possa oscillare�   �Il sistema pu� anche 
%oscillare� 
%�per esempi del tutto nuovi�  �per esempi nuovi� 
%�per ottenere l�eventuale convergenza�  �per ottenere la convergenza� 
%�non sempre sono praticamente applicabili�  �sono raramente applicabili� 
%�laddove�  �quando�, �se� 
%�commercialmente esistenti�  �in commercio� 
%�temporalmente efficiente�  �veloce� 
%�in funzione di quantit� che sono note�  �in funzione di quantit� note�. 
%�nell�ottica della minimizzazione dell�errore�  �per minimizzare l�errore�. 
%�Differentemente�  �Al contrario� 
%�pu� essere anche vista come�  �equivale a� 
%�un singolo insieme�  �un insieme� 
%�l�insieme di apprendimento, altrimenti detto  training set�   �l�insieme di 
%apprendimento (training set)� 
%�estrapolando la generalizzazione�  �generalizzando� 
%�senza che ci� provochi alcuna perdit� di generalit��  �senza perdita di generalit�� �� capace di operare appropriatamente distinguendo�  �distingue correttamente� 
% Oltre ad espressioni che si possono (e si devono) sintetizzare, esistono anche 
%espressioni che si possono (e si devono) omettere del tutto: �In tal senso�, �come si � 
%detto�, �E� da rimarcare che�, �Si vuole sottolineare che�, �in sostanza�. 
%Una delle cose pi� detestabili � l�uso ormai dilagante della congiunzionedisgiunzione �e/o�, che significa �o l�uno, o l�altro o entrambi�. In italiano esiste la 
%disgiunzione �o�, che ha esattamente questo significato. La paura che il lettore possa 
%erroneamente pensare ad un �or esclusivo� non � giustificata. Se il lettore conosce 
%l�italiano, sa benissimo che �A o B� non indica mutua esclusione tra A e B. 
%Infine non inventare di sana pianta espressioni il cui significato � noto solo a 
%te: �errore d�uso�, �settorializzazione�, �macro-passi�, �dimensioni di scelta� e via 
%dicendo. 
%13) Non ripetere la stessa idea in punti diversi del testo.
%Ogni cosa ha un suo posto giusto in cui deve essere scritta e basta scriverla l�, 
%una volta sola. Le uniche ripetizioni ammesse sono nel sommario, che dovendo 
%sintetizzare il lavoro deve ripetere ci� che � scritto da qualche parte nel testo. Ma tra 
%diverse parti del testo non devono esistere sovrapposizioni. 
%14) Spezza le frasi lunghe o accorciale cambiando la costruzione. 
%Un paio di esempi dovrebbero bastare: 
%�La fase di addestramento ha richiesto la generazione delle coppie inputoutput ottenute con una procedura di tipo pseudo-casuale, a partire da una matrice 
%O/D rappresentativa degli spostamenti nell�area, che costituiscono gli esempi di 
%addestramento per la rete neurale in numero pari a 350�  �Per l�addestramento della 
%rete neurale ho usato 350 esempi, costituiti da coppie input-output, ricavati con una 
%procedura pseudo-casuale dalla matrice O/D.� 
%�Il mercato si tiene in una zona molto centrale della citt� (centro storico): una 
%zona molto ricca di negozi, che contribuiscono a catalizzare la presenza notevole di 
%persone in questa particolare zona della citt��   �Il mercato si tiene nel centro 
%storico, una zona ricca di negozi e quindi molto frequentata�. 
%15) Usa i paragrafi per tradurre le idee in parole. 
% La difficolt� di tradurre i pensieri in un testo � dovuta in gran parte al fatto che 
%i pensieri non sono sequenziali, il testo s�. Perci�, per inciso, � pi� facile scrivere un 
%buon ipertesto che un buon testo. 
% Segui la regola �un�idea = un paragrafo�. Ti sar� pi� facile  organizzare 
%sequenzialmente il pensiero, seguirne lo sviluppo logico e riprodurlo nel testo. Ogni 
%paragrafo - la parte di testo compresa tra un ritorno a capo e l�altro - dovrebbe avere 
%senso compiuto anche se staccata dal resto del testo, dovrebbe essere cio� necessaria e 
%sufficiente ad esprimere un�idea, un concetto. Un paragrafo pu� anche consistere in 
%pi� periodi. 
% Osservando la lunghezza dei paragrafi ti accorgerai anche di quali idee hai 
%esposto �troppo�, scrivendo inutili dettagli (paragrafo molto pi� lungo degli altri), e 
%quali invece richiedono di essere approfondite meglio (paragrafo molto pi� corto degli 
%altri). Bilanciare i paragrafi corrisponde a mantenere costante il livello di dettaglio 
%nella presentazione. Infine, riporto un paio di esempi di riscrittura di intere parti di testo, seguendo 
%le regole esposte fin qui. 
%Esempio 1: �Queste considerazioni inducono a ritenere che, nonostante la rigorosit�
%delle formulazioni matematiche proposte dai vari autori, esistono comunque delle 
%quantit� (matrice di dispersione dei flussi rilevati e della matrice iniziale O/D) la cui 
%valutazione, spesso approssimata, pu� comunque inficiare la bont� dei risultati finali 
%in termini di stima corretta della matrice O/D.� 
%La principale, �Queste considerazioni inducono a ritenere che�, � priva di 
%contenuto e si pu� omettere; �la rigorosit�� � implicita nel fatto che le formulazioni 
%siano �matematiche�; i due �comunque� sono superflui; i �risultati�  sono sempre 
%�finali� per definizione; �la bont� dei risultati�in termini di stima corretta�� � 
%semplicemente �la bont� della stima��. L'intera frase si pu� riscrivere cos�: 
%�Le approssimazioni introdotte nella stima della matrice di dispersione dei 
%flussi rilevati e nella stima della matrice O/D iniziale possono inficiare la qualit� della 
%stima della matrice O/D.� 
%La riscrittura non comporta solo un risparmio di parole del 50% circa, a tutto 
%vantaggio della chiarezza e della facilit� di lettura, ma anche alcuni effetti paradossali: 
%ci� che prima era scritto tra parentesi � diventato un complemento del soggetto nella 
%proposizione principale. 
%Esempio 2: �I risultati sono confrontati con quelli forniti da uno degli algoritmi
%tradizionali precedentemente richiamati, ed in particolare si � fatto riferimento al 
%metodo di massimizzazione dell�entropia di Willumsen (1978) la cui formulazione 
%viene di seguito esplicitata.� 
%�Confrontiamo i risultati con quelli forniti dal seguente metodo di massimizzazione 
%dell'entropia (Willumsen, 1978).� 
%Quindici parole anzich� trentasette, identico contenuto. 3. Errori comuni 
% Riporto nel seguito espressioni italiane comunemente usate male, espressioni inglesi usate in 
%sostituzione di quelle italiane, espressioni inglesi italianizzate. Alcune sono diffuse, per adesso, solo nel 
%linguaggio parlato ma a scopo preventivo le cito ugualmente. 
%3.1 Espressioni italiane usate male 
%Consistere di: anche se in inglese �to consist� regge sempre la preposizione �of�, in italiano il verbo 
%�consistere� regge la preposizione �in�, non �di�.  Solo quando il significato � �essere composto da 
%parti� si pu� usare �di�: �la novit� consiste in quest�idea�, �il libro consiste in (o di) trenta capitoli�. 
%Comportare si riferisce ad una conseguenza. Non � sinonimo di �richiedere� che invece si riferisce ad 
%una pre-condizione, ad una causa. Ad esempio �L�uso di questo programma, comporta un grande 
%risparmio di tempo, ma richiede [non �comporta�] un calcolatore potente�. 
%Dimostrare significa rendere evidente la verit� di un�affermazione attraverso passaggi logici. Non � 
%sinonimo di �mostrare�, che significa rendere evidente una realt�, un dato di fatto senza alcun bisogno 
%di inferenze logiche. �Il teorema si dimostra tramite il lemma 2.1�; �gli esperimenti mostrano che 
%l�algoritmo A � pi� robusto dell�algoritmo B�. 
%Risultare � intransitivo. ��tale da risultare un�elaborazione distribuita�   ��tale da produrre 
%un�elaborazione distribuita�. 
%Significativo non � sinonimo di �rappresentativo�. Significativo � ci� che porta informazione; 
%rappresentativo � ci� che riassume, che d� un�idea di qualcosa. �Questo risultato � significativo ed �
%rappresentativo dell�intero esperimento�. 
%3.2 Espressioni inglesi in sostituzione di espressioni italiane 
% Molte parole inglesi sono entrate nell�uso comune, soprattutto nell�ambiente informatico e, 
%pi� in generale, tecnologico. E� ridicolo sforzarsi di tradurre per forza tutto in italiano: nessuno 
%vorrebbe avere un �topo� sulla scrivania, n� �ferraglia� e �morbidume� possono essere valide 
%traduzioni di  hardware e  software� D�altra parte in molti casi esistono i corrispondenti termini 
%italiani, il cui uso non ha nulla di ridicolo ed � quindi raccomandabile. 
% Esempi di parole inglesi non (o difficilmente) sostituibili e il cui uso � quindi sempre 
%giustificato:  back-up, benchmark, browser, directory, floppy disk, mouse, overflow, password, 
%scheduling, pattern, web� 
% Esempi di parole inglesi quasi sempre ingiustificate: bottleneck  (collo di bottiglia), bound
%(limite), design (progetto, progettazione, ma non �disegno�), feasible (ammissibile), feature
%(caratteristica), keyword (parola chiave), network  (rete), overhead  (sovraccarico), paper  (articolo), 
%planning (pianificazione), search  (ricerca), set  (insieme), switch  (scambio o interruttore, secondo il 
%contesto), task (compito o processo), threshold (soglia, valore limite), tool (strumento)�
%3.3 Espressioni inglesi italianizzate 
% Ancor peggio che usare male espressioni italiane o sostituirle con espressioni inglesi 
%equivalenti � senz�altro il tentativo di creare parole italiane da parole inglesi. L�esito � spesso ridicolo. 
%Ecco un breve elenco. 
%Backuppare: ammesso che �fare una copia di riserva� sia un�espressione troppo lunga, si pu� ripiegare 
%su �fare una copia di back-up� o �fare un back-up�. 
%Branchare: �to branch� si traduce con �dividere�, �scomporre�, �partizionare� o, nel peggiore dei casi, 
%�eseguire il branching�. 
%Crashare: �cadere�, �interrompersi�, �fallire�, �rompersi�, �crollare��
%Collectare: �to collect� si traduce con �raccogliere�, �collezionare�. 
%Debuggare: �correggere� un programma. 
%Deletare: �cancellare�. 
%Fillare: i moduli non si �fillano�, si riempiono. 
%Fittare: �adattarsi�, �combaciare�, �adeguarsi�. 
%Fliccherare: il �flickering� si traduce con �tremolio�; quindi lo schermo non �flicchera�, tremola. 
%Forwardare: �rinviare� un messaggio. 
%Killare: �interrompere�, �cancellare� un processo attivo su un calcolatore. 
%Input, output: �ingresso� e �uscita�. 
%Inputare: a parte la grafia scorretta (�np� non esiste nell�ortografia italiana), un dato o un file si �d� in 
%ingresso� ad un programma. 
%Lockare: �bloccare�, �chiudere�. (S-)Loggarsi: �(s-)connettersi�, �(s-)collegarsi�, �iniziare (terminare) una sessione di lavoro�. 
%Linkare: �collegare�, �collegare con un puntatore� o �puntare� nel caso di ipertesti. 
%Mailare: �inviare un messaggio di e-mail� (o di posta elettronica) o anche �mandare un�e-mail�, dato 
%che il termine e-mail, al pari di fax, fa ormai parte della lingua corrrente. 
%Postare: �spedire�, �inviare�. 
%Randomizzato: �casualizzato�. 
%Referare: �revisionare� un testo per la pubblicazione. 
%Resettare: �azzerare� un contatore, �riavviare� una macchina. 
%Restorare: �ripristinare�, �recuperare� un file cancellato. 
%Reversare: eseguire l�iperazione di �reverse�, cio� �invertire�. 
%Runnare: �eseguire� un programma. 
%Scannerare, scannerizzare:  to scan si traduce con �esaminare�, �analizzare�, �esplorare�, secondo i 
%casi. Riferito all�azione dello scanner si traduce con �scandire�. Lo scanning � la �scansione�. 
%Settare: �tarare�, �fissare�, �impostare�. 
%Sharare: �condividere�. 
%Shiftare: �spostare�, �traslare�. 
%Skippare: �saltare�, �tralasciare�. 
%Splittare: �dividere�, �sdoppiare�, �duplicare�, �distribuire�. 
%Startare: �avviare�. 
%Stoppare: a meno che non si tratti di gergo calcistico (nel calcio esiste lo stopper), to stop si traduce 
%con �fermare�, �arrestare�. 
%Switchare: �(s)cambiare�, �passare da� a��. 
%Taggare: �etichettare�, �classificare�. 
%Testare: �provare�, �collaudare�. 
%Up(Down-)gradare: �aggiornare�, �sostituire con la nuova (vecchia) versione�. 
%L�esito � ancora pi� ridicolo nei casi in cui esiste gi� un identico termine italiano con tutt�altro 
%significato. Ecco alcuni esempi. 
%Applicare: in inglese to apply significa tra l�altro �fare domanda per partecipare�, in italiano no. Non si 
%applica ad un congresso, ci si iscrive. 
%Attitudine: in italiano il termine �attitudine� ha due significati: disposizione, inclinazione naturale per 
%un�attivit� (�Non ho attitudine per il disegno�), oppure postura della persona, atteggiamento (�Si mise 
%in attitudine di preghiera�). Nel primo caso la derivazione � dal latino �aptus�=�adatto�, nel secondo 
%dal latino �actus�=�gesto�. In inglese si traducono rispettivamente in aptitude e in attitude. Ma attitude
%in inglese significa atteggiamento in senso lato, non necessariamente riferito ad un gesto o ad una 
%postura del corpo. Non � quindi corretto tradurre  attitude in �attitudine� quando significa 
%�atteggiamento�. Ad esempio non si pu� avere �un�attitudine scettica� verso qualcosa o qualcuno, ma 
%piuttosto �un atteggiamento scettico�. 
%Confidente: non � colui che ha fiducia (in inglese  confidence), ma colui che confida, cio� trasmette 
%un�informazione segreta o riservata, e quindi colui in cui si ha fiducia. Chi confida qualcosa � un 
%�confidente� (sostantivo); chi confida in qualcosa � �fiducioso� (aggettivo). 
%Disegnare (Disegno): l'inglese  to design si traduce con �progettare�, non �disegnare�, perch� si 
%riferisce ad un'attivit� pi� astratta, non necessariamente grafica e anche quando si riferisce all'attivit� 
%grafica si tratta di �disegno tecnico�, non di disegno in generale. 
%Fissare: in inglese to fix significa, tra l�altro, correggere un errore in un programma. In italiano no. Non 
%si �fissano i bachi�, si correggono gli errori. 
%Introdurre(-rsi): in inglese  to introduce someone/oneself significa presentare qualcuno/presentarsi. 
%L�italianizzazione rende possibili frasi straordinariamente equivoche: �Ho incontrato una ragazza 
%simpatica e, per fare amicizia, mi sono subito introdotto.� 
%Istanza: l�istanza in italiano non ha niente a che vedere con la instance in inglese. In inglese instance
%significa �esempio�, �esemplare�, �caso� (nel senso di �caso particolare�, non di �fatalit��). In italiano 
%�istanza� siginifica �richiesta�, in particolare richiesta formale o pressante. A problem instance � �Un 
%esemplare di problema�. 
%Processare: oltre a non essere �inputati�, i dati non si processano, si elaborano. 
%Realizzare: il verbo to realize si traduce con �rendersi conto di�, �accorgersi di�, non con �realizzare�, 
%che significa tutt�altro. 
%Ricoverare: eseguire la  recovery di un processo sospeso non significa �ricoverare�  il processo. 
%Piuttosto, �riattivarlo�, �ricuperarlo�, �ripristinarlo�. Riferito a persone l�equivoco � ancora pi� marcato perch� in inglese to recover significa �rimettersi in salute�, cio� esattamente il contrario di ci� 
%che significa �ricoverare� in italiano. 
%Ristorare: va bene �recuperare� un file cancellato e �ripristinarlo�, ma dargli anche da mangiare�
%Ritornare: in italiano � intransitivo. Non si ritorna un messaggio, lo si rimanda o lo si rinvia o lo si 
%rispedisce; non si ritorna una risposta, la si d�. 
%Scannare: lo scanner non gronda di sangue. Si �scandisce� un�immagine, se ne fa una �scansione�. 
%Sottomettere: in inglese  to submit significa �sottoporre� nel senso di �presentare�,  �sottoporre a 
%giudizio altrui�. L�italiano �sottomettere� significa invece �soggiogare�. 
%Sottoscrivere: spesso usato come erronea traduzione dell�inglese to subscribe nel senso di �abbonarsi�, 
%�iscriversi�. In italiano �sottoscrivere� significa invece �firmare per adesione, per conferma, per 
%accettazione�. 
%Sperimentare:  to experience significa �fare esperienza di�, �provare�, nel senso di provare piacere, 
%provare dolore, provare difficolt�. In italiano non si �sperimentano� queste cose, perch� �sperimentare� 
%indica un�azione volontaria.  I experienced some problems with this program si traduce con �Ho 
%incontrato alcune difficolt� con questo programma�, non con �Ho sperimentato alcune difficolt���. 
%Supporto: in italiano il verbo to support si traduce sia in �supportare� che in �sostenere�. Ma i nomi 
%�sostegno� e �supporto� sono intercambiabili solo se si riferiscono ad un oggetto concreto. In senso 
%figurato, morale, psicologico, intellettuale, esiste solo il �sostegno�, non il �supporto�. Si pu� 
%�supportare una tesi�, ma si adducono argomentazioni �a sostegno�, non �a supporto�, di una tesi. 
%--------------------------------------------------------------
% Frontmatter
%--------------------------------------------------------------
\frontmatter
%--------------------------------------------------------------
% titlepage.tex (use thesis.tex as main file)
%--------------------------------------------------------------
\begin{titlepage}
	\begin{center}
   	\large
      \hfill
      \vfill
      \begingroup
			\spacedallcaps{\myUni} \\ 
			\myFaculty \\
			\myDegree \\ 
			\vspace{0.5cm}
         \includegraphics[scale=.065]{logo/unifi}\\
         \vspace{0.5cm}    
         Tesi di Laurea    
      \endgroup 
      \vfill 
      \begingroup
      	\color{RoyalBlue}\spacedallcaps{\myTitle} \\ \bigskip
      \endgroup
      \spacedlowsmallcaps{\myName}
      \vfill        
			\begin{flushleft}
			\hspace{2.5cm} Relatore \hspace{.6cm}\textit{\myProf}\\ 
			      \vspace{0.3cm}
			\hspace{2.6cm}Co-Relatore\hspace{.3cm}\textit{\myOtherProf}\\
			\end{flushleft}
      \vfill
      \myTime\\
      \hspace{.5cm}
      \myVersion
      \vfill                      
	\end{center}        
\end{titlepage}   
%--------------------------------------------------------------
% back titlepage
%--------------------------------------------------------------
   \newpage
	\thispagestyle{empty}
	\hfill
	\vfill
	\noindent\myName: 
	\textit{\myEngTitle}, 
	\myDegree \textcopyright\ \myTime-\myShortVersion

%--------------------------------------------------------------
% back titlepage end
%--------------------------------------------------------------
\renewcommand\contentsname{Indice}
\addtocontents{toc}{~\hfill\textbf{Pagina}\par}
\tableofcontents
\pagestyle{scrheadings}
%--------------------------------------------------------------
% Mainmatter
%--------------------------------------------------------------
\mainmatter
\cleardoublepage 
\pagenumbering{arabic}

\chapter{Sommario}
%\myChapter{Sommario}

In questo lavoro si affronta il problema dell'analisi in linea della deambulazione umana mediante metodi di apprendimento automatico e lo sviluppo di un'applicazione per \textit{smartphone Android}.\\
Una HMM (\textit{Hidden Markov Model}) a quattro stati addestrata in differita su segnali giroscopici provenienti da sessioni di cammino e corsa su tapis-roulant a diverse velocit� ed inclinazioni, viene usata per segmentare in linea le fasi del cammino grazie ad una versione modificata dell'algoritmo di decodifica di Viterbi.\\
Il giroscopio usato � contenuto in una IMU (\textit{Inertial Measurement Unit}) collocato sul collo del piede ed orientato con l'asse sensibile sul piano mediale laterale.\\
L'applicazione per \textit{smartphone} permette di controllare la IMU via \textit{Bluetooth}, nonch� di segmentare e visualizzare in linea il segnale giroscopico relativo a deambulazione. \\
La validazione del sistema viene fatta stimando la distanza percorsa calcolata da dati (cadenza e velocit�) ottenuti in sessioni di cammino su tapis-roulant in laboratorio, e confrontandola con la distanza misurata mediante GPS in sessioni di cammino all'aperto, in condizioni non controllate.\\
%I risultati sono ... 

\addtocontents{toc}{\protect\mbox{}\protect\hrulefill\par}
\addtocontents{toc}{\protect\mbox{}\protect\hrulefill\par}
\addtocontents{toc}{\cftpagenumbersoff{part}}
\part{Stato dell'arte}

\chapter{Introduzione}
%\myChapter{Introduzione}

%-------------------	Descrizione del problema -------------------------------
Ipotesi: \textit{� possibile costruire un sistema intelligente e portatile in grado di riconoscere (classificare) e analizzare i movimenti di un individuo e di fornirne in linea (vedi appendice \ref{sec:real_time_sys}) informazioni a riguardo}.\\

Il problema del riconoscimento (classificazione) di un movimento umano generico � un problema irrisolto ed estremamente complesso visto il numero di esorbitante di parametri coinvolti. 
In questo lavoro viene affronta l'analisi di una singola attivit� nota: la deambulazione entro un intervallo di velocit� e pendenza del terreno.

Nello specifico il problema � quello dell'individuazione in linea delle tempistiche di eventi che costituiscono una deambulazione normale come studiato dalla Chinesiologia (vedi capitolo \ref{cap:chinesiologia}). Tale problema � noto come \emph{problema della segmentazione automatica ed in linea della deambulazione umana}.\\

%------------------  Motivazioni del problema Perch� si pone?--------------------------------
La soluzione del problema della segmentazione avrebbe risvolti immediati nella Medicina riabilitativa per la diagnosi e/o assistenza a persone con problemi di deambulazione, nella Robotica e Computer Grafica per la emulazione/simulazione della deambulazione umana, nel mondo dello sport agonistico per l'apprendimento di specifiche tecniche motorie.\\

%---------------------� gia stato affrontato? 
Il problema della segmentazione � stato ampiamente affrontato nella letteratura scientifica (letteratura biomedica, biomeccanica, ingegneria medica) con svariate combinazioni di materiali e metodi. 

Per quanto riguarda i materiali, sono state proposte soluzioni basate sull'osservazione diretta di un fisiatra; basate sulla stereofotogrammetria, sistemi di telecamere ad emissione di luce infrarossa e marcatori riflettenti; basate su strumenti inerziali, sensori fisici: accelerometri, giroscopi, elettromagnetometri. Questi ultimi sono stati usati in diverse combinazioni, numero e disposizione sul corpo.\TODO{citazione}\\ 

Per quanto riguarda i metodi, sono state proposte soluzioni di tipo cinematico basato sullo studio delle forze che agiscono sul corpo nella deambulazione; di tipo analitico (studio di funzioni e curve) sui segnali di sensori inerziali; alle Macchine a stati finiti per lo studio di sequenze temporali, e pi� di recente all'Apprendimento Automatico (Reti Neurali, Logica Fuzzy) per la capacit� di astrarre sulle variazioni dei singoli individui.\TODO{citazione}

% -------------------------------� stato gia stato risolto? 
Nonostante il vasto numero di lavori, non � ancora stata data una soluzione soddisfacente al problema. 
Tutti i metodi proposti peccano di dipendenza dagli strumenti che usano, vale a dire che variando questi ultimi, variano le prestazioni dei metodi e di dipendenza dai soggetti sui quali vengono fatti gli esperimenti.\TODO{citazione}\\

%------------------------- Come voglio affrontare il problema? 
La scelta dell'utilizzo delle HMM � giustificata dai seguenti motivi. 
Le HMM (vedi appendice \ref{cap:hmm}) sono uno strumento stocastico di riconoscimento di schemi (\textit{Stochastic Pattern Recognition}) usato in campi come il riconoscimento vocale \cite{tutorial_hmm_application_speech_recognition} e lo studio della visione artificiale per il riconoscimento gestuale \cite{HMM_gesture_recognition}. Uno studio dimostra il potenziale delle HMM per la segmentazione della deambulazione equina \cite{stride_segmentation_technique_hmm}. \\

Inoltre vi sono studi che usano le HMM come struttura gerarchica per affrontare il problema della classificazione di attivit� umane \cite{human_physical_activity_classification_ml, spatio_temporal_params_gait_gyr}. Per sviluppi futuri, del lavoro qui presentato, nella direzione della risoluzione del problema della classificazione, gli studi citati sono a favore dell'utilizzo delle HMM.

\chapter{Cenni alla Chinesiologia}
%\myChapter{Cenni alla Chinesiologia}
\label{cap:chinesiologia}
$\kappa\iota\nu\eta\sigma\iota\varsigma \quad (kin\bar{e}sis)$:  mobilit�,\\
$\lambda o\gamma \iota \alpha \quad (logia)$: studio di\\

\begin{quotation}
%The study of the principles of mechanics and anatomy in relation to human movement.
Lo studio dei principi su cui si fondano la meccanica e l'anatomia del movimento umano. 
\end{quotation}
\begin{flushright}
\emph{www.merriam-webster.com}\cite{merriam-webste_Online}
\end{flushright}
\vspace{1cm}

La Chinesiologia studia il movimento umano sotto diversi aspetti: biomeccanico, del controllo motorio e della psicologia del moto.\\
L'approccio biomeccanico \cite{Hall_basic_biomechanics} consiste nell'applicazione dei principi della Meccanica allo studio di organismi viventi: principalmente propriet� fisiche di materiali biologici, segnali biologici, modellazioni e simulazioni biomeccaniche.

Per restringere l'ambito, noi ci concentriamo sulle interazioni biomeccaniche dell'apparato locomotore (scheletro e muscoli).
La branca della Meccanica Classica (vedi appendice \ref{meccanica_classica}) 
che viene utilizzata dalla Biomeccanica � la Cinematica \cite{Encyclopedia_Britannica_online}
che si occupa di descrivere la posizione ed il moto di oggetti nello spazio, senza riferimento alle forze o masse coinvolte (vale a dire alle cause e agli effetti di tale moto). 

\section{Analisi dei movimenti del corpo umano}
Per un trattamento rigoroso dei movimenti del corpo umano, � necessario stabilire dei piani di riferimento, lungo i quali collocare le diverse parti del corpo (vedi figura \ref{fig:HumanBodySPL}).

\begin{figure}
	\centering
		\includegraphics[width=0.90\textwidth]{imgs/Human_body_by_Da_Vinci-SPL.jpg}
	\caption{Immagine riadattata. Originale cortesia di \url{http://sciencephoto.com} }
	\label{fig:HumanBodySPL}
\end{figure}

\begin{definition}[Piani che tagliano il corpo umano]
I movimenti del corpo umano vengono descritti in riferimento a tre piani:
	\begin{enumerate}
		\item \textbf{Frontale o Coronale}:  piano verticale che divide il corpo in parte anteriore e posteriore.
		\item \textbf{Sagittale}: piano verticale che divide il corpo in parte sinistra e destra.
		\item \textbf{Trasversale o Orizzontale}: piano orizzontale che divide il corpo in parte superiore ed inferiore. 
	\end{enumerate}
\end{definition}
Ad esempio la deambulazione o corsa avviene principalmente lungo il piano sagittale, sollevare le braccia lateralmente comporta un movimento sul piano frontale, mentre la rotazione della testa per guardarsi intorno avviene principalmente lungo il piano trasversale. 


\section{L'andatura durante la deambulazione}
Viene definita Andatura (Gait) la sequenza di movimenti degli arti inferiori che un animale compie su una superficie solida durante la locomozione \cite{Herbrand_gait}. Gli animali possiedono diverse forme di andatura che scelgono in base alla velocit�, ed al terreno ed ad altre variabili. \\
La Deambulazione (o Camminata) � una delle principali forme di andatura degli animali aventi arti inferiori ed avviene tipicamente a velocit� inferiore a quelle della Corsa (che a sua volta � una forma di locomozione).  
\begin{definition}[Deambulazione Normale]
\label{deambulazione_normale}
Una Deambulazione normale (negli esseri umani) � composta di due macro fasi per ciascun piede: 
\begin{enumerate}
	\item \STANCE (stance), in cui il piede supporta tutto il peso del corpo e
	\item \SWING (swing), in cui il piede � in aria e porta avanti il baricentro del corpo, mentre il peso del corpo � sull'altro piede.
\end{enumerate}
I due arti inferiori sono sempre alternativamente nelle due fasi e per circa il 25\% del tempo sono in contatto simultaneo con il pavimento.
\end{definition}
\section{Breve storia dello studio della deambulazione umana}
Un primissimo contributo allo studio dell'andatura umana � stato dato dai  fratelli Wilhelm Weber, fisico, e Eduard Weber, anatomista. I Weber, nel loro libro \emph{The Mechanics of Human Motions} \cite{Weber_mechanics_human_motion}, pubblicato nel 1836, definiscono e misurano per la prima volta la durata delle fasi della Deambulazione Normale (vedi definizione \ref{deambulazione_normale}), usando solamente un cronometro ed un telescopio con una scala.\\
Un grande contributo a questo campo � stato dato dal fisiologo francese $\grave{E}$tienne Jules Marey, che nel 1873 pubblic� il trattato  \emph{Animal Mechanism: a Treatise on Terrestrial and Aerial Locomotion} dove con l'uso di scarpe a camera d'aria collegate a un registratore e della {Cronofotografia Geometrica}\footnote{pi� riprese fotografiche vengono impresse sulla stessa fotografia, in modo che possa essere ripresa una sequenza di azioni della persona. Si tratta di un antenato della cinepresa.} e con l'uso di soggetti vestiti con abiti aderenti e neri con bottoni di metallo e strisce riflettenti, riusci� a misurare la durata del contatto del piede con il suolo, durante la camminata in piano, su un terreno regolare. Inoltre egli introdusse il concetto dell'efficienza energetica del movimento.\\
Il fotografo inglese Edward Muybridge, nel 1887 con l'uso della fotografia seriale, con 48 fotocamere elettriche sincronizzate riusci� a catturare la fase in volo di un cavallo al galoppo.\\
L'anatomista tedesco Wilhelm Braune, ed il matematico tedesco Otto Fisher, negli anni 1890, con l'uso di un sistema a 4 fotocamere, un tubo luminoso attaccato al corpo ripreso ed un sistema di riferimento rettangolare, riuscirono a analizzare per la prima volta l'andatura in 3D ed a stabilire i metodo per il calcolo dei parametri meccanici dello stesso. \\
Nel 1938 Elftman e 20 anni dopo Frankel diedero grossi contributi agli studi di tipo cinetico sul {passo normale e patologico}\footnote{passo affetto da una qualunque tipo di deformazione rispetto al passo normale}.\\
M.P. Murray, negli anni '60, con l'uso della fotografia a luce interrotta e pi� tardi con il sistema 3D, diede dei contributi allo studio cinetico in pazienti normali e patologici. \\
J Perry con l'uso di goniometri elettronici monoassiali ha sviluppato un nuovo sistema di terminologie per l'andatura sia normale che patologica. 

\section{Ciclo di deambulazione (Gait Cycle)}
L'unit� di base per la descrizione dell'andatura nella deambulazione � il ciclo di andatura (Gait Cycle) (vedi figura \ref{fig:CicoloDelPasso}). Si tratta della sequenza di azioni compiute dal corpo dall'istante del contatto di un tallone a terra fino al contatto successivo dello stesso tallone. Il ciclo di andatura ha una durata compresa nell'intervallo di [0.5-2] secondi, in base alla velocit�. Il ciclo di andatura � suddiviso in due fasi: una di appoggio, in cui il piede � a contatto con il terreno, ed una aerea. 
\begin{figure}[h]
	\centering
		\includegraphics[width=0.95\textwidth]{imgs/CicoloDelPasso.jpg}
	\caption{Ciclo completo di andatura }
	\label{fig:CicoloDelPasso}
\end{figure}

La fase di appoggio (dal contatto del tallone al distacco dell'alluce dal terreno) viene ulteriormente suddivisa in tre sottofasi: 
\begin{enumerate}
	\item contatto iniziale;
	\item appoggio intermedio (detta anche fase di risposta al carico);
	\item fase propulsiva (o di contatto finale).
\end{enumerate}
 

%\section{Gait Cycle}
\begin{figure}[h]
	\centering
		\includegraphics[width=0.95\textwidth]{imgs/CicoloDelPassoAppoggio.jpg}
	\caption{Ciclo completo di andatura }
	\label{fig:CicoloDelPassoAppoggio}
\end{figure}

Anche la fase in cui il piede � in aria, viene suddivisa in tre sottofasi:
\begin{enumerate}
	\item propulsione iniziale (initial swing), 
	\item propulsione mediale (mid swing) 
	\item decelerazione (terminal swing). 
\end{enumerate}

\subsection{Confronto fra Falcata (Stride) e Passo (Step)}

La falcata, che va dal contatto del tallone con il suolo fino al successivo contatto dello stesso piede, � sinonimo di ciclo di andatura (Gait Cycle), mentre il passo comincia dal contatto di un tallone e termina al contatto dell'altro tallone. Una falcata coincide esattamente con due passi (vedi figura \ref{fig:AndaturaConfrontoPasso}).
\begin{figure}[h]
	\centering
		\includegraphics[width=0.95\textwidth]{imgs/AndaturaConfrontoPasso.jpg}
	\caption{Andatura e Passo a confronto}
	\label{fig:AndaturaConfrontoPasso}
\end{figure}

\section{Parametri che descrivono il pattern (schema) della deambulazione}
%\begin{wraptable}{r}[-9mm]{0.30\textwidth}
\begin{table}%
\begin{tabular}{|p{3cm}|p{2cm}|p{5cm}|}
\hline
\textsc{nome parametro} & \textsc{unit� di misura} & \textsc{descrizione}\\
\hline
\hline
Andatura &(sec)& durata di un completo ciclo di andatura\\
\hline
Passo &(sec)& durata di un completo passo sinistro o destro\\
\hline
Contatto &(sec, \%)& durata del periodo in cui i piedi rimangono a contatto con il terreno\\
\hline
Contatto a piede singolo &(sec, \%)&durata del periodo in cui solo un piede rimane a contatto con il terreno\\
\hline
Doppio contatto &(sec, \%)& durata del periodo in cui entrambi piede rimangono contemporaneamente a contatto con il terreno\\
\hline
Fase aerea &(sec, \%)& durata del perodo in cui il piede � in aria\\
\hline
\end{tabular}
\newline
\caption{Parametri temporali}
\label{parametri_temporali}
\end{table}
%\end{wraptable}


\begin{table}%
\begin{tabular}{|p{3cm}|p{2cm}|p{5cm}|}
\hline
\textsc{nome parametro} & \textsc{unit� di misura} & \textsc{descrizione}\\
\hline
\hline
Lunghezza dell'andatura &(cm)& distanza tra due punti di contatto successivi dello stesso tallone\\
\hline
Lunghezza del passo &(cm)& distanza tra due punti di contatto successivi di talloni opposti\\
\hline
Larghezza del passo &(cm)&distanza laterale tra due punti di contatto successivi di talloni opposti\\
\hline
Angolo del piede &(grad)& angolo tra il collo del piede e lo stinco \\
\hline
\end{tabular}
\newline
\caption{Parametri spaziali}
\label{parametri_spaziali}
\end{table}


\begin{table}%
\begin{tabular}{|p{3cm}|p{2cm}|p{5cm}|}
\hline
\textsc{nome parametro} & \textsc{unit� di misura} & \textsc{descrizione}\\
\hline
\hline
Cadenza & (passi/ min) & numero di passi al minuto  \\
\hline
Velocit� del passo &(m/s) & numero di metri percorso al secondo \\
\hline
\end{tabular}
\newline
\caption{Parametri di velocit�}
\label{parametri_velocit�}
\end{table}

\chapter{Lo stato dell'arte}
%\myChapter{Lo stato dell'arte}
Il mondo scientifico che si � confrontato con il problema della segmentazione della deambulazione, e pi� in generale sull'analisi della deambulazione ed individuazione delle sue varie fasi, � molto vasto e variegato. 
Per avere una visuale completa sul panorama di ricerca in questo ambito, si pu� suddividere lo stato dell'arte per materiali e metodi usati. 


\section{Materiali} 
Per materiali si intende tutto l'arsenale fisico, usato per captare, misurare e raccogliere dati riguardanti la deambulazione. I materiali possono, a loro volta, essere suddivisi in quelli che permettono di fare misure dirette dei parametri della deambulazione, ovvero spostamento, velocit� e accelerazione sia lineare che angolare di arti e articolazioni; e quelli che permettono di fare misurazioni indirette della deambulazione con sistemi di tracciamento del movimento.

\subsection{Osservazione diretta del paziente} Il fisiatra � figura medica che si occupa di diagnosi, terapia e riabilitazione di persone con problemi di limitazioni di attivit�, in questo caso motorie. Un fisiatra � in grado, ad occhio nudo, di fare una stima accurata dei parametri della deambulazione umana. 

\subsection{Stereofotogrammetria} 
\label{sec:stereofotogrammetria}
Metodologia di ricostruzione di oggetti ed del loro moto in tre dimensioni basato sulla sovrapposizione di immagini riprese da pi� telecamere posizionate intorno all'oggetto (vedi figura \ref{fig:vicon}). Sull'oggetto vengono posizionati dei marcatori riflettenti (vedi figura \ref{fig:markers}) che sono i punti di riferimento per le telecamere. I marcatori sono dei bulbi sferici in plastica retroriflettenente (superficie che riflette la luce alla sua sorgente minimizzando l'angolo di riflessione quindi minimizzando anche la dispersione di luce) di circa 14mm di diametro, attaccati ad una superficie quadrata della stessa dimensione, alle quali viene applicato del nastro biadesivo, mediante il quale vengono incollate sulla parte del corpo che vuole studiare. La ricostruzione del moto dell'oggetto avviene mediante il calcolo degli spostamenti relativi dei marker. Il sistema commerciale utilizzato in questo lavoro � Vicon\tm \cite{vicon.com}.
\TODO{METTERE UNA NOSTRA FOTO}
\begin{figure}
	\centering
		\includegraphics[width=.4\textwidth]{imgs/ReflectiveMarker.jpg}
	\caption{Marcatori riflettenti per il sistema Vicon\tm.}
	\label{fig:markers}
\end{figure}

\begin{figure}
	\centering
		\includegraphics[width=1\textwidth]{imgs/viconMotionCapture.jpg}
	\caption{Strumentazione del sistema stereofotogrammetrico Vicon\tm}
	\label{fig:vicon}
\end{figure}


\subsection{Sensori Inerziali}
\label{sensori_inerziali}
I sensori inerziali misurano, le forze inerziali (vedi appendice \ref{meccanica_classica}) che agiscono su di essi (vedi appendice \ref{sec:sensori}). 
\begin{itemize}
	\item \textbf{Resistori Sensibili alla forza}: misurano la forza esercitata dalla massa della persona, sul piano di deambulazione, accelerata dalla forza di gravit�. 
	
\begin{table}%
\begin{tabular}{p{3cm}|p{8cm}}
\hline
\textsc{Sensore} & FSR (\textit{Force Sensitive Resistors})\\
\hline
\textsc{Posizione del sensore} & sotto le scarpe\\
\hline
\textsc{Hardware} & trasduttori (vedi appendice \ref{sec:sensori}) progettati come degli interruttori meccanici, sensibili al peso FSR\\
\hline
\textsc{limitazioni} & 
	\begin{itemize}
			\item non � possibile distinguere le variazioni di peso dovute alla deambulazione da quelle dovute allo spostamento di peso di altra natura. 
			\item passi tremolanti riducono l'affidabilit� della percezione di un evento che stabilisca l'inizio o la fine di una fase del passo.  
	\end{itemize} \\
\hline
\end{tabular}
\newline
\caption{Sensori di forza: FSR}
\label{tab:sensori_forza}
\end{table}
		
	\item \textbf{Giroscopi} misurano l'accelerazione (momento) angolare. Questo pu� essere fatto meccanicamente con un giroscopio a rotazione o elettronicamente con un giroscopio a vibrazione (vedi appendice \ref{sec:gyroscope}). 

\begin{table}
\begin{tabular}{p{3cm}|p{8cm}}
\hline
\textsc{Sensore} & Giroscopio\\
\hline
\textsc{Posizione del sensore} & coscia, stinco, coscia e stinco\\
\hline
\textsc{Hardware} & Giroscopi\\
\hline
\textsc{vantaggi}&
\begin{itemize}
	\item non influenzati dalla forza di gravit�;
	\item non influenzati da vibrazioni o scosse dovute alla deambulazione;
	\item meno influenzate (rispetto alle FRS) dal posizionamento sul corpo.
\end{itemize}\\
\hline
\textsc{limitazioni} & 
	\begin{itemize}
			\item non � possibile distinguere le variazioni di peso dovute alla deambulazione da quelle dovute allo spostamento di peso. 
			\item passi tremolanti riducono l'affidabilit� della percezione di un evento che stabilisca l'inizio o la fine di una fase del passo.  
	\end{itemize}\\  
\hline
\end{tabular}
\newline
\caption{Giroscopi}
\label{tab:sensori_acc_ang}
\end{table}

	\item \textbf{Accelerometri} misurano il peso per unit� di massa, questa quantit� � nota come forza specifica o \textit{g-force}. In altre parole, misurando il peso il sensore misura l'accelerazione in un riferimento inerziale relativo all'accelerometro stesso.  

\begin{table}
\begin{tabular}{p{3cm}|p{8cm}}
\hline
\textsc{Sensore} & Accelerometro\\
\hline
\textsc{Posizione del sensore} & piede, collo del piede, stinco, polpaccio, coscia, schiena\\
\hline
\textsc{Hardware} & Accelerometri MEMS (\textit{Micro-Electro-Mechanical Systems})\\
\hline
\textsc{Vantaggi}&
\begin{itemize}
	\item permettono di individuare quasi tutte le fasi del cammino;
	\item sono dimensioni ridotte e costano poco.
\end{itemize}\\
\hline
\textsc{limitazioni} & 
	\begin{itemize}
		\item influenzati dalla forza di gravit�;
		\item il loro posizionamento sul corpo � critico.
	\end{itemize}\\  
\hline
\end{tabular}
\newline
\caption{Accelerometri: MEMS}
\label{tab:sensori_acc}
\end{table}
	\item \textbf{Magnetometri} misurano il campo magnetico. Vi sono pi� modi per misurare il campo magnetico, quello maggiormente usato detto \textit{Fluxgate} funziona mediante un meccanismo elettrico (vedi appendice \ref{sec:magnetomerter}).
\end{itemize}

\begin{table}
\begin{tabular}{p{3cm}|p{8cm}}
\hline
\textsc{Hardware} & Magnetometro\\
\hline
\textsc{Forza misurata} & il campo magnetico terrestre\\
\hline
\textsc{Posizione del sensore} & piede, collo del piede, stinco, polpaccio, coscia, schiena\\
\hline
\textsc{Vantaggi}&
\begin{itemize}
	\item il campo magnetico, non essendo influenzato dal movimento della persona o dalla forza di gravit�, offre un punto di riferimento per l'orientamento del corpo, e le misure hanno un'accuratezza maggiore. 
	\item In combinazione con altri sensori giroscopio permette di determinare 5 fasi del cammino. 
\end{itemize}\\
\hline
\textsc{limitazioni} & 
	\begin{itemize}
		\item influenzati dall forza di gravit�; 
		\item il loro posizionamento sul corpo � critico.
	\end{itemize}  \\
\hline
\end{tabular}
\newline
\caption{Magnetometri}
\label{tab:sensori_mag}
\end{table}


I sensori qui presentati sono stati usati per l'analisi della deambulazione in diversi numeri e combinazioni. Una classificazione pu� essere fatta sulla base della grandezza fisica misurata dal singolo sensore o dall'insieme di sensori scelti. Tale grandezza fisica pu� essere:
\begin{description}
	\item [Forza] Il corpo umano esercita una forza (peso) sulla superficie su cui si trova. Un sensore per misurare tale forza, deve essere posizionato, tra il corpo e la superficie. La collocazione naturale di un sensore di forza � sotto la suola delle scarpe. 
I trasduttori \footnote{Un trasduttore � un dispositivo che converte un tipo di energia in un altro.} per tale scopo vengono progettati interruttori meccanici dipendenti dal peso o FSR.

Metodi di riabilitazione o di aiuto di persone con la sindrome del piede cadente (foot drop) basati sulla Stimolazione Elettrica Funzionale (FES) usavano inizialmente le FSR \cite{FES_foot_drop_correction}. 
Mentre i primi sistemi di questo tipo usavano microinterruttori puramente meccanici, ovvero azionati direttamente dalla pressione del piede, sistemi pi� recenti sono azionati da valori soglia (threshold) ottenuti mediante uno o pi� trasduttori FSR posizionati sotto il tallone, il metatarso e l'alluce \cite{leg_muscle_electical_stimulator, FES_foot_drop_correction}. Questi metodi a differenza dei primi, sono meno soggetti ad attivazioni dovute a spostamenti di peso non dovuti alla deambulazione. Con un interruttore singolo sotto il tallone si riescono ad ottenere informazioni sugli eventi HO e HS. Aggiungendo ulteriori sensori sotto il metacarpo o l'alluce si hanno informazioni sugli eventi FF e TO \cite{FES_foot_drop_correction}. Un altra tecnica che � stata usata per misurare la forza � usare suole sensibili alla forza, che ricoprono l'intera suola con una matrice di sensori \cite{real_time_parap_gait_event_detection}. L'accuratezza ed affidabilit� di tali sistemi dipende soprattutto dai materiali usati.\\
Un'altra tecnica � basata su un sensore di pressione connesso ad un piccolo tubo, incollato al perimetro esterno della scarpa. In questo modo le variazioni nella forza esercitata sulla suola si manifestano in variazioni pneumatiche, che vengono poi misurate dal sensore di pressione \cite{foot_contact_masure}.  \\
Sistemi di individuazione di eventi di deambulazione basati su sensori di forza, sono considerati abbastanza standard per essere usati come riferimento per determinare l'accuratezza di metodi novelli per la medesima funzione. 
Ci� � dovuto al fatto che un sensore posto sotto i piedi sembri essere la scelta pi� naturale per sistemi di misura della deambulazione. 

Danno risultati soddisfacenti per la deambulazione normale {vedi definizione \ref{deambulazione_normale}}. Gli svantaggi dei sensori della forza sono l'impossibilit� di discernere tra variazioni di carico dovute alla deambulazione da quelle dovute a spostamenti del peso (ad esempio da una gamba all'altra). Una deambulazioni strascicata riduce notevolmente l'affidabilit� della individuazione degli eventi della deambulazione \cite{heel_contact_event_detection}. Altri difetti di questi sistemi sono la breve durata e la limitata applicabilit� \cite{swing_phase_detection_stroke_gait}.

	\item [Accelerazione angolare] La maggior parte degli studi che hanno usato un giroscopio hanno optato per lo stesso tipo di sensore unidimensionale nella configurazione a singolo sensore \cite{stair_climbing_detection_gyr, practical_gait_analysis_gyr, unrestrained_measurement_stride_length_gyr} o tre sensori \cite{spatio_temporal_params_gait_gyr}. In tutti i casi i dati sono stati elaborati in differita (vedi appendice \ref{sec:online}). Sono state testate innumerevoli combinazioni di posizionamenti dei sensori: coscia \cite{unrestrained_measurement_stride_length_gyr}, stinco \cite{stair_climbing_detection_gyr}, coscia e stinco \cite{practical_gait_analysis_gyr} e su entrambe gli stinchi ed una sola coscia \cite{spatio_temporal_params_gait_gyr}.\\
	Dal segnale del sensore sono stati calcolati l'angolo dell'articolazione del ginocchio \cite{practical_gait_analysis_gyr}, l'angolo dell'articolazione dell'anca mediante l'integrazione della velocit� angolare \cite{unrestrained_measurement_stride_length_gyr}. Uno dei problemi pi� comuni del giroscopio � il drift (deviazione) causata da una imprecisa calibrazione dello strumento.
	Un altro approccio � il cosiddetto studio di \textit{Wavelet}: la stima del HS, e HO mediante lo studio analitico del segnale del giroscopio in funzione del tempo \cite{spatio_temporal_params_gait_gyr, stair_climbing_detection_gyr}.
	La validazione del sistema � stata fatta su ciascun sistema, confrontando i risultati ottenuti mediante il sistema stereofotogrammetrico Vicon\tm (vedi sezione \ref{sec:stereofotogrammetria}), in condizioni di laboratorio.
	Un grande vantaggio della analisi del moto in generale mediante il giroscopio, � che questi non subisce l'effetto dell'accelerazione di gravit� \cite{gait_event_detection_FES_acc_ML}. Inoltre le vibrazioni dei sensori durante gli esperimenti non influenzano il giroscopio \cite{kimematics_measurements_acc_gyr}. I giroscopi, rispetto agli FSR, sono meno sensibili al posizionamento sul corpo, un giroscopio posto in qualunque posizione di un segmento del corpo lungo lo stesso piano fornisce con variazioni minime gli stessi valori. Infine nessun tipo di movimento che non riguardi l'asse che percepisce il sensore viene catturato dallo strumento \cite{practical_gait_analysis_gyr}. 
	
	\item [Forza e Accelerazione angolare] Al fine di individuare gli eventi HS e HO, � stato messo a punto un sistema composto da tre FSR per catturare il carico verticale ed un giroscopio per misurare la velocit� di rotazione del piede sul piano sagittale (vedi figura \ref{fig:HumanBodySPL}) \cite{reliable_gait_phase_anslysis}. Una volta posizionati gli FSR sotto il tallone, il primo ed il quarto metatarso ed il giroscopio monoassiale sul tallone, il sistema era in grado di individuare le fasi di appoggio e fase in aria in linea. La validazione � stata fatta in condizioni di laboratorio, anche questa volta con sistemi di cattura del movimento. Il sistema � stato incassato in una suola per poter essere usato con un meccanismo di FES di correzione della ricaduta del piede \cite{reliable_gait_phase_detection_gyr}. I vantaggi principali del sistema qui descritto sono l'affidabilit� e la robustezza: riesce a distinguere semplici spostamenti di peso (ad esempio stando fermi, alzandosi o sedendosi) da variazioni di carico dovute alla deambulazione. I difetti del sistema sono collegati ai difetti degli FSR: la loro poca durata.
	\item [Accelerometria] La tecnologia MEMS (Micro-Electro-Mechanical Systems) (vedi appendice \ref{sec:sensori}) permette uno sviluppo di accelerometri in miniatura a basso consumo energetico, adatti al monitoraggio della deambulazione    \cite{walking_movement_pattern_quantification_acc}. I sensori usati negli esperimenti, misurano l'accelerazione in due  \cite{heel_contact_event_detection, swing_phase_detection_stroke_gait, human_body_movement_measurement_acc, real_time_gait_assessment_acc, lower_extreamity_angle_measurement_acc} o tre dimensioni  \cite{gait_event_detection_FES_acc_ML, walking_restoration_assistive_sensors, novel_approach_ambulatory_systems}, avendo pi� sensori monoassiali \cite{gait_event_detection_FES_acc_ML, human_body_movement_measurement_acc, real_time_gait_assessment_acc}, biassiali \cite{heel_contact_event_detection, swing_phase_detection_stroke_gait, lower_extreamity_angle_measurement_acc, gait_phase_automatic_recognition_acc} o triassiali  \cite{novel_approach_ambulatory_systems, walking_restoration_assistive_sensors}. Per misurare sia l'accelerazione lineare che rotazionale, i sensori vengono sistemati su coscia \cite{swing_phase_detection_stroke_gait}, stinchi \cite{gait_event_detection_FES_acc_ML, novel_approach_ambulatory_systems, human_body_movement_measurement_acc}, coscia e stinchi \cite{real_time_gait_assessment_acc, gait_phase_automatic_recognition_acc}, coscia, stinchi e bacino \cite{lower_extreamity_angle_measurement_acc}, piede, coscia, stinchi e bacino \cite{walking_restoration_assistive_sensors} o busto \cite{heel_contact_event_detection}.\\
	L'elaborazione dei dati provenienti dai suddetti sensori � stata fatta sia in modalit� in linea che in differita. L'elaborazione in differita comprende l'analisi temporale dell'accelerazione verticale nel piano sagittale, i cambiamenti da positivo a negativo e viceversa, che sono stati correlati rispettivamente agli eventi HS e HO \cite{heel_contact_event_detection}. Un altro approccio ha dimostrato che � possibile rilevare gli eventi TO, ISw, TSw, HS sulla base dei dati della velocit� angolare e lineare sul piano sagittale e frontale \cite{human_body_movement_measurement_acc}. \\
	Per quanto riguarda i sistemi in linea � stato dimostrato che permettano di rilevare gli eventi LO, MS, TS, PSw \cite{gait_event_detection_FES_acc_ML, gait_phase_automatic_recognition_acc} oppure HS e HO \cite{swing_phase_detection_stroke_gait}. L'uso  di accelerometri necessita di elaborazioni ulteriori del segnale per compensare all'influenza della forza di gravit�. Il posizionamento dei sensori � un altro aspetto delicato, dovuto al movimento di muscoli durante la deambulazione. Un vantaggio degli accelerometri, � che l'evento HO � individuabile anche in individui che non hanno un contatto con il tallone molto definito \cite{gait_event_detection_FES_acc_ML} o con deambulazione strascicata \cite{heel_contact_event_detection}. 
	\item [Accelerometria e Accelerazione angolare] Il metodo di misurazione pu� essere basato su un modello bidimensionale su un piano sagittale \cite{kimematics_measurements_acc_gyr, uniaxial_joint_angles_measure_acc_gyr, walking_features_from_inertials, gait_event_detection_acc_gyr, reliability_acc_gyr_gait_event_id, wearable_sensor_sys} oppure tridimensionale \cite{kinematic_transients_locomotion, 3d_sensing_foot_movement}. Per modelli bidimensionali il gruppo di sensori pu� essere composto da due sensori inerziali unidimensionali ed un giroscopio bidimensionale \cite{kimematics_measurements_acc_gyr} oppure da un unico sensore inerziale bidimensionale che misuri le componenti tangenziali e radiali dell'accelerazione \cite{uniaxial_joint_angles_measure_acc_gyr}. I sistemi tridimensionali usano di solito un giroscopio tridimensionale ed un sensore inerziale tridimensionale \cite{kinematic_transients_locomotion, 3d_sensing_foot_movement}. Le unit� di sensori sono montati su piedi \cite{walking_features_from_inertials, 3d_sensing_foot_movement}, stinco e coscia \cite{kimematics_measurements_acc_gyr, uniaxial_joint_angles_measure_acc_gyr, reliability_acc_gyr_gait_event_id}, o piedi, stinchi e cosce \cite{wearable_sensor_sys, kinematic_transients_locomotion}. I dati sono stati elaborati primariamente in differita ed hanno fornito informazioni sull'accelerazione di segmenti di arti, velocit� di giunture ed eventi quali HS, TO e FF. La validazione di questi sistemi � stata fatta su soggetti sani \cite{kimematics_measurements_acc_gyr, uniaxial_joint_angles_measure_acc_gyr, walking_features_from_inertials, wearable_sensor_sys} e con problemi di deambulazione. I dati ottenuti sono stati confrontati con quelli prodotti da un sistema di ripresa in movimento : WATSMART\tm \cite{kinematic_transients_locomotion}, Vicon\tm \cite{kimematics_measurements_acc_gyr}, interruttori a piede \cite{walking_features_from_inertials, gait_event_detection_acc_gyr, reliability_acc_gyr_gait_event_id}. 
	
	
	\item [Accelerometria, Accelerazione angolare e Campo magnetico] La misurazione del campo magnetico terrestre mediante un magnetometro fornisce il campo gravitazionale terrestre ed una misura di riferimento per l'orientamento del corpo. Al contrario dell'accelerometria, il campo magnetico non � influenzato dai movimenti del corpo.  Sono state prodotte unit� contenenti sensori sia uniassiali \cite{sensor_sys_lower_limbs_FES_control} che biassiali \cite{inertial_magnetic_sensor_joint_angle_measure} in combinazione con un magnetometro. 
I sensori sono stati posizionati al piede e stinco di una gamba oppure di entrambe le gambe.
I sistemi sono in grado di determinare angoli tra diverse articolazioni in tre dimensioni mediante l'analisi dei dati in differita \cite{inertial_magnetic_sensor_joint_angle_measure} o mediante l'analisi tempo reale \cite{sensor_sys_lower_limbs_FES_control}. Tali sistemi hanno permesso di individuare cinque eventi della deambulazione: LO, MS, TS, PSw e fase in aria con l'uso di misure dell'accelerazione e della sua derivata prima \cite{sensor_sys_lower_limbs_FES_control}. Anche gli eventi HS e TO sono stati individuati in corrispondenza di picchi nella rotazione lungo il piano sagittale dello stinco. Un modello di un doppio pendolo pu� essere usato per calcolare la lunghezza di un passo \cite{spatio_temporal_gait_paramerters}.
	\item [Inclinometria] Un sensore di inclinazione pu� essere usato per determinare l'inclinazione di una parte del corpo. Consiste di un elemento inerziale che percepisce la forza di gravit�, come un liquido o un sistema costituito da una massa ed una molla. Quando il sensore � inclinato, la forza di gravit� causa uno spostamento dell'elemento inerziale relativo al sistema del sensore. Tale spostamento viene registrato in una resistenza o capacitanza. Questi sistemi non sono abbastanza affidabili perch� non distinguono la deambulazione da altri tipi di spostamenti del piede \cite{reliable_gait_phase_detection_gyr}. Inoltre rispondono ad accelerazioni, il che produce degli errori grossolani nei loro dati. 
\end{description}
				
\section{Metodi}
La grande sfida del rilevamento della deambulazione � di individuare gli eventi della deambulazione mentre la persona sta camminando ovvero in linea. Tradizionalmente il problema � stato affrontato con un insieme di regole basate su valori di soglia sulle emissioni, principalmente di interruttori da piede, e da tutti gli altri sistemi di sensori descritti in \ref{sensori_inerziali}. Tutti gli algoritmi pubblicati, consistono di gruppi di euristiche che cercano di identificare alcune caratteristiche dei parametri della deambulazione. Le regole sono state applicate con i seguenti approcci: analisi funzionale di dati non elaborati \cite{spatio_temporal_params_gait_gyr, real_time_gait_event_det_wearable, automatic_detection_gait_events_kinematic}, di dati derivati \cite{reliable_gait_phase_detection_gyr}, tecniche di apprendimento induttivo \cite{real_time_parap_gait_event_detection, swing_phase_detection_stroke_gait, gait_event_detection_FES_acc_ML, automatic_detection_gait_events_inductive_learning} oppure macchine a stati finiti \cite{reliable_gait_phase_detection_gyr, finite_state_control_FES}.
%\subsection{Meccanica-Cinematica} 
\subsection{Analisi} L'analisi funzionale comprende metodi matematici per disegnare curve che permettono di estrarre caratteristiche corrispondenti a determinate fasi della deambulazione. Le derivate prime e seconde dei dati dell'accelerazione verticale e orizzontale dal piede permettono di determinare gli eventi HS e TO soltanto usando delle regole basate su valori di soglia, con un'elaborazione in linea. \cite{real_time_gait_event_det_wearable}.
\subsection{Apprendimento Automatico} Per Apprendimento Automatico (o Apprendimento Automatico Induttivo) si intende una branca dell'Intelligenza Artificiale che comprende la progettazione di algoritmi che permettono ad un sistema di apprendere estraendo regole e schemi (\textit{pattern}) da un insieme di dati. Le regole base frutto del processo di apprendimento (addestramento) sono un insieme di associazioni tra valori in ingresso come il segnale di un giroscopio, e dei valori in uscita come le fasi della deambulazione. Le regole vengono classificate in base a delle stime di probabilit�. Un esempio di applicazione dei metodi di Apprendimento Automatico al problema dell'analisi del cammino, correla i dati di un accelerometro in ingresso ed i dati relativi al contatto del piede con il suolo come dati in uscita desiderati, con l'utilizzo di Reti Neurali, per determinare gli eventi HS e HO \cite{swing_phase_detection_stroke_gait}. Un altro studio usa programmi commerciali di Apprendimento Automatico detti \textit{Rough Sets\tm} e Reti Logiche Adattive per l'analisi della deambulazione in linea usando come dati in ingresso segnali accelerometrici \cite{gait_event_detection_FES_acc_ML}.  





\addtocontents{toc}{\protect\mbox{}\protect\hrulefill\par}
\addtocontents{toc}{\protect\mbox{}\protect\hrulefill\par}
\part{Lavoro svolto}
\chapter{Modellazione della deambulazione}
\label{cap:modellazioneDeambulazione}
Il lavoro svolto, pu� essere suddiviso in tre fasi principali: 
\begin{enumerate}
	\item \textbf{Creazione ed addestramento di un modello}: scelta di un modello consono al problema e ottimizzazione dei relativi parametri,
	\item \textbf{Sviluppo applicazione}: implementazione di un programma per \textit{Smartphone} Android\tm 	del modello,
	\item \textbf{Valutazione delle prestazioni dell'applicazione in condizioni di uso reali}: il sistema ottenuto � stato valutato correlando un parametro spaziale del cammino (la distanza percorsa) a partire da quelli temporali ottenuti con la segmentazione (cadenza e velocit�) e successivamente verificandone la correttezza con misurazioni dirette dei parametri spaziali tramite GPS (\textit{Global Positioning System}) sistema di posizionamento globale.
\end{enumerate}

\section{Raccolta dati}
Sono stati scelti $6$ soggetti sani (con deambulazione normale, vedi Definizione \ref{def:deambulazione_normale}) e sono stati sottoposti a $5$ sessioni di cammino e $5$ sessioni di corsa.
Le prove sono state compiute su un tappeto rullante con pendenza $0^\circ$, nell'intervallo di velocit� $[3-7]\,km/h$ per il cammino e $[8-12]\,km/h$ per la corsa. La prima sessione alla velocit� di $3\,km/h$ e con un incremento di $1\,km/h$ in ciascuna sessione successiva. Ciascuna sessione � stata della durata di $2\,minuti$.

\begin{table}[htbp]
	\centering
	\begin{tabular}{|l|c|c|}
		\hline		
			\textbf{Attivit�}& cammino & corsa \\
			\hline
			\textbf{Soggetti}& $6$ & $5$\\
			\hline
			\textbf{Velocit�}& $\{3,4,5,6,7\}\,km/h$ & $\{8,9,10,11,12\}\,km/h$\\
			\hline
			\textbf{Durata}& \multicolumn{2}{c|}{$2\, minuti$ per attivit�}\\
			\hline
			\textbf{Strumenti} & \multicolumn{2}{c|}{IMU, Vicon, tappeto rullante}\\
			\hline
			\textbf{Luogo}& \multicolumn{2}{c|}{Laboratorio}\\
			\hline
			\textbf{Dati raccolti} & \multicolumn{2}{c|}{valori giroscopio monoassiale}\\
			\hline
			\textbf{Frequenza campionamento} & \multicolumn{2}{c|}{$100\, Hz$}\\
			\hline
		\end{tabular}
	\caption{Tabella riassuntiva della raccolti dati}
	\label{tab:TabellaRiassuntivaRaccoltaDati}
\end{table}

Le sessioni sono state tutte eseguite posizionando saldamente una IMU (vedi Appendice \ref{sec:sensori}) sul collo del piede dei soggetti mediante un cinturino in velcro. Il tipo di sensore usato � un giroscopio monoassiale con asse di sensibilit� orientato sul piano mediale-laterale o sagittale (vedi Figura \ref{fig:HumanBodySPL}).\\
I segnali sono stati campionati dal sensore ad una frequenza di $100\, Hz$ e filtrati mediante un filtro passa-basso\footnote{Un filtro elettronico nella Teoria dei Segnali � un circuito elettronico che riceve dei segnali in ingresso e li trasforma secondo un criterio. Un filtro passa-banda (banda alta o banda bassa) lascia passare segnali a frequenza in un intervallo scelto mentre blocca il passaggio (o le attenua) il passaggio di segnali fuori dall'intervallo. Un filtro passa-basso quindi lascia passare le frequenze basse, e smorza quelle a frequenze alte.} a 15Hz mediante un filtro ricorsivo \textit{forward-backward}\footnote{Un filtro ricorsivo riusa i propri valori in uscita come ingresso. In particolare un filtro \textit{forward-backward} � un filtro ricorsivo che viene utilizzato per avere un segnale simmetrico.} di Buttersworth\footnote{Filtro detto massimamente piatto, perch� non solo rifiuta i segnali a frequenze non desiderate, ma ha la stessa sensibilit� per tutte le frequenze desiderate.}. 
Dai dati raccolti, quelli considerati per lo studio sono quelli compresi nell'intervallo dai $50\, secondi$ ai $110\, secondi$. \\

\begin{figure}
	\centering
		\includegraphics[width=1\textwidth]{imgs/patternGyro.jpg}
	\caption{Segnale di un giroscopio posizionato sul piede.}
	\label{fig:gyroPattern}
\end{figure}

Il grafico del segnale del giroscopio posizionato sul piede ha una struttura definita e ripetitiva (vedi Figura \ref{fig:gyroPattern}). Nel grafico, l'ascissa rappresenta i campioni e l'ordinata la velocit� angolare in gradi al secondo ($^\circ/s$). La forma del grafico pu� essere descritta a grandi linee come un picco negativo seguito da un tratto approssimativamente orizzontale, un secondo picco negativo ed uno positivo. Tale sequenza rappresenta un passo completo e si ripete quasi identica a se stessa durante la camminata.


\section{Modello: HMM}
\label{sec:modelloHmm}
Il modello di deambulazione scelto � una HMM minimale a 4 stati sinistra-destra ad emissioni continue (vedi Appendice \ref{sec:tipi_hmm}). Tale scelta � stata fatta per poter modellare la deambulazione con il minimo numero di parametri possibile. Inoltre il segnale del giroscopico � approssimativamente suddivisibile anche visivamente in 4 stati.
In dettaglio l'HMM � stata definita nel seguente modo:
 \begin{equation}
	Deamb\_HMM = <N, M, \mathbf{A}, \mathbf{B}, \pi>
\label{eq:deamb_hmm}
\end{equation}
dove:
 \begin{enumerate}
	\item $N = 4$ � la cardinalit� dell'insieme degli stati $S = \{HS, FF, HO, FO\}$. Gli stati corrispondono agli intervalli fra un evento ed il successivo come mostra la Tabella \ref{tab:4_fasi_deambulazione},
	\item $M = |V|$ � la cardinalit� dell'insieme finito dei valori di osservazione $\omega (^\circ/s)$ del giroscopio,
	\item $\mathbf{A}$ � la matrice di transizione (vedi Appendice \ref{cap:hmm}). Dato che si tratta di una HMM sinistra-destra la matrice di transizione gode della seguente propriet�:
\begin{equation}
\begin{split}
	A = \,& (a_{ij})\quad 1\leq i,j \leq N \quad \text{dove}\\ 
	& a_{ij} > 0 \Leftrightarrow  j = i+1\\ 
	& \text{oppure} \quad j = i+2 \\ 
	& \text{oppure} \quad (i = N \quad\text{e}\quad j = 1)
\end{split}
\label{eq:deambTrMtx}
\end{equation}
	\item $\mathbf{B}$ � la matrice di probabilit� di emissione delle osservazioni $\omega$ per ciascuno stato. Ad ogni stato � stata associata una funzione di densit� di probabilit� gaussiana univariata (vedi Figura \ref{fig:deamb_hmm}). Ci� significa che ad ogni stato devono essere associate la media ($\mu$) e varianza ($\sigma$) della distribuzione gaussiana corrispondente.
	\item $\pi$ � il vettore di probabilit� a priori (probabilit� che l'i-esimo stato sia quello in cui si trova il modello al tempo iniziale).
\end{enumerate}


\begin{figure}[h]
	\centering
		\includegraphics[width=1\textwidth]{imgs/HMM.jpg}
	\caption{Rappresentazione della $Deamb\_HMM$. Il modello ha $4$ stati a ciascuno dei quali � associata una densit� di emissione gaussiana ($B$). Le probabilit� di transizione da uno stato al successivo sono prossime a 1 (nell'immagine ci� � rappresentato dalle frecce spesse), mentre le probabilit� di transizione da uno stato a quello due stati dopo � prossima a $0$ (raffigurato con le frecce sottili) e non vi sono altre possibilit� di transizione.}
	\label{fig:deamb_hmm}
\end{figure}


\begin{table}%
\centering
\begin{tabular}{|c|cc|}
\hline
\textbf{Stato} & \textbf{inizio} & \textbf{fine} \\
%\hline
\hline
$S_1$ - $HS$  & $t_{HS}$ & $t_{FF}$\\
%\hline
$S_2$ - $FF$  & $t_{FF}$ & $t_{HO}$\\
%\hline
$S_3$ - $HO$  & $t_{HO}$ & $t_{FO}$\\
%\hline
$S_4$ - $TO$  & $t_{FO}$ & $t_{HS}$\\
\hline	 
\end{tabular}
\caption{Le quattro fasi della deambulazione e gli eventi (tempo iniziale e finale) che li definiscono. Per semplicit� si indica lo stato $S_i$ con il nome dell'evento da cui � stato originato, ad esempio lo stato $S_1$ viene indicato con l'evento $HS$.}
\label{tab:4_fasi_deambulazione}
\end{table}


%%%%%%%%%%%%%%%%%%%%%%%%%%%%%%%%%%%%%%%%%%%%%%%%%%%%%%%%%%%%%%%%%%%%%%%%%%%%%%%%%%%%%%%%%%%%%%%%%%%%%%%%%%%%%%%%%%%%%%%%%
%%%%%%%%%%%%%%%%%%%%%%%%%%%%%%%%%%%%%%%%%%%%%%%%%%%%%%%%%%%%%%%%%%%%%%%%%%%%%%%%%%%%%%%%%%%%%%%%%%%%%%%%%%%%%%%%%%%%%%%%%
\section{Addestramento e Validazione del Modello} 
\label{sec:addestramentoModello}
Come prima operazione per l'addestramento delle HMM, � stato etichettato un sottoinsieme dei dati: ad una sequenza $\Omega$ di dati � stato  associato un vettore $Y$ di etichette. Questo � stato fatto con l'uso del sistema di telecamere Vicon\tm (vedi Sezione \ref{sec:stereofotogrammetria}).

In lavori precedenti \cite{walking_features_from_inertials}, gli eventi ($t_{HS}$, $t_{FF}$, $t_{HO}$, $t_{FO}$ ) vengono determinati mediante soglie, sulla velocit� angolare $\omega_k$, come mostra la Tabella \ref{tab:regole_threshold_eventi}, dove $\widetilde{\omega_k}$ � il valore non filtrato del giroscopio.

\begin{table}%
\centering
\begin{tabular}{|c|l p{5cm}|}
\hline
\textbf{Evento} & \multicolumn{2}{c|} {\textbf{regola}}\\
%\hline
\hline
$t_{HS}$ & $ \omega_k \leq 0 $ e $ min_1 \leq  \omega_k$ & $\displaystyle\max_{\forall k}|\widetilde{\omega_k} - \omega_k|$\\
%\hline
$t_{FF}$ & $|\omega_k|\geq  50$�/s&\\
%\hline
$t_{HO}$ & $|\omega_k|\geq  50$�/s &\\
%\hline
$t_{FO}$ & $\omega_k = 0$ & se $\omega$ da negativo � diventato positivo e cresce fino al suo massimo.\\
\hline	 
\end{tabular}
\caption{Regole basate sulle soglie (\textit{threshold based}) mediante le quali vengono ricavati gli eventi che delimitano le 4 fasi del cammino \cite{walking_features_from_inertials}.}
\label{tab:regole_threshold_eventi}
\end{table}


Nel lavoro presentato i dati acquisiti con la IMU sono stati etichettati con dati acquisiti mediante stereofotogrammetria, procedura ritenuta lo standard di riferimento per lo studio della cinematica del cammino \cite{reliable_gait_phase_analysis}.

Il sistema ottico commerciale di analisi del movimento usato � Vicon\tm 460 (Oxford Metrics Ltd., UK) (vedi Sezione \ref{sec:stereofotogrammetria}).
Con Vicon\tm sono state tracciate e misurate le traiettorie di marcatori retro riflettenti posizionati sul corpo dei soggetti. 
Sono state posizionate $6$ telecamere con una frequenza di campionamento di $100\,Hz$, intorno al tappeto rullante, per tracciare la posizione di $5$ marcatori con un'accuratezza di $\pm 1\,mm$ ed il software Workstation\tm.
Le traiettorie dei marcatori sono state usate per estrarre un segnale di riferimento per le fasi della deambulazione, che successivamente sono state usate per valutare l'accuratezza del segnale in uscita dal sistema di riconoscimento degli eventi.


\begin{table}%
\centering
\begin{tabular}{|p{4cm}|p{6cm}|}
\hline
\textbf{Nome} & \textbf{Posizione} \\
\hline
HM (\textit{Heel Marker}) & lato posteriore del tallone;\\
\hline	 
AM (\textit{Ancle Marker}) & lato esterno della caviglia;\\
\hline
TM (\textit{Toe Marker}) &  giuntura metatarso-falange dell'alluce;\\
\hline
ThM (\textit{Thigh Marker}) & lato esterno della coscia, ad un terzo dell'altezza della coscia a partire dal ginocchio;\\
\hline
LSM (\textit{Lombaro Sacral Marker}) & tra la zona lombare e sacrale.\\
\hline
\end{tabular}
\caption{Posizionamento dei marcatori sul corpo.}
\label{tab:marker_positioning}
\end{table}


Le regole riportate in Tabella \ref{tab:vicon_rules}, Tabella \ref{tab:min_max_HM_TM} e Figura \ref{fig:vicon_traced_TM_HM} usate per generare il segnale di riferimento Vicon\tm si basano sul tracciamento di soli $2$ marcatori: HM e TM.

\begin{table}%
\centering
\begin{tabular}{|p{2cm}|p{4cm}|p{4cm}|}
\hline
&\textbf{min} & \textbf{max} \\
\hline
HM & inizio fase Sw & Evento HS\\
\hline	 
TM & primo picco: fine Fase HO, secondo picco fine fase Sw&  fase \textit{Stance}\\
\hline
\end{tabular}
\caption{Valori minimi e massimi assunti dai marcatori HM e TM.}
\label{tab:min_max_HM_TM}
\end{table}


\begin{table}%
\centering
\begin{tabular}{|p{3.3cm}|p{3.3cm}|p{3.3cm}|}
\hline
\textbf{Evento} & \textbf{Condizione} & \textbf{Posizione in Figura \ref{fig:vicon_traced_TM_HM}} \\
\hline
HO & $y(HM) > 80\,mm$ & A\\
TO & $\max y(HM)$ & B\\
HS & $\min y(HM)$ & C\\
FF & $y'(TM) = 0$ & D\\
\hline
\end{tabular}
\caption{Regole usate per generare il segnale di riferimento Vicon\tm.}
\label{tab:vicon_rules}
\end{table}

\begin{figure}
	\centering
		\includegraphics[width=1\textwidth]{imgs/ViconMeasurementsPappas.jpg}
	\caption{Tracciamento mediante Vicon\tm dei marcatori HM e TM. Immagine riadattata da \cite{reliable_gait_phase_detection_gyr}}
	\label{fig:vicon_traced_TM_HM}
\end{figure}

La stima (dei parametri del modello) della massima verosimiglianza (\textit{Maximum Likelihood Estimation} - MLE) $\ell(\Omega, Deamb\_HMM )$ non � un problema convesso dato. Il problema dei massimi locali viene aggirato con un'attenta inizializzazione dei parametri. 

In particolare poich� sul \textit{dataset} i dati sono annotati e conosciamo quindi la fase del cammino cui appartengono, � possibile procedere all'addestramento con approccio supervisionato. L'associazione di ogni stato del modello ad
una fase cammino permette di stimare la probabilit� $\pi_i$ dell'i-esimo stato di essere il primo elemento di una sequenza come 
\begin{equation}
\begin{split}
\pi_i & = N_i/N_{tot} \quad i = 1,...,Q\\
\end{split}	 
\label{eq:hmmTrainingPI}
\end{equation} 
ovvero la frazione degli $N_i$ dati relativi alla fase i-esima, rispetto al totale dei dati del \textit{training set} $N_{tot}$.
Analogamente la matrice delle probabilit� di transizione viene calcolata come 
\begin{equation}
\begin{split}
a_{ij}& = C/N_i   \quad  \text{se}\quad j = (i+1)\%Q \quad \text{e} \quad i,j = 1,...,Q \\
a_{ij}& = 1- C/N_i\quad \text{se} \quad j = i\\
a_{ij}& = 0       \quad \text{altrimenti} 
\end{split}	 
\label{eq:hmmTrainingA}
\end{equation}

dove $C$ � il numero di cicli di deambulazione nel \textit{training set}.\\ 

I parametri relativi alle probabilit� di emissione ( $\mu$ e $\sigma$) dei singoli stati sono stimati a partire dai dati a seconda della fase del cammino cui appartengono. Assumendo un modello probabilistico gaussiano a descrivere le emissioni degli stati della HMM, i parametri delle densit� di emissione dell'i-esimo stato sono stimati valutando le medie e le deviazioni standard empiriche delle osservazioni etichettate come appartenenti alla fase $i$ del cammino 
\begin{equation}
\begin{split}
\mu_i &= \dfrac{1}{N_i} \displaystyle \sum_{t=1}^T{\Omega(t)} \quad i = 1,...,Q\\
\sigma_i &= \sqrt{\dfrac{1}{N_i -1}\displaystyle \sum_{t=1}^T({\Omega(t)-\mu_i})^2}
\end{split}	 
\label{eq:hmmTrainingB}
\end{equation}


La procedura di validazione serve a stabilire la capacit� del modello di generalizzare su dati nuovi, ovvero l'accuratezza con cui eseguir� la segmentazione della deambulazione di individui per i quali non � stato addestrato.

La validazione � stata fatta con l'approccio della validazione incrociata ad esclusione di un campione (\textit{leave-one-subject-out cross-validation} o LOOCV) oppure validazione a rotazione. 

Il metodo consiste nell'addestrare un modello su $P-1$ soggetti (in questo caso $P = 6$) e verificarne la validit� su 1 soggetto (quello escluso dall'addestramento). Il risultato della procedura � il modello $Deamb\_HMM_1$.  
La procedura viene ripetuta $P$ volte in modo da addestrare e verificare su tutti i soggetti. Questo produce $P$ modelli 
$Deamb\_HMM_1, ..., Deamb\_HMM_P$. La validazione ha confermato il corretto funzionamento dei modelli.\\
% il modello meno performante, ha classificato l'\TODO{80\%} dei dati, mentre quello pi� performante il \TODO{99,9\%}.\\


Le fasi della creazione addestramento e validazione incrociata del modello � stata fatta in differita, mediante un Mathworks\tm  MATLAB\tm  ed l'\textit{HMM toolbox} \cite{HMM_toolbox}. 
Nella fase di verifica, viene usato l'algoritmo di decodifica di Viterbi (vedi Appendice \ref{alg:viterbi}) sui dati del giroscopio, per trovare la sequenza di stati che con maggior verosimiglianza ha prodotto le osservazioni (sequenze di valori del giroscopio).\\

Il modello sinistra-destra della HMM (vedi Appendice \ref{sec:HMM}) pu� portare a considerare cicli di deambulazione aggiuntivi detti inserzioni (\textit{insertions}), o meno di quelli effettivi delezioni (\textit{deletions}). Il problema delle inserzioni viene risolto mediante l'applicazione di un'euristica: per i presupposti del tipo di deambulazione considerata (velocit� e pendenza del terreno), tutti i cicli di deambulazione di durata inferiore a $250\, ms$ sono inserzioni.


%%%%%%%%%%%%%%%%%%%%%%%%%%%%%%%%%%%%%%%%%%%%%%%%%%%%%%%%%%%%%%%%%%%%%%%%%%%%%%%%%%%%%%%%%%%%%%%%%%%%%%%%%%%%%%%%%%%%%%%%%%%%%%%%%%%%%%%%%%%%%%%%%%%%%%%%%%%%%%%%%%%%%%%%%%%%%%%%%%%%%%%%%%%%%%%%%%%%%%%%%%%%%%%%%%%%%%%%%%%%%%%%%%%%%%%%%%%%%%%%%%
\section{Parametri ottenuti}
Dato che la validazione ha confermato la correttezza del sistema, per la segmentazione � stato addestrato una HMM con tutti i soggetti (in modo da avere un modello ancora pi� performante), ed i parametri ottenuti sono presentati in Tabella \ref{tab:prior}, Tabella \ref{tab:matrice_transizione} e Tabella \ref{tab:matrice_emissione}. In Tabella \ref{tab:osservazioni} � riportato un esempio di dati in ingresso al modello .
\begin{table}%
\centering
\begin{tabular}{|c|c c c c|}
\hline
\textbf{A} & \textsc{FF} & \textsc{HO} & \textsc{TO}&\textsc{HS} \\
%\hline
\hline
\textsc{HS} & $0.985$ & $0.015$ & $0.0$ & $0.0$\\
%\hline
\textsc{FF} & $0.0$ & $0.998$ & $0.002$ & $0.0$\\
%\hline
\textsc{HO} & $0.0$ & $0.0$ & $0.991$ & $0.009$\\
%\hline
\textsc{FO} & $0.007$ & $0.0$ & $0.0$ & $0.993$\\
\hline	 
\end{tabular}
\caption{Matrice di Transizione}
\label{tab:matrice_transizione}
\end{table}


\begin{table}%
\centering
\begin{tabular}{|c|c c|}
\hline
\textbf{B} &\textsc{$\mu(^\circ/s)$} & \textsc{$\sigma ^2 (^\circ/s)$}\\
%\hline
\hline
\textsc{HS} & $-52.2$ & $12711.8$\\
%\hline
\textsc{FF} & $-11.8$ & $898.7$\\
%\hline
\textsc{HO} & $-180.1$ & $35806.8$\\
%\hline
\textsc{FO} & $233$ & $14980.3$\\
\hline	 
\end{tabular}
\caption{Matrice di Emissione}
\label{tab:matrice_emissione}
\end{table}


\begin{table}%
\centering
\begin{tabular}{|c|c|}
\hline
 Stato&\textsc{$\pi$} \\
%\hline
\hline
\textsc{HS} & $0.180$ \\  
%\hline
\textsc{FF} & $0.278$ \\
%\hline
\textsc{HO} & $0.279$ \\
%\hline
\textsc{FO} & $0.264$\\
\hline	 
\end{tabular}
\caption{Distribuzione di probabilit� iniziale}
\label{tab:prior}
\end{table}
Le seguenti sono un esempio di osservazioni: 12400 dati Giroscopio
\begin{table}%
\centering
\begin{tabular}{|c|c|}
\hline
 time&\textsc{$O$} \\
%\hline
\hline
\textsc{1} & $0.37694$\\  
\textsc{2} & $0.37694$\\
\textsc{3} & $0.37694$\\
\textsc{4} & $0.37694$\\
\textsc{5} & $-1.9058$\\
\vdots & \vdots\\
\textsc{24800} & $-16.831$\\
\hline	 
\end{tabular}
\caption{Osservazioni: esempio di segnale giroscopico}
\label{tab:osservazioni}
\end{table}

\chapter{Applicazione Android}
%\myChapter{applicazione Android\tm}
%%%%%%%%%%%%%%%%%%%%%%%%%%%%%%%%%%%%%%%%%%%%%%%%%%%%%%%%%%%%%%%%%%%%%%%%%%%%%%%%%%%%%%%%%%%%%%%%%%%%%%%%%%%%%%%%%%%%%%%%%%%%%%%%%%%%%%%%%%%%%%%%%%%%%%%%%%%%%%%%%%%%%%%%%%%%%%%%%%%%%%%%%%%%%%%%%%%%%%%%%%%%%%%%%%%%%%%%%%%%%%%%%%%%%%%%%%%%%%%%%%
\section{Introduzione}%descrivere cosa fa a grandi linee %%%%%%%%%%%%%%%%%%%%%%%%%                           INTRODUZIONE
La seconda parte del lavoro � relativa allo sviluppo di un'applicazione per uno  \textit{Smartphone}  Android\tm che segmenta in linea il cammino, a partire da dati giroscopici (vedi Figura \ref{fig:schema_app}).
\begin{figure}[h]
	\centering
		\includegraphics[width=1\textwidth]{imgs/ArchitetturaProgramma.jpg}
	\caption{Architettura dell'applicazione per \textit{Smartphone}}
	\label{fig:schema_app}
\end{figure} 
Una volta posizionata la IMU (vedi Appendice \ref{sec:imu}) sul collo del piede, lo \textit{Smartphone} vi interagisce tramite una connessione \textit{Bluetooth}.\\ 
L'utente controlla la IMU mediante l'applicazione, quindi pu� impostarne i parametri: ad esempio la frequenza di campionamento e gli intervalli di campionamento per ciascun sensore su ciascun asse spaziale. Impostati i parametri l'utente pu� acquisire dati dalla IMU mediante un servizio di \textit{data logging} (nel \textit{file system} dello \textit{Smartphone}), oppure li pu� visualizzare come un grafico dinamico. Inoltre � possibile abilitare la segmentazione in linea del segnale di deambulazione, sulla base di un modello HMM addestrato in differita (vedi Sezione \ref{sec:creazione_addestramento_modello}), impiegando una versione dell'algoritmo di Viterbi \cite{short_time_viterbi_online_hmm_deconding}. Nell'ultimo caso il grafico dinamico rappresentante il segnale del giroscopio viene tracciato in linea\footnote{La segmentazione ha una latenza massima di \textbf{10ms}} con un colore a seconda della fase della deambulazione: HS-blu, FF-giallo, HO-rosso, TO-verde. \\

%Sulla base di un modello HMM addestrato in differita (vedi sezione \ref{sec:creazione_addestramento_modello}), impiega una versione dell'algoritmo di Viterbi \cite{short_time_viterbi_online_hmm_deconding} per elaborare a run-time i dati giroscopici acquisiti dalla IMU (e trasferiti via Bluetooth), effettuando la segmentazione e consentendo output di tipo grafico (colorando in modo differente la traccia del segnale giroscopico in funzione della fase del cammino), e di data logging. 

%%%%%%%%%%%%%%%%%%%%%%%%%%%%%%%%%%%%%%%%%%%%%%%%%%%%%%%%%%%%%%%%%%%%%%%%%%%%%%%%%%%%%%%%%%%%%%%%%%%%%%%%%%%%%%%%%%%%%%%%%%%%%%%%%%%%%%%%%%%%%%%%%%%%%%%%%%%%%%%%%%%%%%%%%%%%%%%%%%%%%%%%%%%%%%%%%%%%%%%%%%%%%%%%%%%%%%%%%%%%%%%%%%%%%%%%%%%%%%%%%%
\section{Metodologia di programmazione seguita	: \textit{Agile e Unit Test}}%agile %%%%%%%%%%%%%%%%%         METODOLOGIA DI PROGRAMMAZIONE
\label{sec:agile}
Per l'implementazione dell'algoritmo di Viterbi e le funzioni utilizzate nella fase di analisi del segnale, � stata utilizzata una tecnica nota come \textit{Agile Programming} (AP): una tecnica che permette lo sviluppo di programmi in cicli di lavoro iterativi ed incrementali in termini di funzionalit�. Tale metodologia � stata corroborata da una tecnica di verifica del codice sorgente detta \textit{Unit Test} (UT) verifica unitaria. Per \textit{unit} (unit�) si intende la pi� piccola parte funzionante di un programma che pu� essere verificata, nei linguaggi orientati ad oggetti un metodo costituisce un'unit�. Per UT in Java\tm si intende la verifica del funzionamento di ciascun metodo di una classe. 


Tra i benefici apportati dall'UT si possono annoverare:
\begin{itemize}
	\item Semplificazione della procedura di riscrittura del codice sorgente (\textit{refactoring}): procedura di modifica della struttura di un programma che non ne altera il funzionamento complessivo.
	\item Automatizzazione dell'integrazione di moduli di programma: l'interazione fra moduli deve essere verificata ed in mancanza di UT viene fatta su misura per ogni caso. 
	\item Creazione di documentazione ``concreta'': i vari casi di verifica (\textit{test}) costituiscono degli esempi concreti di utilizzo di ciascun modulo e dell'interazione fra di essi. Un vantaggio che la documentazione fornita mediante UT ha rispetto alla documentazione classica � che ogni modifica apportata al programma si ripercuote sul funzionamento dei casi di verifica, mentre la documentazione classica rischia di diventare obsoleta se tale cambiamento non viene trascritto.
\end{itemize}
La limitazioni maggiori sono: 
\begin{itemize}
	\item Lo UT � orientato alla funzionalit� di singoli moduli di programmi, quindi aspetti come la performance o funzionalit� che necessitano di molti moduli sono difficilmente verificabili mediante UT.
	\item La scelta dei casi di verifica tra l'esorbitante numero di possibile scelte � spesso un problema difficile.
	\item Vi sono casi in cui lo UT non pu� essere applicato, come ad esempio nei programmi non deterministici o in un ambiente \textit{multithread}. 
\end{itemize} 

\begin{description}
	\item [Scrittura di uno UT]. Lo UT istanza l'oggetto dalla classe da verificare, ed invoca il metodo con valori per i quali � noto il comportamento (teorico) del metodo da verificare.\\
Il test viene eseguito, per la prima volta, prima di aver implementato il metodo da verificare, ed ovviamente fallisce\footnote{Eclipse IDE \tm utilizza il \textit{framework} JUnit\tm che semplifica la gestione degli UT.}. Il passo successivo � di implementare il minimo indispensabile per far funzionare il test. Se il test viene superato dal codice scritto, viene scritto un altro test, ed un altro pezzo di codice, iterando la procedura finch� non si ha tutto il programma desiderato.\\ 

A titolo di esempio di seguito verr� illustrato lo sviluppo della classe che gestisce le operazioni di tipo statistico:
\begin{lstlisting}[language=java, style=eclipse, caption=Sviluppo di una classe mediante la tecnica di programmazione Agile, label=code:agile1]
// la classe da costruire 
public class StatisticsOperations{...}
// l'insieme di test
public class StatisticsOperationsTest extends TestCase {...}
\end{lstlisting}


Le operazioni da implementare per le HMM sono: 
\begin{itemize}
	\item verificare la validit� di valori di probabilit�,
	\item verificare la completezza di un insieme di alternative probabilistiche,
	\item calcolare la Probabilit� di un dato assumendo quanto estratto da una distribuzione Normale ($ \mathcal{N}(x,\mu, \sigma)$) con media $\mu$ e varianza $\sigma^2$ note.
\end{itemize}

Creazione delle le firme dei metodi

\begin{lstlisting}[language=java, style=eclipse, caption=Firme dei metodi da implementare, label=code:agile2]
public class StatisticsOperations{	
	public static boolean areCompleteProbabilisticAlternatives(ArrayList<Object> alternatives) {...}
	public static boolean isValidProbabilityValue(double value) {...}
	public static double gaussian(double x, double mu, double sigma_square) {...}
}
\end{lstlisting}

\item Creazione di una verifica vuota (destinato a fallire). L'obbiettivo di questa e della successiva fase � quello di specificare i requisiti della classe da produrre: nel momento in cui vengono create le verifiche, si hanno chiari i funzionamenti dei metodi ed i vincoli che questi devono rispettare.  
\item Implementazione della verifica per cinque casi rappresentativi dei degli intervalli esterni ed interni a quello di riferimento (0,1).

\begin{lstlisting}[language=java, style=eclipse, caption=Casi di test di contorno, label=code:agile3]
public class StatisticsOperationsTest extends TestCase {
	... // metodi setUp e tearDown che per ora non uso
	
	public void testIsMinusOneACorrectProbabilityValue() {
		boolean expected = false;
		boolean actual = StatisticsOperations.isValidProbabilityValue(-1);
		assertEquals(expected, actual);
	}
	public void testIsZeroACorrectProbabilityValue() {
		boolean expected = true;
		boolean actual = StatisticsOperations.isValidProbabilityValue(0);
		assertEquals(expected, actual);
	}
	public void testIsPointFiveACorrectProbabilityValue() {
		boolean expected = true;
		boolean actual = StatisticsOperations.isValidProbabilityValue(.5);
		assertEquals(expected, actual);
	}
	
	public void testIsOneACorrectProbabilityValue() {
		boolean expected = true;
		boolean actual = StatisticsOperations.isValidProbabilityValue(1);
		assertEquals(expected, actual);
	}
	
	public void testIsTwoACorrectProbabilityValue() {
		boolean expected = false;
		boolean actual = StatisticsOperations.isValidProbabilityValue(2);
		assertEquals(expected, actual);
	}
	
	public void testIsRandNumACorrectProbabilityValue() {
		boolean expected = false;
		double randomProbabilityValue = Math.random();
		if (randomProbabilityValue >= 0 && randomProbabilityValue <= 1){
			expected = true;
		}
		boolean actual = StatisticsOperations.isValidProbabilityValue(randomProbabilityValue);
		assertEquals(expected, actual);
	}

}
\end{lstlisting}

\item Sviluppo del corpo del metodo che si sta testando. 
Questa fase � guidata dalle precedenti : il programmatore ha chiare la funzione da implementare e le relative problematiche, perch� presenti nelle verifiche. 

\begin{lstlisting}[language=java, style=eclipse,caption=Implementazione dei metodi da testare, label=code:agile4]
public class StatisticsOperations{	
	public static boolean areCompleteProbabilisticAlternatives(ArrayList<Object> alternatives) {...}
	public static boolean isValidProbabilityValue(double value) {
		if (value >= 0 && value <= 1)
			return true;
		else
			return false;
}
	public static double gaussian(double x, double mu, double sigma_square) {...}
}
\end{lstlisting}
\end{description}

La esecuzione delle verifiche sul metodo riscritto porta ad uno dei seguenti esiti:
\begin{enumerate}
	\item superamento di tutte le verifiche, quindi lo sviluppo del metodo � completato,
	\item fallimento di almeno una verifica, il metodo deve essere corretto e le verifiche devono essere nuovamente eseguite.
\end{enumerate}

%%%%%%%%%%%%%%%%%%%%%%%%%%%%%%%%%%%%%%%%%%%%%%%%%%%%%%%%%%%%%%%%%%%%%%%%%%%%%%%%%%%%%%%%%%%%%%%%%%%%%%%%%%%%%%%%%%%%%%%%%%%%%%%%%%%%%%%%%%%%%%%%%%%%%%%%%%%%%%%%%%%%%%%%%%%%%%%%%%%%%%%%%%%%%%%%%%%%%%%%%%%%%%%%%%%%%%%%%%%%%%%%%%%%%%%%%%%%%%%%%%
\section{Strumenti e ambiente di lavoro}%%%%%%%%%%%%%%%%%%%%%%%%%%%%%%%%%%%%%%%%%%%%%%%              ORGANIZZAZIONE DEL CODICE
\subsection{Android \textit{Manifest}} AndroidManifest.xml (vedi Appendice \ref{sec:android_app_structure}) � il file che viene usato come indice dal compilatore Dalvik (vedi Appendice \ref{sec:dalvik}) per conoscere l'ordine in cui deve compilare i file del programma. 


\begin{lstlisting}[language=XML, style=xmlandroid, caption=AndroidManifest testata, label=code:manifest_permissions1]
<?xml version="1.0" encoding="utf-8"?>
<manifest xmlns:android="http://schemas.android.com/apk/res/android"
	package="imu.Interface" android:versCode="1" android:versName="1.0">
\end{lstlisting}
Il nome del Java\tm \textit{package}, serve come identificativo univoco per l'applicazione (app-id).

\begin{lstlisting}[language=XML, style=xmlandroid,caption=AndroidManifest permessi,label=code:manifest_permissions2, firstnumber=3]
	<uses-permission android:name="android.permission.BLUETOOTH" />
	<uses-permission android:name="android.permission.BLUETOOTH_ADMIN" />
	<uses-permission android:name="android.permission.WRITE_EXTERNAL_STORAGE" />
	<uses-sdk android:minSdkVersion="8" />
	...
\end{lstlisting}

%%%%%%%%%%%%%%%%%%%%%%%%%%%%%%%%%%%%%%%%%%%%%%%%%%%%%%%%%%%%%%%%%%%%%%%%%%%%%%%%%%%%%%%
%%%%%%%%%%%%%%%%%%%%%%%%%%%%%%%%%%%%% modifica 24 nov %%%%%%%%%%%%%%%%%%%%%%%%%%%%%%%%%
%%%%%%%%%%%%%%%%%%%%%%%%%%%%%%%%%%%%%%%%%%%%%%%%%%%%%%%%%%%%%%%%%%%%%%%%%%%%%%%%%%%%%%%
Dato che il nucleo (\textit{kernel}) di Android-OS � basato su Linux (vedi Appendice \ref{app:android}) usa una politica di controllo di accessi (\textit{Access Control})\footnote{Accesso controllato dell'utente di un \textit{filesystem} o di un servizio} molto simile a quello del sistema operativo Linux. Tutti i servizi che non sono forniti da un programma diverso da quello in uso possono essere utilizzati solo se l'utente d� il permesso di utilizzarli, vale a dire che un programma non pu� interagire con altri programmi se non con l'esplicita autorizzazione dell'utente. Per questo motivo l'applicazione deve elencare tutte le richieste di permessi di cui necessita. L'elenco deve essere fatto nella porzione iniziale dell'\verb|AndroidManifest|.\\
 Dichiarazione dei permessi che l'applicazione deve avere dall'utente per accedere ad alcune parti dell'API\footnote{ \textit{Application Programming Interface}: interfaccia ad un codice che viene messo a disposizione come servizio in modo che terzi possano usarlo per sviluppare altro software, avvolte anche in altri linguaggi.
 In dettaglio una API nei linguaggi orientati ad oggetti, sono concepiti come l'insieme di tutti i metodi pubblici che le classi pubbliche offrono.} (\textit{Application Programming Interface}) di Android\tm e per interagire con altri applicazioni.
 La riga 4 chiede il permesso di fare la ricerca (\textit{discovery}) di dispositivi e di associarsi  a dispositivi trovati, la riga precedente invece il permesso di connettersi a dispositivi associati \footnote{L'associazione � la prima connessione, nella quale vi � una forma di identificazione e associazione dei due dispositivi. Per connessione invece si intendono le connessioni successive alla prima, in cui un dispositivo conosce gi� l'altro e, pi� rapidamente, possono iniziare una conversazione.}. L'ultimo permesso serve per poter scrivere su un dispositivo di archiviazione (\textit{storage}) esterno.

 
\begin{lstlisting}[language=XML, style=xmlandroid,caption=AndroidManifest dichiarazione dell'attivit� principale, label=code:manifest_application, firstnumber=8]
	<application android:icon="@drawable/spinningtop02"
		android:label="@string/app_nameicon" 
		android:name="imu.objects.MailBoxes">
\end{lstlisting}

Vengono dichiarati il nome dell'applicazione, l'icona, ed un'etichetta. In questo punto si dichiara anche il nome dell'oggetto che pu� essere visibile a livello globale di applicazione. Di fatto l'oggetto \verb|MailBoxes| che � un vettore di \verb|MailBox| (vedi Sezione \ref{sec:mailbox}) che sono a loro volta utilizzati per lo scambio di dati tra \textit{thread}.

\begin{lstlisting}[language=XML, style=xmlandroid,caption=AndroidManifest elenco di tutte le attivit� ,label=code:manifest_activities, firstnumber=11]		
		<activity android:name=".IMUinterface" android:configChanges="orientation">
			<intent-filter>
				<action android:name="android.intent.action.MAIN" />
				<category android:name="android.intent.category.LAUNCHER" />
			</intent-filter>
		</activity>
		<activity android:name=".Saveactivity" android:label="@string/save_name" />
		<activity android:name=".SensSetupactivity" android:label="@string/sens_name" />
		<activity android:name=".ProvaActivity" android:label="@string/prova_name" />
		<activity android:name=".EmbeddedSensDataPlottingvActivity"
			android:label="@string/prova_name" android:configChanges="orientation"
			android:theme="@android:style/Theme.NoTitleBar.Fullscreen" />
		<activity android:name=".TemporalDataPlottingvActivity"
			android:configChanges="orientation" android:theme="@android:style/Theme.NoTitleBar.Fullscreen" />
\end{lstlisting}	

Sezione di \verb|AndroidManifest| in cui si elencano i componenti fondamentali (vedi Appendice \ref{sec:android_main_components}) utilizzati nell'applicazione.\\

%L' applicazione che ha una struttura molto semplice, ed utilizza solo \verb|Activity| ed \verb|Intent|. 
La \textit{Activity} principale (\textit{Main Activity}) � \verb|IMUInterface|.\\ 
La dichiarazione di una \textit{Activity} � composta da un nome e da una classe Java\tm.
Inoltre sono dichiarate altre impostazioni iniziali che riguardano soprattutto l'interfaccia utente (UI), ad esempio la gestione degli orientamenti dello schermo.

\subsection{Codice sorgente}
Tutto il codice sorgente (Java\tm) deve essere contenuto nella cartella \emph{src} del progetto. 
Il codice sorgente � stato organizzato 5 \textit{package} (cartelle di Java\tm):
\begin{itemize}
	\item \verb|activities|: contenente le \textit{Activity} che corrispondono indicativamente ciascuna ad una schermata sullo della UI,
	\item \verb|objects| : dati, modelli o contenitori su cui compiere operazioni,
	\item \verb|services|: operazioni fondamentali dell'applicazione, che in questo caso coincidono con gli algoritmi che si applicano alle HMM. 
	\item \verb|tests|: verifiche (\textit{test}) prodotte dall'utilizzo della metodologia Agile (vedi Sezione \ref{sec:agile}),
	\item \verb|util|: operazioni di supporto al resto del codice sorgente. 
\end{itemize} 

\subsection{Risorse}
Le risorse sono costituite da tutti i \textit{file} che fanno parte dell'applicazione, ma non sono codice sorgente Java\tm (vedi Appendice \ref{sec:android_app_structure}) vale a dire \textit{file} XML, immagini, testo ecc. Una sezione significativa delle risorse � la sezione \textbf{layout} (impaginazione) che rappresenta la struttura della UI. Questo contiene una serie di file XML, preferibilmente, uno per ciascuna \textit{Activity} ed avente lo stesso nome.

\subsubsection{Impaginazione (\textit{Layout})}
Un file di \textit{layout} � un documento XML in cui viene descritta la struttura dell'interfaccia utente di una \textit{Activity}.
Un \textit{layout} deve essere composto di elementi grafici predefiniti in Android\tm oppure deve essere di tipo \verb|View| o  \verb|ViewGroup|\footnote{Sia la \textit{View} che la \textit{ViewGroup} appartengono al \textit{package} \textit{android.view} e sono componenti importante dell'interfaccia utente}. Ogni file di \textit{layout} deve contenere un elemento radice, ovvero un elemento grafico che contene tutti gli altri elementi grafici. Ogni elemento grafico ha un identificativo unico, con cui pu� essere richiamato dal codice sorgente, tramite l'indicizzazione automatica.
Eclipse\tm permette di definire un \textit{layout} sia in modalit� XML che in modalit� grafica (WYSIWYG\footnote{\textit{What You See Is What You Get}}). Ad esempio il \textit{layout} della \textit{Activity} principale (\textit{ImuInterface}) � il seguente 

\begin{lstlisting}[language=XML, style=xmlandroid,caption=LinearLayout:impaginazione  della schermata principale,label=code:mainActivity_layout_linear]
<?xml version="1.0" encoding="utf-8"?>
<LinearLayout xmlns:android="http://schemas.android.com/apk/res/android"
	android:orientation="vertical" android:layout_width="fill_parent"
	android:background="#111111" android:layout_height="fill_parent">
\end{lstlisting}

Un \verb|LinearLayout| � un tipo di \textit{layout} che pu� contenere altri \textit{layout} che serve ad organizzare oggetti. Sono impostate per \textit{default} alcune delle propriet� (orientamento, larghezza ecc) dell'oggetto.  

\begin{lstlisting}[language=XML, style=xmlandroid,caption=TextView: oggetto contenente testo ,label=code:mainActivity_layout_textView, firstnumber=5]
	<TextView android:layout_height="wrap_content" 
		android:text="@string/ImuActivity_title" android:textStyle="bold"
		android:id="@+id/AppName" android:layout_width="320dip">
	</TextView>
\end{lstlisting}	

Questo � un oggetto che contiene testo non modificabile dall'utente. Qui viene visualizzato il titolo della schermata, contenuto nella variabile di stringa \verb|"@string/ImuActivity_title"|.

\begin{lstlisting}[language=XML, style=xmlandroid,caption=EditText:oggetto contenente campi di testo modificabili,label=code:mainActivity_editText, firstnumber=9]
	<EditText android:id="@+id/editText1" 
		android:layout_height="wrap_content" 
		android:layout_width="match_parent">
	</EditText>
	<EditText android:id="@+id/editText2"
		android:layout_height="wrap_content" 
		android:layout_width="match_parent">
	</EditText>
	....
\end{lstlisting}	

Vi sono il titolo della schermata, quattro campi di testo (di cui sono riportate solo 2 a titolo illustrativo). 


\begin{lstlisting}[language=XML, style=xmlandroid,caption=Esempio di LinearLayout contenente un Button,label=code:mainActivity_button, firstnumber=25]
	<LinearLayout android:background="#111111"
		android:layout_height="fill_parent" 
		android:layout_weight="1"
		android:layout_width="fill_parent" 
		android:orientation="horizontal">
		...
		<Button android:id="@+id/button2" android:layout_height="wrap_content"
			android:text="Start" android:gravity="center_vertical|center_horizontal"
			android:layout_width="55dip">
		</Button>
		...
	</LinearLayout>
	...
\end{lstlisting}

La parte seguente della schermata, a partire dall'alto, � un altro \verb|LinearLayout| contenente campi di testo e pulsanti.\\
La schermata risultante dal \textit{layout} � quella dell'immagine \ref{fig:main_activity}.
\begin{figure}
	\centering
		\includegraphics[width=.5\textwidth]{imgs/imuControloPanel.jpg}
	\caption{Schermata della Activity iniziale dell'applicazione}
	\label{fig:main_activity}
\end{figure}

\subsubsection{File grafici}
I file grafici sono distribuiti su tre cartelle \verb|drawable-hdpi|, \verb|drawable-ldpi|,\verb|drawable-mdp| in cui salvare formati a definizioni diverse delle stesse immagini per diversi dispositivi. 

\subsubsection{Valori}
Il file pi� importante in questa cartella � \verb|strings.xml| in cui vengono salvate tutti i contenuti testuali della UI. L'utilit� maggiore di mantenere le stringhe in un file unico, � l'internazionalizzazione\footnote{Traduzione in almeno due lingue di tutti i contenuti testuali di un'applicazione.} dell'applicazione: viene utilizzato il principio della separazione dei contenuti dalla forma, in modo che uno possa essere modificato indipendentemente dall'altro.

\subsubsection{Altri File}
Qualunque tipo di file, che non sia compatibile con quelli menzionati precedentemente, viene disposto nella cartella \verb|raw|. Ad esempio in fase di sviluppo, per testare l'algoritmo di Viterbi, � stato usato un file contenente i dati di un giroscopio, tale file poteva essere disposto solo nella cartella \verb|raw|. 

\subsection{Librerie}
Le librerie usate sono quelle di Android\tm2.2 (vedi Appendice \ref{sec:lib_Android}).
%%%%%%%%%%%%%%%%%%%%%%%%%%%%%%%%%%%%%%%%%%%%%%%%%%%%%%%%%%%%%%%%%%%%%%%%%%%%%%%%%%%%%%%%%%%%%%%%%%%%%%%%%%%%%%%%%%%%%%%%%%%%%%%%%%%%%%%%%%%%%%%%%%%%%%%%%%%%%%%%%%%%%%%%%%%%%%%%%%%%%%%%%%%%%%%%%%%%%%%%%%%%%%%%%%%%%%%%%%%%%%%%%%%%%%%%%%%%%%%%%%
\section{Architettura}%%%%%%%%%%%%%%%%%%%%%%%%%%%%%%%%%%%%%%%%%%%%%%%%%%%%%%%%%%%                         FUNZIONAMENTO 
\subsection{Unit� di Controllo}

\subsection{Interfaccia Utente}
La linea guida per lo sviluppo della UI � stata quella della massima ergonomia, in linea con la filosofia di sviluppo di interfacce utente di Android\tm. \\
La navigazione tra le schermate dell'applicazione � guidata da \textit{widget}: oggetti grafici della UI di un programma, che hanno lo scopo di facilitare all'utente l'interazione con il programma. \\
All'avvio del programma, viene visualizzato il pannello di controllo della IMU (vedi Figura \ref{fig:app_UI}), da qui vengono impostati e monitorati i valori dei parametri di acquisizione dei dati. 

\begin{figure}[h]
	\centering
		\includegraphics[width=1\textwidth]{imgs/UI.jpg}
	\caption{Interfaccia utente dell'applicazione: da sinistra, Pannello di controllo, Impostazione dei parametri dei sensori, Configurazione delle modalit� di salvataggio.}
	\label{fig:app_UI}
\end{figure}
 
La creazione di un \textit{layout} pu� essere fatta in due modi: in XML oppure a tempo d'esecuzione (\textit{runtime}) in Java\tm utilizzando oggetti di tipo \verb|View| e \verb|ViewGroup|. Il primo metodo � il pi� usato, per una migliore separazione fra l'aspetto grafico dell'applicazione ed il codice sorgente che ne gestisce il comportamento, sia per la possibilit� di usare strumenti grafici di creazione delle schermate, ad esempio Eclipse\tm.\\
Ad ogni oggetto grafico dell'interfaccia utente � associato un insieme di eventi possibili. Ad esempio alcuni degli eventi associati all'oggetto \verb|Button| (pulsante) sono: 
\begin{lstlisting}[language=java, style=eclipse,label=code:button_events]
	onKeyDown(int keyCode, KeyEvent kevent)
	onKeyUp()
	onTouchEvent(int keyCode, KeyEvent kevent)
	setOnClickListner(OnClickListner l)
\end{lstlisting}

Gestire il comportamento dell'oggetto \verb|Button| corrisponde a scegliere gli eventi ai quali l'oggetto deve essere suscettibile e l'azione che questi deve compiere implementando i corpi dei metodi scelti. 
Come esempio in seguito verr� riportata la creazione del \verb|Button| ``\textit{Stop}'' della del Pannello di Controllo.
L'oggetto viene creato in XML nel seguente modo: 

\begin{lstlisting}[language=XML, style=xmlandroid, caption=Android Button Form Widget, label=code:button_widget]
	<Button android:id="@+id/button2" 
					android:text="Start" 
					android:layout_height="wrap_content"
					android:layout_width="80dip"/>
					android:gravity="center_vertical|center_horizontal"
\end{lstlisting}

La prima riga specifica il tipo di oggetto (\verb|Button|) ed il suo identificativo unico \textit{button2}.
La seconda la stringa che appare sull'oggetto. Dalla terza alla quinta, la relazione che il pulsante deve avere con il resto della schermata.

Una volta creato l'oggetto XML, il codice sorgente Java\tm che ne gestisce il comportamento � il seguente:
\begin{lstlisting}[language=java, style=eclipse,label=code:button_creation]
	private Button startButton = (Button) findViewById(R.id.button2);
\end{lstlisting}

Il pulsante di nome \textit{startButton} viene creato, grazie al metodo 
\begin{verbatim}
findViewById(int id)
\end{verbatim}

con l'identificativo dato all'oggetto alla creazione in XML. Il metodo \verb|findViewById| funziona perch� ogni ogni elemento nei file di \textit{layout} in XML vengono compilati in una risorsa di tipo \verb|View| con l'identificativo scelto nella creazione. A tempo di esecuzione, tali oggetti vengono caricati e visualizzati. 
Il pulsante \textit{Start} avvia l'acquisizione di dati dalla IMU. Se la connessione \textit{Bluetooth} non � funzionante, la pressione del pulsante segnala un errore.
\begin{lstlisting}[language=java, style=eclipse,label=code:button_useage]
	// verifica l'esistenza della connessione
	if (mBluetoothAdapter == null) {
	...
	//metodo per stare in ascolto sull evento "`click"' del pulsante
	startButton.setOnClickListener(new View.OnClickListener() {
			public void onClick(View v) {
				//messaggio di impossibilit� di lettura dei dati
				Toast.makeText(imuContext, R.string.can_t_start_reading_data, Toast.LENGTH_LONG).show();
			}
	});
	...
}
\end{lstlisting}
 Se invece la connessione esiste viene iniziata la procedura di acquisizione dei dati della IMU.


\subsection{Comunicazione Bluetooth}
La piattaforma di Android\tm fornisce il supporto per lo stack di rete \textit{Bluetooth}, che permette a due dispositivi muniti di comunicare senza filo. Il framework delle applicazioni (Application Framework, vedi Figura \ref{fig:android_system_architecture}) permette fornisce un accesso alle funzionalit� del \textit{Bluetooth} mediante le Android \textit{Bluetooth} API \footnote{Application Programming Interface, ovvero Interfaccia di Programmazione di un Applicazione cio� l'insieme di tutte le funzionalit� o servizi di un programma resi disponibili a programmatori terzi per l'utilizzo di un codice sorgente.}. 

Le funzionalit� principali che la API del \textit{Bluetooth} sono: 
\begin{itemize}
	\item Ricerca (discovery) di dispositivi \textit{Bluetooth}, 
	\item Richiesta al adattatore \textit{Bluetooth} (\textit{Bluetooth Adapter}) della lista di dispositivi abbinati (paired)\footnote{Due dispositivi sono detti paired quando questi sono l'uno a conoscenza dell'esistenza dell'altro.},
	\item istanziazione di un dispositivo \textit{Bluetooth} (\verb|BluetoothDevice|) usando un indirizzo MAC\footnote{Media Access Control, indirizzo fisico univoco di 6 byte} noto,
	\item creazione di una \textit{server-socket} (\verb|BluetoothServerSocket|) per accettare richieste di connessione da altri dispositivi \textit{Bluetooth}.
	\item trasferire dati da ed ad altri dispositivi,
	\item gestire molteplici connessioni.
\end{itemize}

Alcune delle classi della \textit{Bluetooth} API che devono essere implementate sono:
\begin{itemize}
	\item \verb|BluetoothAdapter|: l'adattatore (radio) \textit{Bluetooth} locale. Rappresenta il punto di entrata per ogni interazione mediante \textit{Bluetooth};
	\item \verb|BluetoothDevice|: dispositivo \textit{Bluetooth} remoto;
	\item \verb|BluetoothSocket|: punto di connessione che permette lo scambio di dati mediante canali di flussi di dati;
	\item \verb|BluetoothServerSocket|: una \textit{socket} aperta di un server che attende richieste di connessione da dispositivi bluetooth remoti. Se la richiesta viene accettata la server \textit{socket} crea una \verb|BluetoothSocket| connessa.
\end{itemize}

Per poter utilizzare il Bluetooth, vi sono richieste 4 attivit� fondamentali da portare a termine mediante le Bluetooth API:
\begin{enumerate}
	\item \textbf{Impostare il Bluetooth}: assumendo che il dispositivo supporti il bluetooth e che sia abilitato, l'impostazione del \textit{Bluetooth} viene fatta in due passaggi:
	\begin{enumerate}
	\item {Ottenere l'adattatore del proprio dispositivo}: 
	\begin{lstlisting}[language=java, style=eclipse,label=code:bluetooth_adapter]
		BluetoothAdapter bluetoothAdapter = BluetoothAdapter.getDefaultAdapter();
		//il metodo getDefaultAdapter() fornisce l'accesso alla radio bluetooth del dispositivo
			if (bluetoothAdapter == null) {
    	// Il dispositivo non supporta il Bluetooth
		}
	\end{lstlisting}
	\item {Abilitare l'adattatore ottenuto}: affinch� si possa comunicare mediante il \textit{Bluetooth}, l'adattatore deve essere attivo. Se non lo � la prassi � di richiedere all'utente il permesso di poterlo abilitare.
		\begin{lstlisting}[language=java, style=eclipse,label=code:enable_bluetooth]
		if (!mBluetoothAdapter.isEnabled()) {
		//l'adattatore bluetooth � disabilitato
	    Intent enableBtIntent = new Intent(BluetoothAdapter.ACTION_REQUEST_ENABLE);
	    // crea l'intenzione di abilitare l'adattatore
	    startActivityForResult(enableBtIntent, REQUEST_ENABLE_BT);
	    //richiede all'utente, mediante una finestra di comunicazione di sistema (dialog), il permesso di abilitare l'adattatore
			}
		}
	\end{lstlisting}
\end{enumerate}
	\item \textbf{Ricercare dispositivi}: una volta ottenuto l'adattatore, si possono trovare dispositivi \textit{Bluetooth} remoti in due modi: mediante una procedura nota come scoperta di dispositivi (\textit{device discovery}) oppure interrogando una lista di dispositivi gi� trovati e abbinati (\textit{paired}):
	\begin{description}
	\item[Scoperta di dispositivi]: procedura di scansione dell'area locale alla ricerca di dispositivi con \textit{Bluetooth} abilitato. Nel momento in cui un dispositivo viene trovato, risponde con informazioni che lo possano identificare (nome, classe ed indirizzo MAC). Mediante questa'informazione il dispositivo che ha fatto la ricerca pu� decidere di iniziare la procedura di connessione ai dispositivi scoperti. Questa procedura pu� essere fatta manualmente precedentemente all'avvio del applicazione. 
	\item [Interrogazione della lista dei dispositivi abbinati]:
	\begin{lstlisting}[language=java, style=eclipse,label=code:query_paired_devices]
		Set<BluetoothDevice> pairedDevices = bluetoothAdapter.getBondedDevices();
		// Se vi sono dispositivi abbinati
		if (pairedDevices.size() > 0) {
		    // scorrere la lista
		    for (BluetoothDevice device : pairedDevices) {
		        // da ciascun dispositivo si possono ottenere informazioni come
		        device.getName();
		        device.getAddress());
		    }
		}
	\end{lstlisting}	
\end{description}
	\item \textbf{Connettere dispositivi}: due dispositivi appaiati possono stabilire una connessione criptata. Possono cos� comunicare mediante un canale di comunicazione RFCOMM\footnote{Comunicazione su frequenza radio (\textit{Radio Frequency Communication}) � un protocollo di trasporto di dati che emula una porta seriale.}. Affinch� la connessione avvenga deve essere implementato un meccanismo \textit{client-server}. Assumendo che vi sia un server in ascolto, la procedura di connessione da parte di un dispositivo \textit{client} consiste ha due passaggi:
	\begin{enumerate}
	\item Usare il dispositivo \textit{Bluetooth} remoto \verb|BluetoothDevice| per ottenere una \textit{socket} \verb|BluetoothSocket| con l'uso del metodo 
	\begin{verbatim}
		createRfcommSocketToServiceRecord(UUID)
	\end{verbatim}
	Tale metodo crea la \textit{socket} che si connetter� al dispositivo remoto. L'UUID (\textit{universally unique identifier}) o identificatore universalmente unico � una stringa a 128-bit che viene creato nella fase iniziale di scoperta. 
	\item Inizio della connessione mediante il metodo \verb|connect()|. La chiamata del metodo scatena una SDP (\textit{Service Discovery Protocol}) protocollo di scoperta dei servizio, ovvero una richiesta al dispositivo remoto di verifica la validit� dell'UUID. Se la verifica ha esito positivo, il dispositivo remoto accetta la richiesta e condivide un canale RFCOMM. Dato che il metodo \verb|connect()| � bloccante, la procedura di connessione deve sempre fatta da un \textit{thread} dedicato. 
\end{enumerate}
	\begin{lstlisting}[language=java, style=eclipse,label=code:client_connect]
	private class ConnectThread extends Thread {
    private final BluetoothSocket mmSocket;
    private final BluetoothDevice mmDevice;
 
    public ConnectThread(BluetoothDevice device) {
        BluetoothSocket tmp = null;
        mmDevice = device;
 
        // Ottenimento di un BluetoothSocket da connettere ad un BluetoothDevice
        try {
            // MY_UUID is the app's UUID string, also used by the server code
            tmp = device.createRfcommSocketToServiceRecord(MY_UUID);
        } catch (IOException e) { }
        mmSocket = tmp;
    }
 
    public void run() {
        // se � ancora in atto la fase di scoperta di altri dispositivi, questa rallenter� la connessione, e raggiunto un tempo limite sul ritardo il Android OS terminer� la procedura con un errore. 
        mBluetoothAdapter.cancelDiscovery();
 
        try {
            //Connessione del dispositivo. La chiamata sar� bloccata finch� la connessione verr� effettuata o fallir� ed in quel caso verr� segnalato un errore
            mmSocket.connect();
        } catch (IOException connectException) {
            // Impossibile connettersi; 
            // Chiudere il \textit{socket} prima di uscire dal metodo
            try {
                mmSocket.close();
            } catch (IOException closeException) { }
            return;
        }
 
        // Ottenuta la connessione, affidare ad un altro \textit{thread} la gestione della stessa
        manageConnectedSocket(mmSocket);
    }

}
\end{lstlisting}
\begin{verbatim}
BluetoothDevice
\end{verbatim}
	\item \textbf{Trasferire dati tra dispositivi}: Una volta stabilita una connessione tra due dispositivi, ciascuno avr� un \verb|BluetoothSocket| connesso ed la procedura di comunicazione consiste di due fasi:
	\begin{enumerate}
	\item Ottenere i canali I/O (\textit{Input/Output}) di comunicazione via \textit{socket}:
	\begin{verbatim}
	 InputStream
	 OutputStream
	\end{verbatim}
	\item leggere e scrivere vettori di byte sui canali ottenuti. Per tali operazioni � necessario usare dei processi dedicati, perch� sono operazioni bloccanti:
	\begin{itemize}
	\item \verb|read(byte[])| si blocca finch� c'� qualcosa da leggere nel canale.
	\item \verb|write(byte[])| si blocca se il dispositivo remoto non sta leggendo (chiamando la \textit{read}) abbastanza rapidamente ed i buffer di comunicazione intermedi sono pieni.
\end{itemize}
\end{enumerate}
\end{enumerate}

\begin{lstlisting}[language=java, style=eclipse,label=code:connect_handling]
private class ConnectedThread extends Thread {
		...
    private final InputStream mmInStream;
    private final OutputStream mmOutStream;
 
    public ConnectedThread(BluetoothSocket socket) {
				...
        
        try {
            mmInStream = socket.getInputStream();
            mmOutStream = socket.getOutputStream();
        } catch (IOException e) { 
        	...
        }
    }
 
    public void run() {
        byte[] buffer = new byte[1024];  // buffer per il canale
        int bytes; // byte ottenuti dalla read()
 
        // Lettura dell'InputStream
        while (true) {
            try {
                bytes = mmInStream.read(buffer);
                // Invio dei byte letti alla UI Activity
                mHandler.obtainMessage(MESSAGE_READ, bytes, -1, buffer)
                        .sendToTarget();
            } catch (IOException e) {
                break;
            }
        }
    }
 
    /* Chiamare il metodo dalla Activity principale per inviare dati ad un dispositivo remoto*/
    public void write(byte[] bytes) {
        try {
            mmOutStream.write(bytes);
        } catch (IOException e) { }
    }
}
\end{lstlisting}



\subsection{Gestione di Processi}
Il sistema operativo Android\tm � \textit{multithread}. Un applicazione che risponda rapidamente alle richieste dell'utente deve necessariamente smistare i compiti a pi� \textit{thread}, dando la precedenza ai \textit{thread} di risposta all'utente. Il pi� importante di questi � lo UI \textit{Thread} (User Interface Thread). Una regola di base della programmazione di applicazioni Android\tm �, nella creazione di un'\textit{Activity}, non si deve sovraccaricare, o ancora peggio, eseguire istruzioni potenzialmente bloccanti all'interno dello UI \textit{Thread}. La prassi per la gestione di operazioni con un comportamento imprevedibile dal punto di vista temporale, come ad esempio le operazioni di rete o di lettura e scrittura di file, � di incaricare altri \textit{thread} che possano, se necessario, essere eseguiti in background, o fermati se necessario.\\
Per la comunicazione fra processi � stato ritenuto utile avere un protocollo di comunicazione fra \textit{thread} mediante una struttura nominata "\textit{mailbox}", in cui $n$ \textit{thread} possano inserire e consumare singoli dati.\\
\subsubsection{MailBox}
\label{sec:mailbox}
L'oggetto \verb|MailBox| � un \textit{buffer} di tipo generico a singola posizione, a cui si pu� accedere mediante un inserimento di tre modalit�: sincronizzata, temporizzata e condizionata. 
\begin{enumerate}
	\item \textbf{Sincronizzata}: l'inserimento e prelievo di dati dalla \textit{mailbox} � potenzialmente bloccante, un \textit{thread} inserisce un dato se c'� spazio nel buffer, un altro \textit{thread} utilizza il dato, se esiste e lo rimuove dal buffer. Ciascuno attende il proprio turno per compiere l'azione. La sincronizzazione viene gestita con un meccanismo che Java\tm fornisce mediante la parola chiave \textit{synchronized}. Nel linguaggio Java\tm un metodo � detto sincronizzato se nella sua firma viene usata la parola chiave \textit{synchronized}.
Le propriet� di un metodo sincronizzato sono 2:
\begin{enumerate}
	\item Due metodi dello stesso oggetto avente il metodo non possono essere eseguiti contemporaneamente (\textit{interleave}). Nel momento in cui un metodo di un oggetto � sincronizzato, il primo \textit{thread} che invoca tale metodo possiede in mutua esclusione il \textit{monitor} dell'oggetto. Ci� significa che altri \textit{thread} che invochino qualunque altro metodo sincronizzato dell'oggetto vengono bloccati fino a che il primo \textit{thread}, non rilascia la risorsa. 
	\item I valori contenuti nell'oggetto in questione sono visibili a tutti i \textit{thread}. Ci� avviene perch� , tali valori vengono modificati da un solo \textit{thread} alla volta,
\end{enumerate}
	
	
	
	
	I metodi di accesso sono:

\begin{lstlisting}[language=java, style=eclipse,label=code:mailbox_put_get]
	synchronized public void put(T object)
	synchronized public T get()
\end{lstlisting}


	Un \textit{thread} alla volta pu� prendere, in mutua esclusione, l'accesso in scrittura al buffer, chiamando per primo il metodo \verb|put| (vedi Algoritmo \ref{alg:mailbox_put}). Il valore contenuto nel buffer � sempre visibile a tutti i thread. Questo risultato � stato ottenuto dichiarando la variabile con la parola chiave \verb|volatile|. 
\begin{lstlisting}[language=java, style=eclipse]
	volatile private T buffer = null;
\end{lstlisting}
Normalmente, ci� che avviene al momento in cui un \textit{thread} prende l'accesso in mutua esclusione ad una variabile, � che si copia nel proprio \textit{stack} di lavoro la variabile e modifica quella e solo prima di rilasciare la variabile la aggiorna all'esterno. Altri \textit{thread} che fossero abilitati a vedere la variabile nel momento in cui � in modifica, vedrebbero un valore incongruente con il valore reale della stessa. La parola chiave \textit{volatile}, impedisce ai \textit{thread} che la modificano di crearsene una copia, e li vincola a lavorare direttamente sulla variabile\footnote{la parola chiave \textit{volatile} impedisce un possibile ripristino del valore di una variabile, il che rende potenzialmente pericolosa per la perdita di dati} garantendone la visibilit� in tutti i momenti. Dato che i metodi put e get sono sincronizzati (\textit{synchronized}), il problema della copia in locale del valore di una variabile da parte di un \textit{thread} non persiste, perch� ogni thread ha accesso esclusivo all'oggetto. L'utilit� di usare la parola chiave \textit{volatile} � semplicemente quella di impedire l'operazione di copia della variabile in locale da parte dei \textit{thread}, che � superflua.\\ 

\begin{lstlisting}[language=java, style=eclipse,label=code:mailbox_put]
synchronized public void put(T object){
	//se il buffer � vuoto
	if (buffer == null){
		//inserisci l'oggetto
		buffer = object;
		//notifica tutti i \textit{thread} in attesa sul buffer
	}else{
		try{
			//altrimenti attendi una notifica sul buffer
			this.wait();
			buffer = object;				
		}catch (InterruptedException e) {
			e.printStackTrace();
		}
	}
	this.notifyAll();
}
\end{lstlisting}
 
Un \textit{thread} pu� prelevare il dato immesso da un altro nella mailbox, usando il metodo \verb|get()|(vedi Algoritmo \ref{alg:mailbox_get}). Mentre il dato viene prelevato, questo non pu� essere modificato. Appena il \textit{thread} ha preso il dato, lo elimina dal buffer.

\begin{lstlisting}[language=java, style=eclipse,label=code:mailbox_get]
synchronized public T get(){
		T temp = null;
		//il buffer � vuoto
		if(buffer == null){
			try {
				//attendi una notifica sul buffer
				this.wait();
				//salva il contenuto del buffer
				temp = buffer;
				//svuoto il buffer
				buffer = null;
			} catch (InterruptedException e) {
				e.printStackTrace();
			}
		} else {
			temp = buffer;
			buffer = null;			
			//notifica tutti i thread in attesa sul buffer
		}
		this.notifyAll();
		return  temp; 
	}
\end{lstlisting}

Queste due operazioni implicano che devono essere eseguite in ordine, ed alternati ($put(dato_i)$, $get()$, $put(dato_j)$, $get()$ ecc ).

\item \textbf{Temporizzata}: l'inserimento ed la rimozione del contenuto del buffer sono bloccanti ma solo per un periodo di tempo \verb|t|. Una chiamata \verb|get(t)| in attesa viene sbloccata dopo tempo \verb|t| e se il buffer � vuoto restituisce null e termina. Una chiamata \verb|put(t,o)|, in attesa viene sbloccata dopo tempo \verb|t| e se il buffer non � vuoto, ne sovrascrive il contenuto.

\item \textbf{Condizionata}: questa versione � la generalizzazione della versione temporizzata. L'inserimento e prelievo di dati dalla \textit{mailbox} � bloccante finch� non si verifica una condizione, dopodich� una \verb|get(cond)| in attesa viene sbloccata se l'operazione restituisce \verb|null|; una \verb|put(cond,o)| in attesa viene sbloccata e sovrascrive il valore presente nel buffer.
\end{enumerate}

Soffermandosi sulla prima forma di scambio di messaggi sulla mailbox, i casi che si possono verificare sono quelli illustrati nelle figure \ref{fig:putonempty}, \ref{fig:putonfull}, \ref{fig:getonempty}, \ref{fig:getonfull}. In Tali illustrazioni sono raffigurate quattro esecuzioni parallele di due \textit{thread} di cui uno invoca il metodo \verb|get()| e l'altro il metodo \verb|put(...)|. La linea verticale centrale rappresenta il tempo. Sulla linea vi sono delle frecce con la punta rivolta verso sinistra o destra. Le frecce stanno ad indicare le istruzioni del metodo \verb|put| a sinistra, del metodo \verb|get| a destra, che vengono eseguite in quell'istante di tempo. L'esatto momento in cui viene eseguita un istruzione � irrilevante, e molto difficile da prevedere, ci� che � importante � l'ordine in cui le operazioni si alternano. I segmenti rossi stanno a simboleggiare gli intervalli di tempo in cui la \verb|get| (se a sinistra) o la \verb|put| (se a destra) si bloccano e sono in attesa.\\
Nella figura \ref{fig:putonempty} si parte dal presupposto che il buffer sia vuoto e che prima operazione che viene eseguita � una \verb|put(o)|: 
\begin{verbatim}
	if(buffer == null)
\end{verbatim}
l'operazione ha esito positivo per l'assunzione fatta inizialmente. 
 A questo punto si possono presentare due situazioni: 
\begin{enumerate}
	\item il \textit{thread} che sta richiamando il metodo \verb|put(o)| mantiene il controllo ed esegue la seconda istruzione dello stesso metodo (caso non riportato in figura, perch� implica l'esecuzione solo del metodo \verb|put|).
\begin{verbatim}
	buffer = o;
\end{verbatim}
	\item il \textit{thread} che sta richiamando il metodo \verb|put(o)| perde il controllo e viene eseguita la prima istruzione del metodo \verb|get()| (caso della figura \ref{fig:putonempty}). A questo punto il la verifica di esistenza di dati nella \textit{mailbox} fallisce: nella funzione \verb|get()|
\begin{verbatim}
if(buffer == null)
\end{verbatim}	
risulta essere \verb|true| ancora per l'assunzione iniziale. 
Il \textit{thread} chiamante la funzione \verb|get| viene messo in attesa sull'oggetto \textit{mailbox} mediante la funzione \verb|wait()|.\\
Il controllo passa al \textit{thread} che esegue \verb|put(o)|, e l'istruzione di inserimento del dato nel \textit{buffer} viene eseguita.
Questo \textit{thread} notifica tutti i \textit{thread} in attesa su tale semaforo e termina. \\
Il \textit{thread} chiamante la funzione \verb|get()| viene avviato eseguendo l' istruzione per copiare il valore del \textit{buffer} ed al passo successivo eliminarlo. L'alternativa a tale scenario sarebbe, la ripresa di controllo d pare di un \textit{thread} che invochi il metodo \verb|put(o)|. Questo verrebbe bloccato e fino alla notifica da parte del \verb|get()|.
\end{enumerate}

%\begin{figure}[h]
%	\centering
%		\includegraphics[width=.65\textwidth]{imgs/mailboxPutOnEmpty.jpg}
%	\caption{MailBox: put() viene eseguito prima di get() su un buffer vuoto.}
%	\label{fig:putonempty}
%\end{figure} 
%Le altre tre combinazioni fra metodi che viene invocato per primo e stato del buffer sono similmente spiegabili. 
%\begin{figure}[h]
%	\centering
%		\includegraphics[width=.65\textwidth]{imgs/mailboxPutOnFull.jpg}
%	\caption{MailBox: put() viene eseguito prima di get() su un buffer pieno.}
%	\label{fig:putonfull}
%\end{figure} 
%
%\begin{figure}[h]
%	\centering
%		\includegraphics[width=.65\textwidth]{imgs/mailboxGetOnEmpty.jpg}
%	\caption{MailBox: get() viene eseguito prima di put() su un buffer vuoto.}
%	\label{fig:getonempty}
%\end{figure} 
%
%\begin{figure}[h]
%	\centering
%		\includegraphics[width=.65\textwidth]{imgs/mailboxGetOnFull.jpg}
%	\caption{MailBox: get() viene eseguito prima di put() su un buffer pieno.}
%	\label{fig:getonfull}
%\end{figure} 



\subsubsection{Verifiche per il protocollo di comunicazione \textit{MailBox}}

Sono state fatte delle prove per collaudare il sistema mediante due \textit{thread}, un produttore ed un consumatore che comunicano tramite una \verb|MailBox|. Il produttore genera una sequenza ordinata di numeri dispari, crea un oggetto \verb|Point| nel seguente modo:
\begin{lstlisting}[language=java, style=eclipse,caption=Creazione di un oggetto Point,label=code:mailbox_experiment]
new Point(x++, Math.sin(x))
\end{lstlisting}

L'oggetto cos� creato viene inserito dallo stesso produttore nella \verb|MailBox|. Il consumatore pu� tentare di prelevare l'oggetto chiamando \verb|get()|.\\ 
I seguenti sono i risultati di una parte di sessione di comunicazione fra il produttore ed il consumatore. 

\begin{table}%
\centering
\begin{tabular}{l l|l l}
\ldots\\
put:& (70913.0, 0.921) & put:& (70919.0, 0.993)\\
get:& (70913.0, 0.921) & get:& (70919.0, 0.993)\\
put:& (70915.0, -0.737) & put:& (70921.0, -0.519)\\ 
get:& (70915.0, -0.737) & get:& (70921.0, -0.519)\\
put:& (70917.0, -0.307) & put:& (70923.0, -0.561)\\
get:& (70917.0, -0.307) & get:& (70923.0, -0.561)\\
\ldots
\end{tabular}
\caption{Parte di una sequenza di stringhe scambiate tra due \textit{thread}, un produttore ed un consumatore, che comunicano tramite una \textit{mailbox}. Il produttore genera una sequenza ordinata di numeri dispari, crea un oggetto \textit{point(x, sin(x))} ed il consumatore preleva il valore dalla \textit{MailBox}, permettendo al produttore di inserire il valore successivo.}
\end{table}

La stessa prova � stata riportata nell'ambiente Android
Stesso esperimento su Android\tm, con una semplice \textit{Activity} per visualizzare il punto generato dal \textit{thread} produttore, come il centro di un cerchio.
\begin{figure}
	\centering
		\includegraphics[width=1\textwidth]{imgs/mailBoxCommTestAndroid.jpg}
	\caption{Prova di funzionamento del protocollo di comunicazione fra \textit{thread} implementato su Android}
	\label{fig:mailbox_comm_test_android}
\end{figure}


\newpage
\subsection[HMM e Viterbi]{Modello deambulazione ed algoritmo di decodifica: implementazione di HMM e Viterbi}

L'interfaccia \verb|HiddenMarkovModel| � pensata per fornire uno scheletro per tutti i tipi di HMM. 
Chi implementa l'interfaccia, dovrebbe partire dalla creazione di un insieme di stati $S$ ed un insieme di simboli di osservazione $V$. A questo punto le dimensioni delle strutture dati che conterranno i parametri sono note, e le implementazioni dei seguenti metodi accessori agli stessi dovrebbero assicurarsi che vengano rispettate tali dimensioni.
\begin{itemize}
	\item \verb|setA(ArrayList<Object> mtx)|, \verb|getA()|: rendere accessibile la matrice di transizioni
	\item \verb|setB(ArrayList<Object> mtx)|,  \verb|getB()|: rendere accessibile la matrice di emissioni
	\item \verb|setPi(ArrayList<Double> vec)|, \verb|getPi()|rendere accessibile il vettore di \textit{prior}
	\item \verb|areValidObservations(ArrayList<Object> observations)|, chi implementa l'interfaccia potrebbe anche decidere di verificare la validit� di una nuova osservazione, verificandone l'appartenenza all'insieme delle osservazioni.
\end{itemize}
\subsubsection{Scheletro di un HMM}
Una prima implementazione dell'interfaccia � la classe \verb|HMM| che obbliga chi la vuole usare ad implementarla solo passando al costruttore un insieme di stati. Questo viene ottenuto rendendo privato il costruttore di default
\begin{lstlisting}[language=java, style=eclipse, caption=Costruttori HMM , label=code:hmm1]
private HMM() {...}
public HMM(HashMap<String, Object> states) {...}
\end{lstlisting}
L'assegnazione dei parametri avviene solo dopo una serie di controlli sui dati in ingresso che mirano a garantire una serie di propriet� degli stessi.\\
Assegnazione della matrice di transizioni(vedi Codice \ref{code:HMM_setA}): \\per prima cosa la matrice deve essere quadrata con dimensione pari al numero di stati. Successivamente a tale verifica si deve verificare la validit� dei valori di probabilit� di transizione per ciascuna coppia di stati, nonch� la loro completezza come alternative probabilistiche (vale a dire che la loro somma � uguale a 1). Questa � un operazione costosa, ma necessaria, per garantire un minimo di correttezza.

\lstset{language=Java, basicstyle=\ttfamily,keywordstyle=\color{javapurple}\bfseries,
backgroundcolor=\color{lightBlue},stringstyle=\color{javared},commentstyle=\color{javagreen},
morecomment=[s][\color{javadocblue}]{/**}{*/},numbers=left,numberstyle=\tiny\color{black},
stepnumber=1,numbersep=10pt,tabsize=4,showspaces=false,showstringspaces=false,
frame=single,frameround=fttt, captionpos=b, breaklines=true,breakatwhitespace=false}
\begin{lstlisting}[caption=Impostazione della matrice di transizione, label=code:hmm_setA]
public void setA(ArrayList<Object> mtx) {
		boolean isFitTransitionMtx = true;
		
		if (mtx != null && 	
				// la matrice di transizione deve essere quadrata 
				// della cardinalit� e della stessa dimensione
				// dell'insieme degli stati
				mtx.size() == S.size() && 
			 ((ArrayList<Object>) mtx.get(0)).size() == S.size()) {
			 
			// ogni elemento della matrice deve essere un valore
			// di probabilit� valido: 
			for (int i = 0; i < S.size(); i++){
				isFitTransitionMtx &= 
					StatisticsOperations.areCompleteProbabilisticAlternatives((ArrayList<Object>) transitionsMtx.get(i)); 
			}
			// solo a questo punto assegno i valori alla matrice
			if (isFitTransitionMtx){
				A = transitionsMtx;
			}
		}else {
			//lancio un errore oppure
			//forzo l'utente a inizializzare 
			//A correttamente
		}
	}
\end{lstlisting}

La stessa procedura viene eseguita per l'assegnazione delle \textit{prior} (vedi Codice \ref{code:HMM_setPI}):
\begin{lstlisting}[caption=Impostazione del vettore delle prior, label=code:HMM_setPI]
public void setPI(ArrayList<Object> prior) {
	if (prior != null && 
			prior.size() == S.size() && 
			StatisticsOperations.areCompleteProbabilisticAlternatives(prior)){
		PI = prior;
	} else {
		//lancio un errore oppure
		//forzo l'utente a inizializzare 
		//A correttamente
	}		
}
\end{lstlisting}

L'implementazione delle matrici di emissione � delegato a HMM specializzate, ad emissioni discrete o continue. 
\subsubsection{HMM ad emissioni discrete}
Un'implementazione concreta di un HMM (utilizzabile come oggetto) � quella della HMM ad emissioni discrete
\begin{lstlisting}[caption=Firma della classe HMM ad emissioni discrete, label=code:DiscreteHMM_B1]
public class DiscreteHMM extends HMM {...}
\end{lstlisting}
Questa eredita dalla classe \verb|HMM| tutto ci� che contiene, e fornisce una sua implementazione della matrice di emissioni (vedi Codice \ref{code:DiscreteHMM_B1}). Anche in questo caso vengono eseguiti dei controlli sui valori che vengono assegnati: la dimensione della matrice di emissioni $B:N \times M$ dove $N=|S|$ (Cardinalit� dell'insieme degli stati dell'HMM) e $M=|V|$ cardinalit� dell'insieme dell'alfabeto di osservazioni; i valori assegnati forniscono un'insieme completo di alternative probabilistiche.

\begin{lstlisting}[caption=Impostazione della matrice di emissioni, label=code:DiscreteHMM_B2]
public void setB(ArrayList<Object> mtx) {
		boolean isFitEmissionMtx = true;
		// il numero di righe della matrice di 
		// emissione deve essere pari al numero di stati
		if (mtx != null && 
				mtx.size() == S.size()&& 
			// il numero di colonne di mtx essere
			// pari al numero di simboli osservabili
			((ArrayList<Object>) mtx.get(0)).size() == V.size()) {
			for (int i = 0; i < S.size(); i++) {
				// la somma di tutte le probabilit� di osservazione
				// deve essere pari a 1
				isFitEmissionMtx &= StatisticsOperations
						.areCompleteProbabilisticAlternatives((ArrayList<Object>) mtx
								.get(i));
			}
			if (isFitEmissionMtx) {
				B = mtx;
			}
		} else {
			//lancio un errore oppure
			//forzo l'utente a inizializzare 
			//B correttamente
		}
}
\end{lstlisting}



\subsubsection{HMM ad emissioni continue}
L'implementazione della HMM ad emissioni continue, nella gerarchia di classi, � un'altra diramazione a partire dalla classe \verb|HMM|. Una HMM ad emissioni continue ha una distribuzione di probabilit� sulle emissioni. Dato che la distribuzione pi� comunemente utilizzata � quella Gaussiana (o Normale), � stata implementata una classe con tale distribuzione.
\begin{lstlisting}[caption=Firma della classe HMM ad emissioni continue, label=code:NormalHMM]
public class NormalHMM extends HMM 
\end{lstlisting}

\subsubsection{Operazioni sulle HMM}
La classe di base che gestisce le operazioni sulle HMM discrete � 
\begin{lstlisting}[caption=Firme dell'operatore delle HMM, label=code:HMM_operations]
public class HMMOperations
\end{lstlisting}
Questa permette di eseguire gli algoritmi pi� importanti sulle HMM, come l'algoritmo \textit{Forward-Backward} (vedi Appendice \ref{alg:fwd}, \ref{alg:bckwd}), l'algoritmo di Viterbi (vedi Appendice \ref{alg:viterbi}) e la sua variante in differita.
  
Per le operazioni sulle HMM continue la classe che implementa gli algoritmi sopra menzionati � 
\begin{verbatim}
ContinuousHMMOperations
\end{verbatim}

Dopo aver creato ed inizializzato una \verb|NormalHMM|, si pu� richiamare ad esempio la segmentazione nel seguente modo:
\begin{lstlisting}[caption=Applicazione dell'operatore su HMM, label=code:contHMMOp]
	// HMM ed operatore 
	ContinuousHMMOperations contHMMOp
	NormalHMM nHMM
	...
	// parametri 
	// states, A, Pi, emissionMtx
	...
	// inizializzazione HMM
	// con parametri creati
	nHMM = new NormalHMM(states);
	nHMM.setA(A);
	nHMM.setPI(Pi);
	nHMM.setContB(emissionMtx);
	//collego l'operatore con l'HMM
	contHMMOp = new ContinuousHMMOperations(nHMM);
	...
	// un modo per acquisire osservazioni di 
	// prova � di leggerli da un file
	observation = reader.readLine()
	...
	// segmentazione
	contHMMOp.onlineViterbi(observation)
\end{lstlisting}

\chapter[Valutazione e Risultati]{Valutazione delle prestazioni del sistema in condizioni di uso reali e risultati}
La valutazione del funzionamento e delle prestazioni del sistema di segmentazione della deambulazione creato, � stata fatta grazie alla misurazione indiretta della velocit� di deambulazione mediante il sistema stesso contemporaneamente alla misurazione della medesima velocit� mediante un sistema GPS commerciale e quini un confronto fra i due valori misurati. La procedura seguita in questo capitolo non vuole essere una valutazione formale delle prestazioni del sistema, ma un'indicazione approssimativa del funzionamento dello stesso, questo spiega il numero ridotto di soggetti e tempo di prova. 
\begin{figure}
	\centering
		\includegraphics[width=1\textwidth]{imgs/ValutazionePrestazioniSistema.jpg}
	\caption{Schema riassuntivo della metodologia di verifica del funzionamento del sistema di segmentazione.}
	\label{fig:gyroPattern}
\end{figure}
Per valutare le prestazioni del modello in condizioni di uso reale � stato pensato un sistema a due fasi (vedi Figura \ref{fig:gyroPattern}): 

\begin{enumerate}
	\item Fase di lavoro in laboratorio 	
		\par\textsc{acquisizione dati in laboratorio}: sono stati acquisiti dati di deambulazione (vedi Tabella 	 \ref{tab:TabellaRiassuntivaRaccoltaDatiLab}).
	Le prove sono state compiute su un tappeto rullante con pendenza $0^\circ$, nell'intervallo di velocit� $[2-8]\, km/h$. La prima sessione alla velocit� di $2\, km/h$ e con un incremento di $1\, km/h$ in ciascuna sessione successiva. Ciascuna sessione � stata della durata di $1:30\, minuti$ .
	\begin{table}[htbp]
		\centering
		\begin{tabular}{|l|c|}
			\hline		
				\textbf{Attivit�}& cammino\\
				\hline
				\textbf{Soggetti}& $1$\\
				\hline
				\textbf{Velocit�}& $\{2,3,4,5,6,7,8\}\, km/h$\\
				\hline
				\textbf{Durata}& \multicolumn{1}{c|}{$1:30\,minuti$  per attivit�}\\
				\hline
				\textbf{Strumenti} & \multicolumn{1}{c|}{IMU, Smartphone, tappeto rullante}\\
				\hline
				\textbf{Luogo}& \multicolumn{1}{c|}{Laboratorio}\\
				\hline
				\textbf{Dati raccolti} & \multicolumn{1}{c|}{valori giroscopio segmentati}\\
				\hline
				\textbf{Frequenza campionamento} & \multicolumn{1}{c|}{$100\,Hz$}\\
				\hline
			\end{tabular}
		\caption{Riassunto della procedura di raccolta dati in laboratorio}
		\label{tab:TabellaRiassuntivaRaccoltaDatiLab}
	\end{table}
	
	Le sessioni sono state tutte eseguite posizionando saldamente una IMU (vedi Appendice \ref{sec:sensori}) sul collo del piede dei soggetti mediante un cinturino in velcro. Il tipo di sensore usato � un giroscopio monoassiale con asse di sensibilit� orientato sul piano mediale-laterale (sagittale) (vedi Figura \ref{fig:HumanBodySPL}).\\ 
	I dati sono stati raccolti mediante l'applicazione implementata su \textit{Smartphone}, con relativa segmentazione. 
	
	\par\textsc{stima della cadenza}: la cadenza stimata in laboratorio come definita in tabella \ref{parametri_velocit�} � il numero di cicli di deambulazione per unit� di tempo (cicli/secondo) come mostra la Formula \ref{eq:cadenzaLab}
		\begin{equation}
			c_{LAB} = \dfrac{\text{numero cicli di deambulazione}} {1\, secondo}
		\label{eq:cadenzaLab}
		\end{equation}
	
	 Il numero dei cicli di deambulazione si ottiene contando il numero di volte che si presenta un dato evento (ad esempio HS) in un secondo.
	\par\textsc{relazione fra cadenza e velocit� stimata sul soggetto}: dato che le prove sono state fatte su un tappeto rullante, la velocit� � nota (vedi Figura \ref{fig:cadVSspeedLinearfit}) ed � denominata $TMspeed$. L'obbiettivo � trovare una relazione tra la cadenza calcolata e la velocit� nota, mediante la regressione lineare \ref{eq:cadenzaLab}.
		\begin{equation}
			\theta_1 * c_{LAB}^2 + \theta_2 * c_{LAB} + \theta_3 = TMspeed
		\label{eq:cadenzaLab}
		\end{equation}
\begin{figure}
	\centering
		\includegraphics[width=1.2\textwidth]{imgs/cadVSspeedLinearfit.jpg}
	\caption{Relazione fra la cadenza $c_{LAB}$ e la velocit� del tappeto rullante $TMspeed$.}
	\label{fig:cadVSspeedLinearfit}
\end{figure}

				
I parametri ottenuti dalla regressione sono $\theta =[\theta_1, \theta_2, \theta_3] $
	\begin{table}[htbp]
		\centering
		\begin{tabular}{|l|c|}
			\hline
			$\theta_1$ & $4.4432$\\			
			\hline
			$\theta_2$ & $1.8888$\\
			\hline
			$\theta_3$ & $1.6493$\\
			\hline
		\end{tabular}
		\caption{Soluzione della regressione lineare \ref{eq:cadenzaLab}}
		\label{tab:relazioneCadenzaVelocit�}
	\end{table}
	
		\item Fase di lavoro all'aria aperta
		\par\textsc{acquisizione dati giroscopio e segmentazione} all'aria aperta sono stati acquisiti dati di deambulazione (vedi Tabella \ref{tab:TabellaRiassuntivaRaccoltaDatiAperto}). Le prove sono state compiute su un percorso piano (pendenza media $0^\circ$), ad andatura a velocit� normale per il soggetto in questione. La sessione � stata della durata di circa $2\, minuti$.
		\begin{table}[htbp]
		\centering
		\begin{tabular}{|l|c|}
			\hline		
				\textbf{Attivit�}& cammino\\
				\hline
				\textbf{Soggetti}& 1\\
				\hline
				\textbf{Velocit�}& $\approx 5\, km/h$\\
				\hline
				\textbf{Durata}& \multicolumn{1}{c|}{$\approx 2\, minuti$ }\\
				\hline
				\textbf{Strumenti} & \multicolumn{1}{c|}{IMU, Smartphone}\\
				\hline
				\textbf{Luogo}& \multicolumn{1}{c|}{Aperto}\\
				\hline
				\textbf{Dati raccolti} & \multicolumn{1}{c|}{valori giroscopio segmentati}\\
				\hline
				\textbf{Frequenza campionamento} & \multicolumn{1}{c|}{$100\,Hz$}\\
				\hline
			\end{tabular}
		\caption{Riassunto della procedura di raccolta dei dati all'aperto.}
		\label{tab:TabellaRiassuntivaRaccoltaDatiAperto}
	\end{table}
	\par\textsc{stima della cadenza}: la cadenza della deambulazione all'aperto, $c_{OUT}$, � stata calcolata con la stessa procedura della fase precedente.
	
	\par\textsc{stima della velocit� di deambulazione}: la velocit� di deambulazione all'aperto ricavata dal sistema inerziale, indicata con $IMUspeed$ � stata ottenuta a partire dalla relazione tra cadenza e velocit� calcolata nella Fase 1 della valutazione (vedi Tabella \ref{tab:relazioneCadenzaVelocit�}) e dalla cadenza stimata nella seconda fase.
	
	\par\textsc{stima della distanza percorsa}: $IMUdistance$, dato che il tempo di percorrenza � noto, dalla stima di velocit� si ottiene anche la distanza percorsa.
	\begin{equation}
\begin{split}
\displaystyle IMUdistance = \int_{t_{start}}^{t_{end}}{IMUspeed} \; dt\\
\end{split}
\label{eq:IMUdistance}
\end{equation}

	\par\textsc{acquisizione dati GPS} contemporaneamente alla segmentazione, mediante il dispositivo GPS \footnote{Global Positioning System} integrato nello \textit{Smartphone}, sono state acquisite la velocit� di spostamento $GPSspeed$ e la distanza percorsa. I dati acquisiti sulla distanza percorsa sono stai scartati in quanto con margine di errore elevato. 

		\par\textsc{stima della distanza percorsa mediante GPS} $GPSdistance$, dato che il tempo di percorrenza � noto, dai dati di velocit� acquisiti mediante il GPS si ottiene la distanza percorsa. 
				
\begin{equation}
\begin{split}
\displaystyle GPSdistance = \int_{t_{start}}^{t_{end}}{GPSspeed} \; dt\\
\end{split}
\label{eq:GPSdistance}
\end{equation}
\end{enumerate} 

In fine, i risultati ottenuti con le due modalit� vengono confrontati: 
	\par\textsc{confronto velocit�}: le due velocit� $IMUspeed$ ottenuta con il sistema di misurazione inerziale e $GPSspeed$ vengono comparate (vedi Figura \ref{fig:IMUspeedGPSspeedVStime}, \ref{fig:IMUspeedVSGPSspeed}, \ref{fig:IMUspeedVSGPSspeedBA}). 
	Le due velocit� sono simili: lo scarto quadratico medio (\textit{Root Mean Square})  
	\begin{equation}
	1/2 * (\sqrt{IMUspeed^2 + GPSspeed^2}) = 0.38\, km/h
	\label{eq:ermsSpeed}
	\end{equation}
	Come mostra il grafico Bland-Altman\footnote{Un grafico Bland-Altman � una tecnica grafica di rappresentazione di dati per misurare la concordanza fra due procedure di misurazione di una quantit�. Il metodo � molto usato nella ricerca clinica per confrontare uno strumento di diagnosi nuovo con uno gi� affermato.} \ref{fig:IMUspeedVSGPSspeedBA}, la maggior parte (95\%) dei valori di differenza sono all'interno di un intervallo di confidenza.
	
	\par\textsc{confronto distanza}: le due distanze calcolate dalle velocit�, vengono confrontate (vedi Figura \ref{fig:IMUdistanceGPSdistanceVStime}, \ref{fig:IMUdistanceVSGPSdistance}, ). 
	Le distanze contengono sia gli errori fatti sulle misure delle velocit� che su se stesse.
	Misurando la differenza fra le distanze calcolate, si vede che al termine del periodo di acquisizione dei dati (circa $2\, minuti$) vi � una deriva (o una sovrastima) della $IMUdistance$ inferiore ai $5\,m$ su circa $120\,m$. Il $4\%$ di errore sulla stima della distanza � paragonabile all'errore commesso dai sistemi commerciali GPS, quindi � un valore accettabile.
	\begin{equation}
	|IMUdistance - GPSdistance| < 5\,m
	\label{eq:ermsDistance}
	\end{equation}

\begin{figure}[h]
	\centering
		\includegraphics[width=.8\textwidth]{imgs/speedSVR4.jpg}
	\caption{Confronto fra le due velocit� $IMUspeed$ (traccia rossa) e $GPSspeed$ (traccia blu) rispetto al tempo in $km/h$ in funzione del tempo in secondi. Mostra una forte somiglianza tra tra le due velocit�.}
	\label{fig:IMUspeedGPSspeedVStime}
\end{figure}
\begin{figure}
	\centering
		\includegraphics[width=.8\textwidth]{imgs/speedSVR5.jpg}
	\caption{Grafico a dispersione (\textit{scatter plot}) delle due velocit� $IMUspeed$ e $GPSspeed$. Mostra come ci sia una buona corrispondenza fra i due gruppi di valori.}
	\label{fig:IMUspeedVSGPSspeed}
\end{figure}

\begin{figure}
	\centering
		\includegraphics[width=.8\textwidth]{imgs/speedSVRBlandAltmann.jpg}
	\caption{Confronto fra le due velocit� $IMUspeed$ e $GPSspeed$ mediante un Bland Altman \textit{plot} o grafico a differenza o grafico delle differenze medie. La linea rossa rappresenta la linea a differenza nulla, la linea nera la differenza media, mentre le linee verdi la deviazione standard (o intervallo di confidenza). Il 95\% dei valori sta all'interno dell'intervallo di confidenza. }
	\label{fig:IMUspeedVSGPSspeedBA}
\end{figure}

\begin{figure}
	\centering
		\includegraphics[width=.8\textwidth]{imgs/displSVR1.jpg}
	\caption{Confronto fra le due distanze $IMUdistance$ (traccia blu) e $GPSdistance$ (traccia rossa) rispetto al tempo. Mostra che la IMU sovrastima la distanza. Al termine dell'esperimento (a circa $120\,m$ di distanza percorsa) l'errore � commesso � di circa $5m$.}
	\label{fig:IMUdistanceGPSdistanceVStime}
\end{figure}

\begin{figure}
	\centering
		\includegraphics[width=.8\textwidth]{imgs/displSVR2.jpg}
	\caption{Grafico a dispersione delle due distanze $IMUdistance$ e $GPSdistance$ mostra una elevato grado di allineamento fra le distanze misurate con i due strumenti.}
	\label{fig:IMUdistanceVSGPSdistance}
\end{figure}

\chapter{Risultati e Conclusioni}
%\myChapter{Risultati e Conclusioni}

--------------------------------------------------------------
%% Backmatter
%--------------------------------------------------------------

\addtocontents{toc}{\protect\mbox{}\protect\hrulefill\par}
\addtocontents{toc}{\protect\mbox{}\protect\hrulefill\par}
\part{Appendice}
\appendix
%\appendixpage	
\addtocontents{toc}{\cftpagenumbersoff{chapter}}	
\setcounter{tocdepth}{1}


\chapter{Sulle HMM}
%\myChapter{Sulle HMM} 
%\HMM=Modelli di Markov a Stati Nascosti
\label{cap:hmm}
I HMM sono strumenti di modellazione di sequenze temporali. Un comune esempio di applicazione � il riconoscimento della comunicazione verbale (\textit{Speech Recognition}) \cite{tutorial_hmm_application_speech_recognition}. 
Una sequenza temporale � un segnale.

\begin{definition}[Segnale]
	Un segnale � il risultato osservabile di un processo,
	che in base al numero di sorgenti di provenienza viene detto
		\begin{itemize}
			\item \textbf{Puro}, se a sorgente unica
			\item \textbf{Corrotto} altrimenti
		\end{itemize}
\end{definition}


Un segnale pu� essere di natura discreta o continua rispetto al tempo. Ad esempio un segnale che consista in una sequenza di caratteri viene incluso fra i segnali discreti, mentre la comunicazione verbale fra i segnali continui.\\

La sorgente di un segnale (il processo), a sua volta, pu� essere stazionaria, se le sue propriet� statistiche non variano nel tempo, o non stazionaria altrimenti.

\section*{Problema: Modellare segnali}
Il problema della modellazione dei segnali � di fondamentale importanza, in molti settori.
Le motivazioni principali sono la riproduzione dei segnali e la scoperta delle loro sorgenti.
Esistono varie tipologie di modelli dai quali scegliere per modellare al meglio un dato segnale. La suddivisione maggiore � quella fra modelli deterministici e stocastici. 
\begin{itemize}
	\item {Modelli deterministici}: descrivono il segnale mediante le sue propriet� note. Ad esempio la luce ha una velocit� pari a $300.000 km/s$.
	\item {Modelli stocastici}: descrivono le propriet� statistiche del segnale. Un esempio di modello statistico sono i processi gaussiani, i processi di Markov e le HMM (\textit{Hidden Markov Models}).  In questi modelli si assume che il segnale possa essere caratterizzato come un Processo Parametrico Stocastico, con parametri stimabili in modo algoritmico. 
\end{itemize}

I modelli deterministici descrivono i comportamenti di tutti i parametri di un segnale, in tutte le condizioni possibili. In molti casi per�, i segnali di interesse sono abbastanza complessi, o oscurati, da non poterne conoscerne tutti i parametri, quindi i modelli deterministici risultano inefficaci.
L'alternativa sono i modelli stocastici che descrivono solo un sottoinsieme dei parametri che caratterizzano un segnale, oppure una risultante di questi, e dato che il segnale si comporta solo in parte come i parametri descritti, il modello pu� fornire solo una descrizione probabilistica del segnale. 

\begin{definition}[Spazio campionario $\Omega$]
\begin{equation}
\Omega = \{\omega: \omega \quad \text{� il risultato di un esperimento}\}
\label{eq:sample_space}
\end{equation}
\end{definition}

\begin{definition}[Evento $E$]
\begin{equation}
E \subseteq \Omega 
\label{eq:event}
\end{equation}
\end{definition}

\begin{definition}[Spazio di probabilit� $<\Omega, F, \wp>$].\\

$F = \{E: E \quad \text{gode di qualche propriet�}\} \quad |F|\geq 0$\\

$\wp:F \rightarrow \Re \quad \text{t.c.}$
\begin{enumerate}
	\item $\wp(E) \geq 0$
	\item $\wp(\Omega)=1$
	\item $E_i \cap E_j = \emptyset \Leftrightarrow \wp (E_i \cup E_j) = \wp (E_i)+ \wp(E_j)$
\end{enumerate}
\end{definition}

\begin{definition}[Variabile Stocastica $X$]
Una variabile stocastica quantifica gli elementi dello stato campionario 
\begin{equation}
X : \Omega \rightarrow \Re
\label{eq:ramdom_variable}
\end{equation}
Queste possono essere discrete:
\begin{equation}
\wp(X=k)\overset{\underset{\mathrm{def}}{}}{=}\wp(\{w:X(w)=k\}) 
\label{eq:discrete_rand_vars}
\end{equation}
oppure possono essere continue:
\begin{equation}
\wp(a\leq X \leq b)\overset{\underset{\mathrm{def}}{}}{=} \wp(\{w:a \leq X(w) \leq b\}) 
\label{eq:cont_rand_vars}
\end{equation}
\end{definition}

\begin{definition}[Processo Stocastico $St$]
Dato uno spazio di probabilit� $<\Omega, F, \wp>$ 
\begin{equation} 
St=\{F_t: t \in T\}\quad \text{dove } t \; \text{� il tempo}
\label{eq:processo_stocastico}
\end{equation} 
\end{definition}


\section*{Processi di Markov}
Un processo di Markov � un processo stocastico che gode della propriet� di Markov o assenza di memoria. 
I processi di Markov si possono essere raggruppati in base al tempo che pu� essere discreto o continuo, ed allo spazio degli stati che pu� essere anche esso discreto e finito o continuo. 

\subsection*{Tempo e spazio discreti}
Ad ogni unit� temporale il processo transita casualmente da uno stato ad un altro. Dunque � impossibile prevedere in modo deterministico in che stato si trover� il sistema in un istante di tempo futuro. 

\begin{definition}[Processi di Markov (spazio-tempo) Discreti]$<N,A,\pi>$\\
Processo stocastico con:
\begin{itemize}
	\item un numero finito di stati $S = \{s_1,\ldots, s_N\}$,
	\item un vettore di probabilit� a priori $\pi$ che determina $\forall i=1,...,N$ la probabilit� che il processo sia nello stato $S_i$ a tempo $t_1=0$, ovvero $\wp(q_0 = \pi_i)$, dove $q_k$ � lo stato del processo di Markov a tempo $k$.
	\item una matrice di transizione ($N\times N$) che indica la probabilit� di transire da ogni stato ad ogni altro. 
\end{itemize} 

\end{definition}

\begin{figure}
	\centering
		\includegraphics[width=1\textwidth]{imgs/descreteMarkovProcesses.jpg}
	\caption{Rappresentazione di un Processo di Markov Discreto generico, con una data sequenza di osservazioni $O_t$: un modello con $N$ stati nascosti ed $M$ possibili emissioni, $N$ probabilit� a priori $\pi_i$ (una per stato) a tempo $t = 0$, $N^2$ probabilit� di transizione tra stati $a_{ij}$ (per ciascuna coppia di stati $(S_i, S_j)$ con $i,j = 1,...,N$) per ogni unit� temporale $0 < t < T$ ed $N \times M$ probabilit� di emissione $b_{jk}$ (per ciascuna coppia stato - emissione $(S_i,v_k)$ con $i=1,...,N$ e $k=1,...,M$).}
	\label{fig:descreteMarkovProcesses}
\end{figure}

 Si pu� dare una descrizione probabilistica completa dei processi di Markov Discreti con la seguente equazione:

\begin{equation}
\wp [q_t = s_j | q_{t-1} = s_i, \ldots, q_0 = s_p]
\label{eq:fullProbDescription}
\end{equation}
Ovvero la probabilit� che il processo di Markov si trovi nello stato $s_j$ a tempo $t$ dato che a tempo $t-1$ si trovava nello stato $s_i$, e ..., e nello stato iniziale $q_0$ era nello stato $s_p$.
Un caso speciale di \eqref{eq:fullProbDescription} in cui la probabilit� che il sistema si trovi in un determinato stato nel presente dipende solo dall'istante di tempo precedente e non da tutta la storia a partire dal primo istante di tempo.

\begin{definition}[Modelli di Markov del Primo Ordine]
	\begin{equation}
		\wp[q_t = S_j | q_{t-1} = S_i]
		\label{eq:first_Order_MM}
	\end{equation}
\end{definition}

 Le Catene di Markov ad ogni istante temporale compiono una transizione di stato con una probabilit� nota. Tale probabilit� � descritta nella Matrice delle Probabilit� di Transizione.
\begin{definition}[Matrice delle Probabilit� di transizione]
	\begin{equation}
		A = a_{i,j} = \wp[q_t = S_j | q_{t-1} = S_i]\quad \text{con} 1 \leq i,j \leq N
		\label{eq:trans_prob}
	\end{equation}
\end{definition}

Propriet� delle $a_{ij}$ derivanti dal fatto che sono dei valori di probabilit�:
\begin{enumerate}
	\item $a_{ij}\geq 0$
	\item $\displaystyle\sum_{j=1}^{N}{a_{ij}}=1$
\end{enumerate}
All'istante di tempo iniziale $t = 0$, lo stato della Catena di Markov � determinato da una distribuzione di probabilit�. Un vettore $\pi: 1\times N$ che a ciascuno stato fa corrispondere la probabilit� che sia lo stato iniziale, questa � nota come probabilit� a Priori o probabilit� dello stato iniziale.

\begin{definition}[Distribuzione di probabilit� dello stato iniziale]
\begin{equation}
	\pi_i = \wp(q_1 = S_i) \quad \text{con} 1 \leq i \leq N
\label{eq:prior}
\end{equation}
  
\end{definition}


\section*{HMM}
\label{sec:HMM}
Avvolte le osservazioni sono funzioni probabilistiche di qualche stato nascosto (cio� esiste un processo stocastico  nascosto che produce una sequenza di osservazioni).  

%Il problema � come costruire un modello di marcoviano che le spieghi? 
%\TODO questo algoritmo fa veramente pena, valutare se mantenerlo
%\begin{algorithm}    
%\caption{Creare HMM}
%\label{alg:HMM_create}                                                  
%\begin{algorithmic}[1]                    
%WHILE{\COMMENT{modello = miglior spiegazione delle osservazioni}}
%	\STATE \COMMENT{Stabilire il numero di stati}
%	\STATE \COMMENT{Stabilire a cosa corrisponde ogni stato nella realt� }
%\ENDWHILE
%\RETURN \COMMENT{modello}
%\end{algorithmic}
%\end{algorithm}

\begin{definition}[Matrice di probabilit� delle Osservazioni]
\begin{equation}
\begin{split}
B &= \{b_j(k)\} \\
&\text{dove } b_j(k) = \wp[v_k \quad \text{all'istante } t | q_t = s_j] \quad \text{con } 1\leq j \leq N \quad , 1\leq k \leq M
\end{split}	 
\label{eq:emissionMtxDef}
\end{equation}
\end{definition}

\begin{definition}[HMM a Osservazioni discrete]
 \begin{equation}
	HMM = <N, M, A, B, \pi>
\label{eq:hmm}
\end{equation}
dove:
 \begin{enumerate}
	\item $N = |S|$ � l'insieme degli stati $S = \{s_1,\ldots, s_N\}$,
	\item $M = |V|$ � un insieme finito di Simboli di Osservazione $V = \{v_1, \ldots, v_M\}$ ,
	\item $A$ � la matrice di transizione definita in \eqref{eq:trans_prob},
	\item $B$ � la matrice di probabilit� dei Simboli di Osservazione definita in \eqref{eq:emissionMtxDef},
	\item $\pi$ � il vettore di probabilit� a priori definito in \eqref{eq:prior}.
\end{enumerate}
\end{definition}

Una HMM pu� essere usato per generare una sequenza di osservazioni $O=\{O_1,O_2,\ldots,O_T\}$ nel seguente modo:
\begin{algorithm}    
\caption{genera sequenze con HMM}
\label{alg:HMM_seq_gen}                                                  
\begin{algorithmic}[1]                    
	\STATE $q_1 = \displaystyle\arg\max_{1\leq i \leq N}{\pi_i}$
	\FOR{ $t = 1, \ldots, T ; t++$}
		\STATE $o_t = \displaystyle\max_{1 \leq s \leq N}[b_{{q_t},s}]$
		\STATE $q_t = \displaystyle\arg\max_{1\leq j \leq N}{a_{i,j}}$
	\ENDFOR	
\RETURN \COMMENT{modello}
\end{algorithmic}
\end{algorithm}

\subsection*{HMM ad emissioni continue}
Nel caso in cui le osservazioni siano continue � necessario avere un modello che associ ad ogni stato una distribuzione di emissione, invece che un singolo valore. Un esempio di distribuzione � la gaussiana mono variata o ad una variabile (vedi immagine \ref{img:monovargaussian}). In questo caso la definizione della matrice di emissione descritta in \eqref{eq:emissionMtxDef} diventa
\begin{equation}
B = b_j(x) = \mathcal{N}(x,\mu_j,\sigma_j) = \frac{1}{\sigma_j\sqrt{2\pi}}\textbf{e} ^{-\frac{(x-\mu_j)^2}{2 \sigma_j^2}}\quad
\text{per} 1 \leq j \leq N
\label{eq:continuousEmissionMtx}
\end{equation}

%\begin{figure}[h]
%	\centering
%		\includegraphics[width=.9\textwidth]{imgs/monovargaussian.jpg}
%	\caption{Esempi di distribuzioni gaussiane monovariate}
%	\label{fig:monovargaussian}
%\end{figure}


La formula pu� essere generalizzata su due fronti: numero di variabili in ingresso: gaussiana mutivariata (vedi figura \ref{bivargaussian}) oppure sul numero di gaussiane che vengono combinate: mistura di gaussiane (vedi figura \ref{gaussianmixture}).

%\begin{figure}[h]
%	\centering
%		\includegraphics[width=.9\textwidth]{imgs/bivariategaussian.jpg}
%	\caption{Esempio di distribuzione gaussiana multivariata, in questo caso bivariata}
%	\label{fig:bivargaussian}
%\end{figure}

%\begin{figure}[h]
%	\centering
%		\includegraphics[width=.9\textwidth]{imgs/gaussianmixture.jpg}
%	\caption{Esempi di misture di gaussiane}
%	\label{fig:gaussianmixture}
%\end{figure}



\begin{equation}
B = b_j(x) = \displaystyle \sum_{m = 1}^{M} c_{j,m}\mathcal{N}(x,\bm\mu_j,\bm\Sigma_j)\quad \text{ per } 1 \leq j \leq N
\label{eq:continuousEmissionMtxGauusianMixtures}
\end{equation}

Ovviamente � possibile applicare qualunque tipo di distribuzione al posto della gaussiana.\\
Dalla definizione \eqref{eq:continuousEmissionMtxGauusianMixtures} segue 
\begin{equation}
\int_{-\infty}^{\infty}{b_j(x)}dx = 1 \quad \text{ per } 1 \leq j \leq N
\label{eq:contEmissGaussMixProperty}
\end{equation}


\subsection*{Tipi di HMM}
\label{sec:tipi_hmm}
Vi sono casi particolari di HMM che sono considerati importanti per la loro ricorrenza nella descrizione di sistemi naturali. 
Il pi� generico tipo di HMM � noto come Ergodico, in cui ogni stato � connesso ad ogni altro stato, quindi la matrice di transizione � una matrice senza zeri. \\
Un modello pi� significativo � il modello Bakis o Sinistra-Destra. In questo modello l'aumentare del tempo, causa una variazione monotona crescente modulo $N$ sull'indice degli stati. Questo modello � conforme ai segnali le cui propriet� variano con il tempo. Le matrici di transizione dei modelli Sinistra-Destra, godono della propriet�
\begin{equation}
\begin{split}
 a_{i,j} = 0 \quad \forall j < i\\
 a_{i,j} = 0 \quad j>i+\Delta
\end{split} 
\label{eq:bakisTransitionMtx}
\end{equation}
La seconda condizione serve ad evitare che vi siano grossi balzi in avanti sugli stati, di solito $\Delta = 2$.
Inoltre le probabilit� iniziali hanno il vincolo
\begin{equation}
\pi_i = 1 \Leftrightarrow i = 1
\label{eq:initial_prob_condition}
\end{equation} 
supponendo di avere gli stati ordinati da $1$ ad $N$

\section*[Tre problemi per le HMM]{Tipi di problemi che si affrontano con le HMM}

Generalmente con le HMM si affrontano 3 tipologie di problemi. 

\begin{enumerate}
	\item \textbf{Valutazione}: dati $O$ e $\lambda$, calcolare $\wp(O|\lambda)$ in modo ottimale.
	\item \textbf{Decodifica}: dati $O$ e $\lambda$, trovare la sequenza di stati $Q=q_1,q_2,\ldots, q_T$ ``migliore'', secondo un criterio di ottimalit�, che possa aver generato $O$.
	\item \textbf{Apprendimento}: dato $\lambda$, regolarne i parametri per massimizzare $\wp(O|\lambda)$?
\end{enumerate}


\subsubsection*{Valutazione} 
Il problema della valutazione (\textit{Evaluation}) consiste nel calcolare la probabilit� che una certa sequenza di osservazioni sia stata prodotta da un dato modello.\\
La soluzione del problema permette, dati diversi modelli candidati $\lambda_1, \ldots, \lambda_n$ ed una sequenza di osservazioni $O$, di scegliere $\lambda_k$ con $\displaystyle \max_{1\leq i \leq n}\{\wp(O|\lambda_i)\}$.\\

Una soluzione na�ve del problema della valutazione � quello di enumerare tutte le possibili sequenze di stati (detti anche percorsi)  $Q = q_1, \ldots, q_T$ dove $q_i \in S={s_1,s_2,\ldots, s_N}$ e calcolare per ciascuna la probabilit� che l'osservazione sia stata prodotta da essa e moltiplicare il risultato per la probabilit� che quel particolare percorso venga scelto.
\begin{equation}
\begin{split}
	\wp(O|\lambda) & = \displaystyle \sum_{1\leq t \leq T, i \in \{\text{tutti Q}\}}\wp(O_t|Q_i, \lambda)\wp(Q_i|\lambda) \quad \text{dove}\\
	& \wp(O|Q_i, \lambda)  =  \displaystyle \prod_{1\leq t \leq T} \wp(O_t|q_{i,t}, \lambda) \\ 	 
	& = b_{q_i,1}(O_1)*\ldots*b_{q_i,T}(O_T)\quad \text{e}\\
  & \wp(Q_i|\lambda)  =  \pi_{q_1}*a_{q_1,q_2}*\ldots *a_{q_T-1,q_T} 
\end{split}
\label{eq:naiveEvaluation}
\end{equation}

l'algoritmo ha complessit� esponenziale, $O(N^T)$.\\

La soluzione ottima a questo problema � data dall'algoritmo di programmazione lineare noto come algoritmo \textit{Forward}. L'idea su cui si basa � di considerare dei percorsi parziali per rappresentare osservazioni parziali. Vengono definite le variabili $\alpha_{i,t}$ come la probabilit� di aver osservato la sequenza parziale $O_1, \ldots, O_t$ e di essere nello stato $S_i$ all' istante temporale $t$ 
\begin{equation}
\alpha_{i,t} = \wp(o_1, \ldots, o_t, q_T=S_i|\lambda)
\label{eq:alpha}
\end{equation}

Dalla definizione \eqref{eq:alpha} ricaviamo la matrice $\alpha (N \times T)$ nella quale vengono inserite le probabilit� parziali per ogni combinazione stato-tempo, di modo che al passo temporale successivo il calcolo possa essere basato su tali valori e non ricalcolando tutto dal primo istante di tempo. 
\begin{algorithm}    
\caption{\textit{Forward}}
\label{alg:fwd}                                                  
\begin{algorithmic}[1]
\STATE {}\COMMENT{1. Inizializzazione}                    
\FOR{$j = 1$ to N=|S|} 
\STATE {$\alpha_{j,1}=\pi_j b_j(o_1)$\hspace{1.5cm}}\COMMENT{ \emph{La probabilit� congiunta di partire dal $j$-esimo stato} } 
\ENDFOR \hspace{4cm}\COMMENT{ \emph{ed emettere il primo segnale dallo stesso}} 
\STATE {}\COMMENT {2. Induzione}
\FOR{$t = 1$ to $T-1$} 
	\FOR{$j = 1$ to $N$ }
		\STATE {$\alpha_{j, t+1}=[\displaystyle\sum_{i=1}^N\alpha_{i,t}a_{i,j}]b_j(o_{t+1})$\hspace{.5cm}}
		 \COMMENT{ \emph{La probabilit� congiunta di arrivare al} } 	
		 \ENDFOR	\hspace{4cm}\COMMENT{ \emph{$j$-esimo stato ed emettere il $t+1$-esimo segnale}}
\ENDFOR
\STATE{}\COMMENT{3. Terminazione}
\FOR{$j = 1$ to $N$} 
	\STATE {$\wp(O|\lambda)=\displaystyle\sum_{j=1}^N \alpha_{j,T} $}
\ENDFOR
\end{algorithmic}
\end{algorithm}

La complessit� dell'algoritmo \textit{Forward} � $O(NT)$. Il netto miglioramento � dovuto al fatto che i risultati parziali vengono riutilizzati, limitando il numero di calcoli da svolgere ad ogni istante temporale a $N$. La struttura grafica su cui si basa l'algoritmo \textit{Forward} � detta struttura a traliccio. \\
  


\subsubsection*{Decodifica} 
Il secondo problema � noto come problema di Decodifica in cui si cerca la sequenza di stati $Q = q_1,\ldots,q_T$ che ha generato una data sequenza di osservazioni $O = o_1,\ldots, o_T$. Dato che, al contrario del problema della Valutazione, non esiste un'unica soluzione al problema di Decodifica, quello che si fa � di stabilire un criterio di ottimalit� in funzione del quale fare la ricerca. \\
Un possibile criterio di ottimalit� della sequenza � quello di scegliere gli stati $q_t$ che all'istante di tempo $t$ sono i pi� probabili. Tale criterio massimizza il numero atteso di stati corretti individualmente. \\
Per risolvere il problema necessitiamo di due variabili $\beta$ e $\gamma$. La prima � definita come risultato dell' algoritmo \textit{Backward}, che computa il processo inverso dell'algoritmo \textit{Forward}.

\begin{equation}
\beta_{i,t}=\wp(o_{t+1}\ldots o_T|q_t = S_i,\lambda)
\label{eq:beta}
\end{equation}


\begin{algorithm}    
\caption{\textit{Backward}}
\label{alg:bckwd}                                                  
\begin{algorithmic}[1]                    
\STATE {}\COMMENT{1. Inizializzazione}                    
\FOR{$j = 1$ to N=|S|} 
	\STATE {$\beta_{j,T} = 1$}
\ENDFOR
\STATE {}\COMMENT {2. Induzione}
\FOR{$t = T-1$ to $1$} 
	\FOR{$i = 1$ to $N$ }
		\STATE $\beta_{i,t}=\displaystyle\sum_{j=1}^N a_{i,j}b_j(o_{t+1})\beta_{j,t+1}$
	\ENDFOR
\ENDFOR
\end{algorithmic}
\end{algorithm}

La seconda variabile, $\gamma$ � definita come la probabilit� di essere in uno stato $S_i$ al tempo $t$, data un'osservazione $O$ ed un modello $\lambda$.
\begin{equation}
\gamma_{i,t} = \wp(q_t = S_i| O, \lambda)
\label{eq:gamma}
\end{equation}
$\gamma$ pu� essere calcolata in funzione delle variabili di \textit{Forward} e \textit{Backward}:

\begin{equation}
\gamma_{i,t} = \dfrac{\alpha_{i,t}\beta_{i,t}}{\wp(O|\lambda)} = 
							 \dfrac{\alpha_{i,t}\beta_{i,t}}{\displaystyle\sum_{j = 1}^N \alpha_{j,t}\beta_{j,t}}
\label{eq:gammaFnAlphaBeta}
\end{equation}

$\alpha_{i,t}$, fornisce le probabilit� delle osservazioni parziali fino a $t$, mentre $\beta_{i,t}$, da $t+1$ in poi, essendo correntemente nello stato $S_i$. Il denominatore dell'equazione \eqref{eq:gammaFnAlphaBeta} � un fattore di normalizzazione che rende $\gamma$ un valore di probabilit�, quindi sar� valida la propriet�
\begin{equation}
\displaystyle\sum_{i = 1}^N \gamma_{i,t} = 1
\label{eq:propriet�Gamma}
\end{equation}
 Usando $\gamma$ � possibile trovare lo stato $q$ che individualmente � il pi� probabile al tempo $t$:
\begin{equation}
q_t = \arg\max_{0 \leq i \leq N} \gamma_{i,t}
\label{eq:statoPi�Probabile}
\end{equation}

Anche se \eqref{eq:statoPi�Probabile} massimizza il numero di stati pi� probabili scegliendo quelli che individualmente sono i pi� probabili, potrebbe creare problemi nel momento in cui si considera una sequenza di stati. Se la HMM ha anche transizioni nulle, la sequenza generata da \eqref{eq:gammaFnAlphaBeta} potrebbe essere non valida, perch� non vi � nessun controllo sulla probabilit� di co-occorrenza di stati. \\
Il criterio di ottimalit� pu� essere cambiato ad individuare la miglior sequenza di lunghezza $T$ che massimizzi la probabilit� di tutti gli stati nella sequenza. Per trovare il cammino migliore, viene usato un metodo di programmazione dinamica detto algoritmo di Viterbi.
	L'algoritmo di Viterbi individua la migliore sequenza di stati che rispondano a una data sequenza di osservazioni: \begin{equation}
\delta_{i,t} = \max_{q_1,\ldots q_{t-1}}\wp(q_1 \ldots q_t = S_i,o_1 \ldots o_t|\lambda)\\
\label{eq:delta}
\end{equation}

Nella matrice $\delta$ vengono inserite le probabilit�, mentre per tenere traccia del percorso migliore si usa una variabile $\psi$:
\begin{equation}
\psi_{i,t} = \arg\max_{q_1,\ldots q_{t-1}}\wp(q_1 \ldots q_t=i,o_1 \ldots o_t|\lambda)
\label{eq:psi}
\end{equation}

La procedura completa per trovare il percorso di stati pi� probabile � la seguente:
\begin{algorithm} 
\caption{Viterbi}
\label{alg:viterbi}                                                  
\begin{algorithmic}[1]                    
\STATE{}\COMMENT{Inizializzazione}
\FOR {$i = 0$ to $N$}
	\STATE {$\delta_{i,1} = \pi_i b_i(o_1)$}
	\STATE {$\psi_{i,1} = 0$}
\ENDFOR
\STATE{}\COMMENT{Iterazione}
\FOR{$t = 2$ to $T$}
	\FOR{$i = 1$ to $N$} 
		\STATE $ \delta_{i,t}=\displaystyle\max_{1\leq j \leq N} [\delta_{j,t-1}a_{j,i}]* b_i(o_i)$
		\STATE $ \psi_{i,t}=\displaystyle\arg \max_{1\leq j \leq N} [\delta_{j, t-1}a_{j,i}]$
	\ENDFOR
\ENDFOR
\STATE{}\COMMENT{Terminazione}
\STATE $P* = \displaystyle \max_{1\leq i \leq N} [\delta_{i,T}]$
\STATE $q_T* = \displaystyle \arg \max_{1\leq i \leq N} [\delta_{i,T}]$
\STATE{}\COMMENT{Backtracking}
\FOR{$t=T-1$ to $1$}
	\STATE{$q*_t=\psi_{q*_{t+1},t+1}$}
\ENDFOR
\end{algorithmic}[1]
\end{algorithm}
	
	
\subsection*{Algoritmo di Viterbi in Tempo Reale}	
Il problema dell'algoritmo di Viterbi in molte applicazioni reali, � la latenza. I sistemi che utilizzano l'algoritmo hanno spesso necessit� di avere risultati immediati, in tempo reale. Trovare la sequenza di stati pi� verosimile che ha generato una sequenza di osservazioni � (come mostrato dall'algoritmo \ref{alg:viterbi}) fattibile con l'algoritmo di Viterbi, tracciando percorsi nella sequenza temporale di stati ed una volta arrivato a termine (tempo finale $T$), ricostruendo a ritroso il cammino migliore. Un problema significativo dell'algoritmo � che assume che la sequenza temporale sia finita. Vi sono molti casi pratici in cui l'applicazione dell'algoritmo sarebbe utile, per i quali i dati sono un flusso continuo ed incessante di dati. Una possibile soluzione � l'applicazione del'algoritmo di Viterbi a finestre successive di dati, dopo la quale restituire solo una parte iniziale degli decodificati  \cite{synface_low_latency_viterbi}, \cite{automatic_bass_line_transcript}(perch� hanno una probabilit� maggiore di essere corretti). Questa soluzione, conduce a una decodifica sub-ottimale.\\
Un'altra soluzione consiste nel confrontare pi� percorsi su una finestra temporale in espansione, finch� le soluzioni non convergono. Nel lavoro \cite{real_time_viterbi_optimization_multi_target_tracking} per la localizzazione automatica di un soggetto via video, la finestra temporale viene dinamicamente ridimensionata in base a un'euristica che bilanci latenza e accuratezza. Questo tipo di approccio non garantisce la convergenza dei percorsi considerati. L'approccio che noi abbiamo usato nel lavoro � quello proposto da Bloit et al \cite{short_time_viterbi_online_hmm_deconding}. 

\begin{definition}[Cammino locale]
Viene detta cammino locale, la sequenza di stati $s(a,b,i)$ ottenuta applicando l'algoritmo di Viterbi alla finestra temporale dall'istante temporale $a$ all'istante $b$ (con $a < b$) e compiendo il backtracking da uno stato arbitrario $i$ al tempo $b$. 
\end{definition}

\begin{definition}[Punto di Fusione]
Si definisce punto di fusione,l'istante temporale $\tau < T$  t.c per $a \leq t \leq \tau$, tutti i cammini locali appartenente all'insieme dei cammini locali $CL =\{ s(a,b,i), \forall i \in S\}$ (dove $S$ � l'insieme degli stati dell'HMM), sono uguali.
\end{definition}
	
Un punto di fusione gode della seguente propriet�:
i cammini locai fino al punto di fusione (che per definizione sono tutti uguali) sono sempre uguali al cammino globale (quello ottenibile con l'algoritmo di Viterbi originale \ref{alg:viterbi})\footnote{per la dimostrazione consultare \cite{short_time_viterbi_online_hmm_deconding}}.

%\begin{algorithm} 
%\caption{ShortTimeViterbi}                                                  
%\label{alg:short_time_viterbi_decoding}   


\begin{algorithmic}
\STATE{$a = 0$, $b = 0$ }
%\FOR {ogni finestra temporale al tempo $b$}
%	\STATE calcola  $s_t(a,b,i) \forall I \in (a,b,\lambda)$
%	\IF{se viene trovato punto di fusione $\tau > a $}
%	\STATE risultato = $s*_{a,\tau}$
%	\STATE $a = \tau$
%\ENDFOR
\end{algorithmic}
%\label{alg:short_time_viterbi_decoding}   


	
\subsubsection* {Apprendimento}
L'ultimo problema � quello dell'Apprendimento che consiste nel configurare i parametri di $\lambda$ per massimizzare la probabilit� di una sequenza di dati osservati. Non � noto un metodo analitico per risolvere il problema, infatti, data una sequenza finita di osservazioni, non esiste un modo ottimo di stimare i parametri di $\lambda$. Possiamo per� scegliere $\lambda$ in modo da massimizzare localmente $\wp(O|\lambda)$, con un procedimenti iterativi come 
\begin{itemize}
	\item Baum-Welch
	\item Expectation-Modification
	\item Tecniche basate sul gradiente
\end{itemize}
La procedura iterativa � detta riestimazione, e consiste in un miglioramento graduale (in base ad un criterio) ed un aggiornamento dei parametri. Per descrivere le riestimazione, � necessario definire la variabile $\xi$ come la probabilit� di essere in un certo stato in un istante di tempo ed essere in un altro nel successivo istante di tempo:
\begin{equation}
\begin{split}
\xi_t(i,j) &= \wp(q_t = S_i, q_{t + 1} = S_j|O,\lambda)\\
					 &= \displaystyle \dfrac{\alpha_{i,t} a_{i,j}b_i(o_t)\beta_{j, t+1}}{\wp(O|\lambda)}=
\displaystyle \dfrac{\alpha_{i,t} a_{i,j}b_i(o_t)\beta_{j, t+1}}{\displaystyle\sum_{i=1}^{N}\sum_{j=1}^{N}\alpha_{i,t} a_{i,j}b_i(o_t)\beta_{j, t+1}}
\label{eq:xi}
\end{split}
\end{equation}

Possiamo correlare $\gamma$ e $\xi$ nel seguente modo
\begin{equation}
\gamma_{i,t} = \displaystyle\sum_{j=1}^N \xi_t(i,j)
\label{eq:gammaxi}
\end{equation}

\begin{definition}{Numero di transizioni attese}\\
\begin{itemize}
	\item Numero di transizioni attese da $S_i$
		\begin{equation}
			\displaystyle \sum_{t=1}^{T-1} \gamma_{i,t}
			\label{eq:numTransizAtteseDa}
		\end{equation}
	\item Numero di transizioni attese da $S_i$ a $S_j$
		\begin{equation}
			\displaystyle \sum_{t=1}^{T-1} \xi_t(i,j)
			\label{eq:numTransizAtteseDaA}
		\end{equation}
	\item $\overline{\pi}_i$ = numero di volte nello stato $S_i$ a tempo $t = 1$ atteso 
			\begin{equation}
					= \gamma_{i,1}
			\label{eq:piGamma}
			\end{equation}
	\item	$\overline{a}_{i,j} =\displaystyle \dfrac{\text{numero di transizioni attese dallo stato} S_i \text{allo stato} S_j}
																								 {\text{numero di transizioni attese dallo stato} S_i}$
			\begin{equation}
					= \displaystyle\dfrac{\displaystyle\sum_{t=1}^{T-1}\xi_t(i,j)}
														  {\displaystyle\sum_{t=1}^{T-1}\gamma_{i,t}}
			\label{eq:transitionMtxXi}
			\end{equation}																								 
																							
	\item $\overline{b}_j(k) =\displaystyle \dfrac{\text{numero di volte nello stato} S_j \text{osservando} v_k}
																								 {\text{numero di volte atteso nello stato} S_i}$
			\begin{equation}
					= \displaystyle\dfrac{\displaystyle\sum_{t=1	}^{T}\gamma_{j,t}\quad \text{t.c. $o_t = v_k$}}
														  {\displaystyle\sum_{t=1}^{T}\gamma_{j,t}}  
			\label{eq:emissionMtxGamma}
			\end{equation}				 
\end{itemize}
\end{definition}

Dato il modello $\lambda = (A,B,\pi)$ indico il modello riestimato con $\overline{\lambda} = (\overline{A},\overline{B},\overline{\pi})$. L.Baum et al\cite{maximization_technique_stat_analysis_fun_markov_ch} hanno dimostrato che nel processo di riestimazione � vera una delle seguenti alternative:
\begin{enumerate}
	\item $\lambda = \overline{\lambda}$
	\item $ \wp(O|\overline{\lambda}) > \wp(O|\lambda)$
\end{enumerate}

Nel secondo caso � possibile sostituire $\overline{\lambda}$ a $\lambda$, ripetendo l'iterazione, finch� non si verifica una condizione d'arresto, ed il risultato della procedura � detto stima di massiva verosimiglianza (\textit{maximum likelihood estimate}) dell'HMM
\begin{algorithm}    
\caption{Baum-Welsh o \textit{Maximum Likelihood Expectation}.  }
\label{alg:baum_welsh}                                                  
\begin{algorithmic}[1]                   
\REQUIRE{maxlikelihood($\lambda = A,B,\pi$)}
\REPEAT 
	\STATE{\fbox{$\pi_i=\gamma_1(i)$}
		\fbox{{$a_{ij}=\dfrac{\displaystyle\sum_{t=1}^{T-1}\xi_t(i,j)}{\displaystyle\sum_{t=1}^{T-1}\gamma_t(i)}$}}
		\fbox{$b_j(k)=\dfrac{\displaystyle\sum_{t=1, O_t=V_k}^{T}\gamma_t(j)}{\displaystyle\sum_{t=1}^{T}\gamma_t(j)}$}}
	\IF{$\lambda=\tilde{\lambda}$}
		\RETURN{$\lambda$}
	\ELSIF{$\wp(O|\tilde{\lambda})>\wp(O|\lambda)$}
		\STATE maxlikelihood$(\tilde{\lambda})$
	\ENDIF
\UNTIL{\COMMENT{una condizione limite}}
\end{algorithmic}
\end{algorithm}



\section*{Problemi di implementazione di HMM}
Vi sono diversi problemi che si devono affrontare per implementare le HMM ed i vari algoritmi sinora descritti. Alcuni di questi sono 
\begin{itemize}
	\item Ridimensionamento (\textit{scaling}): la procedura di riestimazione, comporta una lunga sequenza di prodotti di valori di probabilit�. Ci� fa in modo che i valori tendano esponenzialmente a zero, quindi vi � un inevitabile problema di underflow. L'unico modo di ovviare al problema � quello di ridimensionare i valori, moltiplicandoli per un coefficiente che non dipenda dallo stato, ma solo dal tempo. Alla fine del processo, i coefficienti di ridimensionamento vengono eliminati.  \\
	Ad esempio nella procedura di riestimazione si calcola la matrice di transizione con l'equazione \eqref{eq:transitionMtxXi}
		\begin{equation}
				\overline{a_{i,j}}= \dfrac{\displaystyle\sum_{t=1}^{T-1}\alpha_{i,t}a_{i,j}b_j(o_{t+1})\beta_{j,t+1}}													{\displaystyle\sum_{t=1}^{T-1}\displaystyle\sum_{j=1}^{N}\alpha_{i,t}a_{i,j}b_j(o_{t+1})\beta_{j,t+1}}
		\label{eq:transitionMtxAlphaBeta}
		\end{equation}
		
	Usando un coefficiente di ridimensionamento � $c_t = $
	\begin{equation}
			\frac{1}{\displaystyle\sum_{j=1}^{N}\alpha_{i,t}}
	\label{eq:scaleFactorOForAlpha}
	\end{equation}
	si ottiene un $\alpha_{j,t}$ scalato 
	\begin{equation}
		\hat{\alpha}_{j,t} = \dfrac{\displaystyle\sum_{j=1}^N \hat{\alpha}_{j,t-1}a_{i,j}b_j(o_t)}
		{\displaystyle\sum_{i=1}^{N}\displaystyle\sum_{j=1}^{N}\hat{\alpha}_{i,t-1}a_{i,j}b_j(o_{t})}
	\label{eq:scaledAlpha}
	\end{equation}
	Nel momento in cui vengono calcolati i $\beta_{j,t}$, vengono eliminati i fattori di ridimensionamento:
	\begin{equation}
			\hat{\beta}_{j,t} = c_t\beta_{j,t}
	\label{eq:scaledBeta}
	\end{equation}
	
	La modifica pi� importante deve essere applicata all'algoritmo \textit{Forward}, perch� si � interessati al valore di probabilit�. In questo caso non � possibile semplicemente sommare $\hat{\alpha_{j,t}}$, perch� sono valori scalati e privi di significato se presi singolarmente. In questo caso viene usata la seguente propriet�:
	\begin{equation}
		\begin{split}
			\prod_{t=1}^T c_t \displaystyle \sum_{i=1}^N \alpha_{i,T}& = 1\\
			\prod_{t=1}^T c_t \wp(O|\lambda) &= 1 \Leftrightarrow \wp (O|\lambda) = \frac{1}{\prod_{t=1}^T c_t}
		\end{split}
	\label{eq:propriet�Alpha}
	\end{equation}
	Qui introduciamo il logaritmo della probabilit�, in modo che il valore sia calcolabile su un computer
	\begin{equation}
		\log (\wp(O|\lambda)) = -\displaystyle\sum_{t=1}^T \log c_t
	\label{eq:logLikelihood}
	\end{equation}
	\item Molteplici sequenze di osservazione. Nelle HMM Sinistra-Destra, non si possono usare singole sequenze di osservazioni per addestrare il modello, perch� questo tipo di modello tende a uscire molto facilmente da uno stato, quindi ad ogni stato corrispondono pochissime osservazioni. Ci� implica che per una quantit� di dati sufficiente a fare una stima affidabile dei parametri del modello si devono usare pi� sequenze di osservazioni.  
	\item Stime dei parametri iniziali. Un problema irrisolto � come scegliere i valori dei parametri iniziali in modo tale che i massimi locali corrispondano al massimo globale della funzione di verosimiglianza (likelihood). Normalmente quello che si fa � scegliere valori casuali oppure uniformi per poi iniziare la procedura di riestimazione. Invece per quanto riguarda i parametri $B$ � necessaria una buona stima iniziale, che solitamente viene fatta con un processo di segmentazione e media delle osservazioni in stati. 
	\item Insufficienza di dati. Un problema frequente � che il numero di osservazioni � troppo basso per consentire di avere una stima abbastanza buona dei parametri del modello. Una possibile soluzione � quella di aumentare il numero di dati, ma ci� � spesso impraticabile. L'approccio inverso � quello di ridurre il numero di parametri, come ad esempio il numero di stati. Ci� � in teoria sempre praticabile, ma spesso poco sensato, in quanto vi sono delle motivazioni valide per avere quei parametri. Una terza alternativa � quella di interpolare tra un insieme di stime di parametri con un'altro da un modello per il quale si ha un numero sufficiente di dati di addestramento. L'idea � quella di progettare insieme al modello, anche una versione ridotta dello stesso, per il quale il numero di dati in possesso sia sufficiente. Date le stime per i parametri del modello $\lambda = (A,B,\pi)$ come per la versione ridotta $\lambda' = (A',B',\pi')$ il modello interpolata � ottenuto come
	\begin{equation}
		\tilde{\lambda} = \epsilon \lambda + (1- \epsilon)\lambda'
	\label{eq:modelInterpolation}
	\end{equation}
	dove $\epsilon$ rappresenta un peso che viene associato ai parametri del modello, ed $(1- \epsilon)$ il peso associato a quelli del modello ridotto. Il valore di $\epsilon$ viene determinato in funzione dei dati di addestramento. Mercer et al \cite{interpolation_markov_params_sparse_data} hanno dimostrato che � possibile stimare l'$\epsilon$ ottimo mediante l'algoritmo \textit{Forward}-\textit{Backward}, espandendo la HMM a partire da \eqref{eq:modelInterpolation}.
	\item Scelta della dimensione e tipo del modello. Si tratta della scelta dei parametri che si deve fare all'inizio per rappresentare al meglio il problema con le HMM. Il tipo di HMM, Ergodico o Sinistra-Destra, la dimensione del modello  cio� numero di stati, l'alfabeto di osservazione, discreti o continui, a distribuzione singola o a misture di distribuzioni. Non vi � un metodo standard, o migliore di prendere queste decisioni, ma devono essere fatte in base al tipo di segnale che si sta modellando.
\end{itemize}

\chapter[Tempo Reale, In Linea, Latenza]{Sistemi in Tempo Reale, algoritmi in Linea ed il problema della latenza e accuratezza}
\label{sec:real_time_sys}
La seguente sezione fornisce un chiarimento sul concetto di tempo reale, che spesso viene usato, anche in letteratura, in modo errato per intendere in linea. L'algoritmo di segmentazione usato in questo lavoro � un algoritmo in linea, non in tempo reale.
\section*{Tempo Reale}
La nozione di Tempo Reale (\textit{Real Time}, RT) si contrappone ad una di tempo logico o virtuale, in quanto misura di una quantit� fisica. La seconda forma di tempo, quella logica, � una misura di tipo qualitativo e rappresenta l'ordine di eventi.\\

In diversi ambiti, la nozione di sistema RT assume significati diversi:

\par\textsc{Informatica}: si parla di Computazione RT (\textit{Real Time Computing} - RTC) o computazione Reattiva: lo studio di sistemi software e hardware soggetti a vincoli di tempo reale. Un esempio di tale sistema sono i sistemi operativi RT (un esempio � LynxOS), che garantiscono tempi di risposta ben definiti, a differenza dei sistemi operativi non RT (anche se solitamente hanno tempi di risposta brevi).\\
Un altro esempio sono i linguaggi di programmazione Sincroni come ChucK, che � un linguaggio concorrente per l'elaborazione di file audio.
\par\textsc{Simulazioni}: RT si riferisce ad una sincronizzazione con le tempistiche reali, ovvero gli eventi nel processo simulato devono avvenire allo stesso tempo degli eventi nel processo reale. Un esempio sono i video giochi.
\par\textsc{Trasferimento di dati ed elaborazione di media}: RT assume un significato pi� soggettivo che riflette la percezione dell'utente finale, significa senza un ritardo percettibile dall'utente.



Sistemi RT possono essere classificati in base alla conseguenza di un ritardo nei tempi di risposta.
	\par\textsc{Sistemi Hard RT}: Un ritardo pu� avere delle conseguenze catastrofiche, ad esempio un sistema di pilotaggio.
	\par\textsc{Sistemi Soft RT}: Un ritardo non ha conseguenze sulla vita o di tipo economico, ad esempio un sistema per la visualizzazione di file video.

Per garantire che tali scadenze vengano rispettate, deve essere noto il tempo di esecuzione massima dei singoli processi 
di un programma. Questo problema � molto complesso e spesso si ottengono solo soluzioni parziali.\\

\section*{Algoritmi in linea}
\label{sec:online}
Un algoritmo � detto in linea (\textit{online}), se � in grado di dare un risultato a partire da un sottoinsieme di dati in ingresso in un determinato ordine. Un algoritmo fuori linea (\textit{offline}) invece, deve avere tutti i dati inizialmente per poter fornire un risultato.\\ Esempi dei due tipi di algoritmi sono l'algoritmo di ordinamento a inserzione (\textit{Insertion Sort}) che ha bisogno di 1 numero in pi� alla volta per poter fornire dopo n esecuzioni una lista ordinata, mentre l'algoritmo di ordinamento per selezione (\textit{Selection Sort}) ha bisogno dell'intera lista di numeri per poter cominciare a ordinare. \\

Dato che un algoritmo in linea prende decisioni basate solo su parte dei dati di cui necessiterebbe la risoluzione del problema in questione, le decisioni prese possono risultare non ottimali. Uno degli obbiettivi dello studio degli algoritmi in linea � di valutare la qualit� delle decisioni possibili in tali circostanze. 
Il metodo che viene utilizzato per formalizzare questa idea � noto come Analisi Competitiva: vengono confrontate le prestazioni relative di un algoritmo in linea e fuori linea ottimale sulla stessa istanza di un problema.\\

\section*{Latenza}
Il concetto di latenza nell'ambito informatico/ingegneristico assume svariati significati. Generalmente fa riferimento ad un ritardo rispetto ad un tempo atteso in un sistema. 
Un significato di latenza che si vuole menzionare in questo lavoro � quella dell'ambito biomedico: la latenza di un sistema di misurazione clinica dal punto di vista di un fisiatra � il ritardo sul tempo di risposta positiva. Nel caso di un segmentatore di deambulazione, la latenza � la differenza tra il tempo in cui si verifica un evento ed il tempo in cui il sistema annuncia l'avvenimento dell'evento. Ci� significa che i risultati errati aumentano la latenza. Quindi un sistema ha una latenza bassa, non solo se ha tempi di risposta brevi, ma ha anche un alta percentuale di risultati corretti. 
%\chapter{Cenni di Meccanica classica}
\label{meccanica_classica}
%\myChapter{Cenni di Meccanica classica}
\section{Cinematica}
La Cinematica\cite{Douglas_classical_mechanics} � lo studio del moto di corpi materiali, senza considerare le cause e conseguenze di tale moto. La parte della Meccanica che si occupa delle cause e conseguenze del moto dei corpi � la Dinamica o Cinetica. La Cinematica fornisce una descrizione geometrica dei possibili moti.\\
Il soggetto mediante il quale vengono condotti gli studi in Cinematica � la particella, vale a dire un corpo immaginario, che occupa un singolo punto dello spazio. Un insieme di particelle le cui distanze relative rimangono invariate in ogni riferimento spaziotemporale. 

\subsection{Moto rettilineo di una particella}


Le quantit� principali di cui si occupa sono
\subsubsection{Posizione} Il vettore posizione della particella $P$ nel punto $A$: $\mathbf{r} = (x_A,y_A,z_A)$ con magnitudine $|\mathbf{r}| = \sqrt{x_A^2 + y_A^2 + z_A^2}$ $(m)$.

\subsubsection{Spostamento} Lo spostamento da $A$ a $B$ di $P$: $\mathbf{r}_{AB} = \mathbf{r}_{B} - \mathbf{r}_{A}$ $(m)$

\subsubsection{Distanza} La distanza $\displaystyle s=\int_{t_1}^{t_2}{\sqrt{\Big(\dfrac{dx}{dt}\Big)^2 + \Big(\dfrac{dy}{dt}\Big)^2+ \Big(\dfrac{dz}{dt}\Big)^2}dt}$ $(m)$ dove $t$ � il tempo. 


\subsubsection{Velocit�} La \textbf{velocit�}
	\begin{itemize}
		\item \textbf{media} $\mathbf{\bar{v}} = \dfrac{\Delta \mathbf{r}}{\Delta t}$ $(m/s)$ con $\Delta t > 0$ 
		\item \textbf{istantanea} $\mathbf{v} = \displaystyle \lim_{\Delta t \rightarrow 0} \dfrac{\Delta \mathbf{r}}{\Delta t}$ con magnitudine $|\mathbf{v}| = \dfrac{ds}{dt}$ $(m/s)$
	\end{itemize}
	
\begin{figure}[h]
	\centering
		\includegraphics[width=.80\textwidth]{imgs/CinematicaLineare.jpg}
	\caption{Moto Rettilineo di una particella}
	\label{fig:CinematicaLineare}
\end{figure}
	
\subsubsection{Accelerazione}
L'accelerazione
	\begin{itemize}
		\item \textbf{media} $\mathbf{\bar{a}} = \dfrac{\Delta \mathbf{v}}{\Delta t}$ con $\Delta t > 0$ $(m/s^2)$
		\item \textbf{istantanea} $\mathbf{a} = \displaystyle \lim_{\Delta t \rightarrow 0} \dfrac{\Delta \mathbf{v}}{\Delta t}$ $(m/s^2) $
	\end{itemize}
	
	

\section{Moto Angolare di una particella}
\subsubsection{Posizione} Il vettore posizione della particella $P$ nel punto $A$ rispetto ad un asse di rotazione $O-z$ �  $\mathbf{r}(t)$. La posizione angolare del punto $P$ � $\mathbf{r}_{\perp}(t) = r_{\perp x} \cos \theta i + r_{\perp x} \sin \theta j $ $(rad)$


\subsubsection{Velocit�} 
La velocit� angolare � data da: 
		$\omega = \dfrac{d\theta}{dt}$ $(rad/s)$
	\begin{figure}[h]
	\centering
		\includegraphics[width=.80\textwidth]{imgs/CinematicaAngolare.jpg}
	\caption{Moto Angolare di una particella}
	\label{fig:CinematicaAngolare}
\end{figure}
	
\subsubsection{Accelerazione}
L'accelerazione angolare � data da:
	$\alpha = \dfrac{d\omega}{dt}$ $(rad/s^2)$
	
	
\section{Dinamica (Cinetica)}
Branca della meccanica che si occupa di forze che producono, arrestano o modificano il moto di corpi. 
Le due leggi fondamentali della Dinamica sono quelle di Newton, in particolare la seconda:
\begin{equation}
F = ma
\label{eq:secondoPrincipioNewton}
\end{equation}


\chapter{Sui sensori}
%\myChapter{Sensori}
\label{sec:sensori}
La seguente appendice ha l'obbiettivo di fornire chiarimenti ulteriori ai sensori valutati ed usti nel lavoro qui presentato.  

Principalmente in questo lavoro usiamo un giroscopio monoassiale. In letteratura, l'analisi della deambulazione viene 
affrontata mediante accelerometri, giroscopi e magnetometri. 
\section*{Accelerometro}
\label{sec:accelerometer}
Un accelerometro (vedi Figura \ref{fig:acc}) � un dispositivo elettromeccanico che misura le forze di accelerazione. Tali forze possono essere sia statiche, come la forza di gravit�, che dinamiche, causate muovendo l'accelerometro.\\
Gli usi immediati di un accelerometro sono 
\begin{itemize}
	\item misurando l'accelerazione statica delle forza di gravit�, si pu� calcolare l'angolo a cui � inclinato lo strumento.
	\item misurando l'accelerazione dinamica si pu� analizzare il modo in cui si sta muovendo il dispositivo  
\end{itemize}
Un applicazione industriale importante � l'airbag nelle macchine, la cui apertura scatta se l'accelerometro percepisce una brusca frenata.
Un'altra applicazione � quella implementata da Apple\tm nei suoi portatili per la protezione del disco rigido: se il portatile dovesse cadere mentre � acceso, l'accelerometro capta la caduta libera ed il sistema operativo viene terminato immediatamente in modo che la testina non sia sul disco. 

\begin{figure}
	\centering
	\includegraphics[width=1\textwidth]{imgs/acc_technical.jpg}
	\caption{ Accelerometro triassiale, cortesia di \url{http://www.dimensionengineering.com}}
	\label{fig:acc}
\end{figure}

\section*{Giroscopio}
\label{sec:gyroscope}
Il giroscopio � uno strumento per misurare l'accelerazione di rotazione (momento angolare) di un corpo. Vi sono diversi tipi di giroscopi, meccanici, a vibrazione, a fibre ottiche ecc..
Un disco rotante in assenza di torsione esterna, mantiene la direzione della sua rotazione 
 Quando viene applicata una torsione viene applicata al disco, ad angolo con il suo asse di rotazione, il disco ruota sul piano determinato dalle due assi (rotazione iniziale e torsione) nella direzione che va dall'asse di rotazione iniziale a quello della torsione.\\
Il tipo di giroscopio che usiamo in questo lavoro � il cosiddetto giroscopio piezoelettrico, o MEMS (\textit{Micro Electro Mechanical System}) o a vibrazione. Si basa sul principio di Coriolis: un oggetto che vibra, continua a vibrare sullo stesso piano se la struttura che lo sostiene � in rotazione. 
La misurazione della velocit� angolare avviene nel seguente modo: un elemento piezoelettrico (oggetto di forma tubolare) oscilla a causa di una rotazione, quindi viene misurata la forza di Coriolis sulla sezione longitudinale dell'elemento, dopo essere stata convertita in un voltaggio elettrico dallo stesso elemento piezoelettrico. 
 
\section*{Magnetometro}
\label{sec:magnetomerter}
Il Magnetometro � uno strumento che misura il campo magnetico. Questo pu� essere fatto in diversi modi. Il metodo pre elettronico, � quello inventato da Coulomb ed usa un ago magnetico sospeso.\\
Il metodo elettronico chiamato elettromagnetometro o Magnetometro \textit{Fluxgate} � basato sulla saturazione di materiali magnetici. Questi ha un centro in ferro, ed intorno ad esso due fili conduttori. Attraverso il primo filo fluisce corrente elettrica. Il ferro � un elemento magnetico, ma in condizioni normali gli assi magnetici sono orientati in direzioni casuali e la forza magnetica totale � prossima a zero. Nel momento in cui comincia a fluire corrente nel filo, gli assi si allineano e creano un campo magnetico percepibile come aumento del campo magnetico creato dalla corrente nel filo. La quantit� di forza magnetica che pu� produrre il ferro � limitata, il ferro giunge ad un livello di saturazione, dopo di che cambia bruscamente polarit�, al che giunge alla saturazione e cambia polarit� e cos� via. Questo processo induce corrente nel secondo filo che avvolge il ferro. Se la procedura avvenisse in un ambiente magneticamente neutrale il voltaggio nei due file dovrebbe combaciare, altrimenti vi sar� un dislivello proporzionale al campo magnetico di disturbo. L'intensit� del campo magnetico terrestre superficiale � circa 50,000 nano Tesla.  
\section*{IMU}
\label{sec:imu}
L'Unit� di Misura Inerziale (\textit{Inertial Measurement Unit}), � l'integrazione di pi� sensori. Questo fornisce le misure fatte dai sensori interni con eventuali correzioni sugli errori sistematici causati dalla temperatura interna dello strumento, umidit� ecc. Le IMU sono usate come sistemi di navigazione inerziali di aerei, missili. \\
La IMU che � stata usata nel lavoro presentato ha la seguente scheda di definizione: 

\begin{table}[htbp]
	\centering
\begin{tabular}{|m{4.2cm}|m{2.8cm}|m{5.3cm}|}
\hline
\multicolumn{3}{|c|}{IMU} \\ 
\hline
\textbf{Sensore}& \textbf{Intervallo di misurazione} & \textbf{Risoluzione} \\
\hline
\hline
\multirow{3}{*}{Accelerometro triassiale}& $x[\pm1\,g-3\,g]$ & \multirow{3}{*}{$[12-14\, bit]$}\\
&$y[1.5\,g-2\,g/8\,g]$&\\
&$z[2\,g-16\,g]$&\\
\hline
Giroscopio triassiale& $[\pm2000-1600\,^\circ/sec]$ & $[12-16\,bit]$\\
\hline
Magnetometro triassiale& $[\pm4\,gauss]$ & $12\,bit$\\
\hline
Termometro & $[-55-155^\circ/C]$& $12\,bit$\\
\hline
\multicolumn{2}{|l|}{\textbf{Connettivit�}} & \textit{Bluetooth} per le brevi distanze\\
%\multicolumn{3}{|r|}{Bluetooth per le brevi distanze}\\
\hline
\multicolumn{2}{|l|}{\textbf{Frequenza di campionamento}}&$\geq 300\,Hz$\\
%\multicolumn{1}{|r|}{$\geq$ 300 Hz}\\
\hline
\multicolumn{2}{|l|}{\textbf{Dimensioni strumento}}&$60 \times 30 \times 40\, mm$\\		
\hline
%\multicolumn{1}{|r|}{60 $\times$ 30 $\times$ 40 mm} 	
\end{tabular}
	\caption{Scheda tecnica IMU}
	\label{tab:SchedaTecnicaIMU}
\end{table}



	
\chapter{Su Android}
%\myChapter{Android}
\label{app:android}
Android � una piattaforma completa\footnote{Comprensiva di tutto il software necessario per un dispositivo mobile}  totalmente open source\footnote{L'intero stack di Android, vale a dire i moduli Linux del sistema operativo, le librerie native, il framework e gli applicativi, � completamente gratuito e modificabile. Viene distribuito sotto licenze business-friendly (Apache/MIT), in modo che chiunque possa estenderlo, modificarlo ed usarlo liberamente.} progettata per dispositivi mobili. Android � di propriet� della societ� Open Handset Alliance, con Google come maggiore azionario. L'obbiettivo di Google � accelerare lo sviluppo della tecnologia mobile ed offrire all'utente un'esperienza sempre pi� ricca ed allo stesso tempo meno costosa. 
Android � pensato per essere pronto all'uso dal punto di vista di tutti i possibili attori:

\begin{itemize}
	\item \textbf{Utenti}: I dispositivi hanno una configurazione di default che permette un funzionamento immediato e performante ma che pu� in un secondo momento essere profondamente riconfigurato su misura.
	\item \textbf{Sviluppatori}: Uno sviluppatore ha bisogno soltanto dell' Kit di sviluppo di Android (Android SDK\footnote{Software Developement Kit}), che comprende anche un emulatore, ma permette anche di sviluppare su un vero dispositivo. Uno sviluppatore ha accesso al codice dell'intera piattaforma Android.
	\item \textbf{Manufattori}: Android � portabile\footnote{Android non fa nessun tipo di assunzione sul tipo di dispositivo su coi verr� montato.} e (eccetto alcuni driver per specifici hardware) permette di far funzionare dispositivi immediatamente. I venditori non sono tenuti a rendere disponibile alla comunit� le proprie aggiunte. In molti casi dispositivi Bluetooth e Wi-Fi, sono gestiti da codice proprietario. Ma dato che lo sviluppo di codice viene regolato da una API (Application Programming Interface ovvero un'Interfaccia di Programmazione di un'Applicazione), il problema � facilmente gestibile. 
\end{itemize}

Android � ottimizzato per dispositivi mobili, che ovviamente hanno il requisito fondamentale della dimensione ridotta. Gli obbiettivi dei progettisti del sistema erano la massimizzazione della durata della batteria, ottimizzazione della memoria, ottimizzazione delle risorse computazionali. 

\section{Android OS}
\begin{figure}
	\centering
		\includegraphics[width=1.5\textwidth, angle=90]{imgs/android_system_architecture.jpg}
	\caption{Architettura di sistema di Android, cortesia di \url{http://developer.android.com}}
	\label{fig:android_system_architecture}
\end{figure}

\subsection{Linux Kernel}
Il sistema operativo Android \cite{android_dev} si basa sulla versione 2.6 di Linux \cite{linux_kernel} per i servizi centrali di sistema come la sicurezza, la gestione della memoria e dei processi, lo stack di rete ed i modelli dei driver (vedi figura \ref{fig:android_system_architecture}). Il Kernel funziona anche da livello di astrazione tra l'hardware e lo stack di software.
Tutte le applicazioni Android vengono eseguite in processi Linux separati, dopo aver avuto i premessi richiesti dal sistema Linux

\subsection{Librerie Native Android}
\label{sec:lib_Android}
Le librerie di Android sono principalmente composte da librerie C/C++ della comunit� open source. 
Queste librerie vengono esposte sotto forma di servizi di sistema per i programmatori che vogliano usarli come funzioni senza conoscerne i dettagli implementativi a livello di application framework (vedi figura \ref{fig:android_system_architecture}).
Le librerie principali sono: 
\begin{itemize}
	\item \textbf{Librerie Standard di (ANSI) C}: un implementazione BSD\footnote{Berkley Software Distribution e licenza Apache/MIT che a differenza della licenza GNU non obbliga sviluppatori a ridistribuire i loro codici alla comunit�} della libreria Standard di C (\textit{libc}), ottimizzata per dispositivi basati sul sistema Linux. Alcuni esempi di servizi della libreria sono l'allocazione di memoria, la gestione dell'input/output ecc.
	\item \textbf{Librerie Media}: basate sulle OpenCORE \cite{packetVideo_openCore} di PacketVideo\footnote{PacketVideo � il membro fondatore Open Handset Alliance}, versione open source della libreria CORE\tm  della stessa compagnia. Queste librerie supportano la visualizzazione (playback) e la registrazione  dei formati audio e video ed immagini statiche pi� popolari (MPEG4, H.264, MP3, AAC, AMR, JPG, e PNG).
	\item \textbf{Surface Manager}: gestisce l'accesso al sottosistema di visualizzazione e compone, in modo trasparente all'utente, la grafica 2D con quella 3D di applicazioni multiple.
	\item \textbf{LibWebCore}: un motore per un web browser, che pu� essere usato sia dal browser di Android che da una vista del web incorporata in un applicativo. LibWeb \cite{LibWeb} � una libreria/toolkit per sviluppare applicazioni Web scritte in Perl.
	\item \textbf{SGL}: motore grafico 2D.
	\item \textbf{Librerie 3D}: un'implementazione basata sulle API di OpenGL\footnote{Open Graphics Library} \cite{OpenGL}. Le librerie usano l'acceleratore grafico 3D, dove disponibile, e il rasterizzatore\footnote{Trasformatore di un oggetto grafico dalla sua descrizione vettoriale in una descrizione visuale, vale a dire pixel o punti che possano essere visualizzati su uno schermo o stampati.} altamente ottimizzato per programmi 3D incluso nella distribuzione. OpenGL � un ambiente per sviluppare grafica sia 2D che 3D, interattiva e portabile.
	\item \textbf{FreeType} \cite{FreeType}: rendering di font con tecnologia bitmap e vettoriale 
	\item \textbf{SQLite} un motore per un database relazionale potente e leggero a disposizione di tutte le applicazioni.
	\item \textbf{OpenSSL} \cite{OpenSSL}:  � un insieme di strumenti Open Source che implementano il Secure Sockets Layer (SSL v2/v3) ed i protocolli Transport Layer Security (TLS v1) ed infine una libreria di crittografia generica di ottimo livello.
\end{itemize}



\subsection{Tempo di esecuzione di Android: Dalvik}
\label{sec:dalvik}
Il linguaggio Java \cite{Java}, JDK\footnote{Java Developement Kit} Tools \cite{JDK_Tools} e le librerie Java sono gratuite, mentre la Java Virtual Machine non lo �. Dato che questo andava contro la politica del progetto,  Google\footnote{Dan Bernstein ed il team di sviluppo} ha sviluppato una versione ex-novo della Java Virtual Machine, ad-hoc per Android\footnote{Fino al 2005, non vi erano alternative alla JVM della Sun, poi sono nate OpenJDK              \cite{OpenJDK} e Apache Harmony \cite{Apache_Harmony}}. I problemi principali che il gruppo di sviluppo hanno affrontato sono quelli della durata della batteria e le risorse (memoria e ram) limitate.
\subsubsection{Java e Android}
Normalmente il codice Java viene compilato e poi il bytecode viene eseguito sulla JVM, sotto Android il bytecode viene ricompilato con il compilatore Dalvik (vedi sezione \ref{sec:dalvik}) che produce un Dalvik-bytecode detto Dex, che viene eseguito dal Dalvik VM (vedi figura \ref{fig:Java_vs_Android_compile_exec}).

\begin{figure}
	\centering
	\includegraphics[width=.9\textwidth]{imgs/JVMvsDalvik.jpg}
	\caption{Comparazione del processo di compilazione di un file Java in ambiente Android con quello classico. Immagine cortesia di \cite{Gargenta_android}}
	\label{fig:Java_vs_Android_compile_exec}
\end{figure}

Il processo � automatizzato dall'IDE\footnote{I Developement Environment} (Eclipse o ANT \cite{Apache_Ant}) che si usa.
La distribuzione Java di Android non � standard: � una variante di Java Standard Edition, in cui le Java AWT e Swing sono state sostituite da Android UI\footnote{User Interface}, appositamente ottimizzate per gli schermi e le schede grafiche di dimensioni ridotte dei dispositivi.
\subsection{Application Framework}
Questa � la parte della piattaforma che permette di sviluppare applicativi Android, fornendo servizi (sensori, posizionamento, telefonia, Wifi ecc) ed abbondante documentazione in merito. 
\subsection{Applications}
Le applicazioni sono i programmi sviluppati dal mondo di sviluppatori Android. Questi possono essere sia gi� istallati all'acquisto del dispositivo, sia scaricati dai mercati Android. 
\subsubsection{APK}
Un applicazione Android � un singolo file, detto APK file. Questi ha tre componenti principali:
\begin{enumerate}
	\item \textbf{Eseguibile Dalvik} Il codice Java compilato come descritto in figura (vedi figura \ref{fig:Java_vs_Android_compile_exec}).
	\item \textbf{Risorse} Tutto cio che � in un applicativo Android, ma non � codice Java: file XML, immagini, audio, video ecc.
	\item \textbf{Librerie} In un applicativo possono essere incluse librerie di codice nativo, ad esempio in C/C++
\end{enumerate} 
\subsection{Struttura di un Android App}
\label{sec:android_app_structure}
Ogni applicativo per Android deve una determinata struttura di cartelle e file per funzionare. Il file pi� importante � l'AndroidManifest. Questo file funziona da collante e da indice per comprendere le componenti dell'applicativo. Contiene i permessi che ha come applicativo, di interagire con il resto del sistema operativo. \\
Lavorando in ambiente di sviluppo Eclipse SDK \tm, con il plugin per Android SDK Manager, la creazione di un nuovo progetto (Android Project), genera la struttura del programma:
\begin{itemize}
	\item \textbf{src} : codice java 
	\item \textbf{gen} : file auto generati per la gestione delle risorse
	\item \textbf{Android 2.2} (Libreria) : tutta la libreria di Android 
	\item \textbf{assets}: risorse che non vengono auto indicizzate in R
	\item \textbf{bin}: file binario
	\item \textbf{AndroidManifest.xml}
\end{itemize}

\section{Le componenti principali di un Applicativo Andorid}
\label{sec:android_main_components}
Lo sviluppo di programmi (Java) per applicativi Android � necessariamente vincolato dal fatto che l'interazione dell'utente avviene mediante lo schermo del dispositivo, la durata della batteria � limitata, la capacit� computazionale � ridotta ecc.
Gli sviluppatori di Android hanno creato un framework per sviluppare applicativi, che risolve la maggior parte dei problemi del programmatore. L'impostazione di base del framework � quella della programmazione ad eventi, con un meccanismo di callback (riferimento a un codice) asincrono. 
Le componenti principali sono:
\begin{enumerate}
	\item \textbf{Acitivity}: un'attivit� � la logica che gestisce una schermata singola che l'utente vede. Gli applicativi hanno di solito molteplici activity che permettono all'utente di navigare l'applicativo secondo la sua logica,
	\item \textbf{Intent}: messaggi asincroni inviati tra le componenti principali,
	\item \textbf{Service}: logica dell'applicativo,
	\item \textbf{Content Provider}: interfaccia per lo scambio di dati tra applicativi,
	\item \textbf{Broadcast Receiver}: metodo per gestire chiamate a livello di sistema in modo asincrono,
	\item \textbf{Application Context}: contesto in cui tutta l'applicazione esiste.
\end{enumerate}

\subsection{Activity}
Ogni applicativo Android ha una \textit{main activity}, che definisce la logica della schermata iniziale.
Nell'ottica di ottimizzare le risorse del dispositivo, le activity sono state progettate in modo da consumare il minimo.
Quando viene lanciata una activity, viene creato un processo Linux, viene allocato dello spazio per gli oggetti UI, costruire oggetti Java a partire dalle definizioni XML (inflation), posizionare oggetti sullo schermo. Per evitare di incorrere in questo costo ogni volta che si ricapita su una schermata, le activity sono state progettate per avere un ciclo di vita, gestito da un activity Manager. Quest'ultimo si occupa di creare, gestire e distruggere le activity, all'occorrenza.
Ogni activity attraversa i seguenti stati (vedi figura \ref{fig:Activity_lifecycle}):
\begin{enumerate}
	\item \textbf{Starting}: l'activity non esiste in memoria. I metodi della classe \texttt{Activity} che permettono di gestire l'evento di creazione di una activity sono \texttt{onCreate()},  \texttt{onStart()} ed \texttt{onResume()} tutti per andare nello stato Running. 
	\item \textbf{Running}: l'activity � sullo schermo e l'utente ci sta interagendo. In ogni dato istante di tempo, pu� esistere solo un'activity in questo stato. Tra tutte le activity, quella nello stato Running ha la massima priorit� per l'utilizzo della risorse, per minimizzare i tempi di risposta all'utente. Il metodo per gestire l'evento � \texttt{onPause} per andare nello stato Paused.
	\item \textbf{Paused}: l'activity � ancora visibile, ma l'utente non vi pu� interagire. Non � uno stato molto comune, perch� date le dimensioni ridotte dello schermo, generalmente le activity occupano tutto lo schermo o niente. Ad esempio quando appare una dialog box su una schermata, la schermata � nello stato Paused. Tutte le activity attraversano questo stato nel momento in cui vengono fermate. Queste activity sono tra quelle a priorit� pi� alta, perch� sono ancora visibili, e la transizione ad un'altra activity deve essere compiuto in modo fluido. Le callback dello stato sono \texttt{onResume()} per tornare nello stato Running e \texttt{onStop()} per andare nello stato Stopped.
	\item \textbf{Stopped}: un'activity si trova in questo stato se non � pi� visibile ma � ancora in memoria. Queste possono essere distrutte oppure tenute in memoria per essere ripristinate nello stato Running. La seconda operazione � molto meno costosa della creazione ex-novo di un'activity. Le callback di questo stato sono le stesse dello stato Starting ed il metodo \texttt{onDestroy()} per andare nello stato Destroyed. 
	\item \textbf{Destroyed}: l'activity viene rimossa dalla memoria, se l'Activity Manager decide che questa non verr� usata per abbastanza tempo da rendere pi� conveniente la ricreazione della stessa al suo trattenimento in memoria.
\end{enumerate}

\begin{figure}
	\centering
	\includegraphics[width=.9\textwidth]{imgs/ActivityLifeCycle.jpg} 
	\caption{Ciclo di vita di una Activity, cortesia di \cite{Gargenta_android}}
	\label{fig:Activity_lifecycle}
\end{figure}

\subsection{Intent}
Le Intent possono essere viste, come dice il nome, delle intenzioni di creare Activity che un mittente comunica.
Queste potrebbero essere usate da un'Activity per creare un'altra activity, oppure per far partire un servizio o per inviare un messaggio in broadcast. Questi possono essere espliciti se il mittente dichiara il ricevente, o impliciti se il mittente dichiara solo il tipo di ricevente. Nel secondo caso ci potrebbero essere dei riceventi in competizione per l'esecuzione del messaggio, ed il sistema lascia all'utente la scelta del esecutore.

\subsection{Servizi}
Questi non hanno un interfaccia utente ed il loro ciclo di vita o esecuzione � trasparente a chi utilizza il sistema. Il ciclo di vita di un servizio � molto semplice: 
inizialmente il servizio viene creato, ed il suo primo stato � detto Starting. Da qui le callback da usare per intercettare la transizione in Running sono \texttt{onCreate()} ed \texttt{onStart()}. Dallo stato Running con la callback \texttt{onDestroy()} il servizio va nell'ultimo stato in cui si pu� trovare: Destroyed.\\
I service che sono particolarmente impegnativi dal punto di vista computazionale dovrebbero essere eseguiti su un proprio thread, eseguibile in background, e non quello della UI, in modo da non rallentare l'interfaccia grafica.
\subsection{Content Provider}
Android, per ragioni di sicurezza, esegue ogni applicativo nella propria "`sandbox"' compartimento stagno, in modo da confinare i dati usati da un programma a quest'ultimo. Mediante gli Intent � possibile scambiare piccole quantit� di dati tra applicativi diversi, la condivisione di quantit� ingenti di dati persistenti viene fatta tramite i 
Content Provider. Per facilitare il compito questo componente aderisce all'interfaccia CRUD: il Content Provider � interfacciato ad una base di dati ed implementa i metodi \texttt{insert()}, \texttt{delete()}, \texttt{update()}, \texttt{query()}.

\begin{figure}
	\centering
	\includegraphics[width=.45\textwidth]{imgs/CRUD.jpg}
	\caption{ CRUD di un Content Provider, cortesia di \url{http://developer.android.com}}
	\label{fig:coInterfacciantent_provider}
\end{figure}

\section{Broadcast Receiver}
Implementazione del pattern Observer (tipo particolare del protocollo Publish/Subscribe) in cui c'� un servizio di prenotazione su un certo evento. Un programma si registra al servizio e nel momento in cui viene lanciato l'evento per il quale si � registrato, il codice viene lanciato. Il sistema operativo lancia eventi in broadcast in continuazione: il sistema � stato avviato, la batteria � scarica, un sms � in arrivo ecc. Ciascuno di questi eventi scatena il lancio dei programmi registrati, o per usare il nome del pattern, i programmi che osservavano l'evento.  
\section{Application Context}
\label{sec:application_context}
Il contesto di un'applicativo Android � l'ambiente in cui i processi con tutti le componenti vengono eseguiti. Il ciclo di vita di un contesto parte con la sua creazione al lancio dell'applicativo, e termina nel momento in cui questi viene terminato. Per avere un riferimento al contesto � sufficiente chiamare \texttt{Context.getApplicationContext()} oppure \texttt{Activity.getApplication()}


\section{Intefaccia Utenti (UI)}\TODO{Completare}
\url{http://developer.android.com/guide/topics/ui/index.html}
\subsection{Layout XML}
\subsection{Eventi di Input}

\subsection{Menu}
\label{sec:menu}
Per default ogni Activity ha un menu di opzioni o azioni, a cui l'utente pu� accedere premendo un taso fisicamente disegnato sullo schermo. 
\subsection{Barra delle Azioni}
\subsection{Dialoghi}

%\myChapter{Ringraziamenti}
prof Angelo Sabatini per avermi accolto a braccia aperte nel suo team ed avermi sostenuto nei momenti pi� difficili
Andrea Mannini per essere stato sempre presente durante l'anno di lavoro
Ing Vincenzo Genovese
Charlie Roche per essere il mio mentore di programmazione, basket, freesby, bigliardo, cinema e amico.
Tommaso Baviera 
Andrea Guitto
Barbara Marini

Sara Porporato $\heartsuit \heartsuit \heartsuit$


%%--------------------------------------------------------------
%	Extras
%--------------------------------------------------------------
%\makeatletter \renewcommand{\@dotsep}{10000} \makeatother
%\listoftables
%\makeatletter \renewcommand{\@dotsep}{4.5} \makeatother
%\listoffigures
%--------------------------------------------------------------
\bibliographystyle{IEEEtran}
\bibliography{biblio}
\end{document}
\------------------------------------	--------------------------
	