\chapter{Introduzione}
%\myChapter{Introduzione}

%-------------------	Descrizione del problema -------------------------------
Ipotesi: \textit{� possibile costruire un sistema intelligente e portatile in grado di riconoscere e monitorare i movimenti di un individuo e di fornirne informazioni a riguardo in tempo reale(vedi \ref{sec:real_time_systems}).
Per ridurre la complessit� del problema di riconoscimento di un movimento generico, prendiamo in considerazione solo gli arti inferiori nella deambulazione entro un intervallo di velocit� e pendenza del terreno}.\\

A questo punto si tratta di riconoscere le 8 fasi della deambulazione normale studiate dalla Chinesiologia (vedi capitolo \ref{cap:chinesiologia}), vale a dire il problema della \emph{segmentazione automatica in tempo reale della deambulazione umana}.

%------------------  Motivazioni del problema Perch� si pone?--------------------------------
La soluzione del problema qui menzionato, avrebbe risvolti immediati in Medicina per la riabilitazione per la diagnosi e/o assistenza a persone con problemi di deambulazione, nella Robotica e Computer Grafica per la emulazione/simulazione della deambulazione umana, nel mondo dello sport agonistico per l'apprendimento di specifiche tecniche motorie ed in innumerevoli altri settori.

%---------------------� gia stato affrontato? 
Il problema della segmentazione della deambulazione � stato ampiamente affrontato in letteratura e con svariate combinazioni di materiali e metodi. 

Per quanto riguarda i materiali, sono state proposte soluzioni parziali che si basano semplicemente sull'osservazione diretta di un fisiatra oppure su strumenti ottici, basati su sistemi di telecamere e marker (segnali riconoscibili dalle telecamere); strumenti inerziali, basati su sensori fisici: accelerometri, giroscopi, elettromagnetometri ecc. Questi ultimi sono stati posizionati in svariate parti del corpo ed in diverse combinazioni. 

Per quanti riguarda i metodi, gli ambiti scientifici nei quali viene affrontato il problema spaziano dalla Cinematica per lo studio delle forze, all'Analisi Matematica per lo studio di curve, alle Macchine a stati finiti per lo studio di sequenze temporali, e pi� di recente all'Apprendimento Automatico per la capacit� di astrarre sulle differenze individuali nel compiere qualsiasi azione.

% -------------------------------� stato gia stato risolto? 
Nonostante il vasto numero di lavori, non � ancora stata data una soluzione soddisfacente al problema. 
Tutti i metodi proposti peccano di dipendenza dagli strumenti che usano, vale a dire che variando questi ultimi, variano le prestazioni dei metodi e di dipendenza dai soggetti sui quali vengono fatti gli esperimenti.
Questo significa che le soluzioni sin ora proposte sono fatte su misura per specifici casi.

%------------------------- Come voglio affrontare il problema? 
Il nostro obbiettivo � creare un metodo che dipenda il meno possibile dai materiali, che tenga conto della variabilit� interpersonale e che abbia una complessit� computazionale abbastanza bassa da poter essere usato su uno Smartphone. 

Le nostre scelte sono quindi cadute su pochi sensori semplici per la raccolta dei dati e sulle HMM (Hidden Markov Models vedi appendice \ref{sec:HMM}) per gestirne la componente temporale, creando cos� un sistema a efficace basso costo, portatile, semplice da utilizzare e .