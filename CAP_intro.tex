\chapter{Introduzione}
%\myChapter{Introduzione}

%-------------------	Descrizione del problema -------------------------------
Ipotesi: \textit{� possibile costruire un sistema intelligente e portatile in grado di riconoscere (classificare) e analizzare i movimenti di un individuo e di fornirne in linea (vedi Appendice \ref{sec:real_time_sys}) informazioni a riguardo}.\\

Il problema del riconoscimento (classificazione) di una generica attivit� motoria umana � irrisolto ed estremamente complesso visto il numero esorbitante di parametri coinvolti.
In questo lavoro viene affrontata l'analisi di una singola attivit� nota: la deambulazione entro un intervallo di velocit� e pendenza del terreno.

Nello specifico il problema � quello dell'individuazione in linea delle tempistiche di eventi che costituiscono una deambulazione normale come studiato dalla Chinesiologia (vedi Capitolo \ref{cap:chinesiologia}). Tale problema � noto come \emph{problema della segmentazione automatica ed in linea della deambulazione umana}.\\

%------------------  Motivazioni del problema Perch� si pone?--------------------------------
La soluzione del problema della segmentazione avrebbe risvolti immediati nella Medicina riabilitativa per la diagnosi e/o assistenza a persone con problemi di deambulazione, nella Robotica e Computer Grafica per la emulazione/simulazione della deambulazione umana, nel mondo dello sport agonistico per l'apprendimento di specifiche tecniche motorie.\\

%---------------------� gia stato affrontato? 
Il problema della segmentazione � stato ampiamente affrontato nella letteratura scientifica (letteratura biomedica, biomeccanica, ingegneria medica) con svariate combinazioni di materiali e metodi. 

Per quanto riguarda i materiali, sono state proposte soluzioni basate sull'osservazione diretta di un fisiatra; basate sulla stereofotogrammetria, sistemi di telecamere ad emissione di luce infrarossa e marcatori riflettenti; basate su strumenti inerziali, sensori fisici: accelerometri, giroscopi, elettromagnetometri. 
\begin{figure}
	\centering
	\includegraphics[width=1\textwidth]{imgs/sensorOnBodyLocations.jpg}
	\caption{Schema riassuntivo dei posizionamenti di diversi sensori su diverse parti del corpo, usati in letteratura. Nella Figura, $a$ rappresenta un accelerometro, $\omega$ rappresenta un giroscopio, $B$ rappresenta un magnetometro. Le frecce accanto alle lettere rappresentano le dimensioni dei rispettivi sensori, quindi una frecce sotto un simbolo significa che il sensore � monoassiale, due significa biassiale e tre (la terza freccia � uscente, quindi rappresentata come un cerchio con un punto al suo centro) triassiale. Le `capsule` grigie che racchiudono uno o pi� sensori sono dette Unit� Sensoriali. Le lettere A (\textit{abdomen}, addome), T (\textit{thigh}, coscia), S (\textit{shank}, stinco) ed F (\textit{foot}, piede) rappresentano le parti del corpo sui quali si trovano. 
	 Figura adattata da \cite{gait_event_detection_analysis}}
	\label{fig:sensorOnBodyLocations}
\end{figure}
Questi ultimi sono stati usati in diverse combinazioni, numero e disposizione sul corpo (vedi Figura \ref{fig:sensorOnBodyLocations}).\\ 

Per quanto riguarda i metodi, sono state proposte soluzioni 
di tipo cinematico basato sullo studio delle forze che agiscono sul corpo nella deambulazione; di tipo analitico (studio di funzioni e curve) basato sull'analisi funzionale dei segnali di sensori inerziali; 
di tipo modellistico basate sugli Automi a Stati Finiti per lo studio di sequenze temporali; 
pi� di recente di tipo informatico-statistico basati sull'Apprendimento Automatico (Reti Neurali, Logica Fuzzy) per la capacit� di astrarre sulle variazioni dei singoli individui \cite{gait_event_detection_analysis}.

% -------------------------------� stato gia stato risolto? 
Nonostante il vasto numero di lavori, non � ancora stata data una soluzione soddisfacente al problema. 
Tutti i metodi proposti peccano di dipendenza dai soggetti per i quali quali vengono creati, vale a dire che variando questi ultimi, variano le prestazioni dei metodi. Questo � indice di bassa capacit� di generalizzazione dei metodi.\\

%------------------------- Come voglio affrontare il problema? 
La scelta dell'utilizzo delle HMM � giustificata dai seguenti motivi. 
Le HMM (vedi Appendice \ref{cap:hmm}) sono uno strumento stocastico di riconoscimento di schemi (\textit{Stochastic Pattern Recognition}) usato in campi come il riconoscimento vocale \cite{tutorial_hmm_application_speech_recognition} e lo studio della visione artificiale per il riconoscimento gestuale \cite{HMM_gesture_recognition}. Uno studio dimostra il potenziale delle HMM per la segmentazione della deambulazione equina \cite{stride_segmentation_technique_hmm}. \\

Inoltre vi sono studi che usano le HMM come struttura gerarchica per affrontare il problema della classificazione di attivit� umane \cite{human_physical_activity_classification_ml, spatio_temporal_params_gait_gyr}. Per sviluppi futuri, del lavoro qui presentato, nella direzione della risoluzione del problema della classificazione, gli studi citati sono a favore dell'utilizzo delle HMM.\\

Il seguente lavoro di tesi � strutturato in tre parti. Nella Parte I, nel Capitolo \ref{cap:chinesiologia}, vengono fornite le nozioni di base sulla deambulazione umana, nel Capitolo \ref{cap:statoArte} viene fatto un sunto dei lavori maggiormente rilevanti tra quelli che si trovano in letteratura sul problema affrontato. Nella Parte II viene presentato il lavoro fatto in questa tesi comprendente una modellazione della deambulazione con l'uso di strumenti di Apprendimento Automatico, riportato nel Capitolo \ref{cap:modellazioneDeambulazione}, sviluppo di un applicazione per \textit{Smartphone} Android\tm, riportato nel Capitolo \ref{cap:androidApp}, una valutazione del sistema creato mediante un confronto con un sistema commerciale GPS ed i risultati ottenuti, riportato nel Capitolo \ref{cap:valutazioneRisultati}, infine le conclusioni sul lavoro fatto e considerazioni in merito a sviluppi futuri, riportate nel Capitolo \ref{cap:conclusioni}. Nella Parte III viene presentato un breve resoconto su quattro argomenti importanti per la comprensione del lavoro di tesi. Una breve presentazione delle HMM (vedi Appendice \ref{cap:hmm}) � necessaria per la comprensione della parte di modellazione. Un accenno ai sensori (vedi Appendice \ref{sec:sensori}) � utile per la comprensione dello Stato dell'arte, inoltre qui viene fornita una scheda tecnica della IMU (Unit� di Misura Inerziale) usata per l'acquisizione di dati. Essendo la piattaforma Android\tm una novit� in grossa espansione, � stata ritenuta necessaria una panoramica sia sul sistema operativo Android-OS\tm che sull'ambiente di sviluppo Android SDK\tm (vedi Appendice \ref{app:android}).
Infine viene presentata una brevissima spiegazione di tre concetti fondamentali spesso oggetto di confusione: tempo reale, in linea e latenza (vedi Appendice \ref{sec:real_time_sys}). 
